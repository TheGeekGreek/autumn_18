\section*{Introduction}
Classical mechanics deals with differential equations originating from extremals of \emph{functionals}, i.e. functions defined on an infinite-dimensional function space. The study of such extremality properties of functionals is known as the \emph{calculus of variations}. To illustrate this fundamental principle, let us consider the \emph{variational formulation} of second order elliptic operators in divergence form based on \cite[167--168]{struwe:fa:2014}.\\ 
For convention, unless explicietly stated otherwise, we will assume that all manifolds are smooth, that is of class $C^\infty$, finite-dimensional, Hausdorff and paracompact with at most countably many connected components.\\ 
Let $n \in \mathbb{N}$, $n \geq 1$, and $\Omega \subseteq \subseteq \mathbb{R}^n$ such that $\wbar{\Omega}$ is a smooth manifold with boundary. Moreover, let $H^1_0(\Omega)$ denote the Sobolev space $W^{1,2}_0(\Omega)$ with inner product
\begin{equation*}
	\langle u,v \rangle_{H^1_0(\Omega)} = \int_\Omega uv + \int_\Omega \nabla u \nabla v.
\end{equation*}
Suppose $a^{ij} \in C^\infty(\wbar{\Omega})$ symmetric, $f \in C^\infty(\wbar{\Omega})$ and consider the second order homogenous Dirichlet problem
\begin{equation}
	\label{eq:homogeneous_Dirichlet_problem}
	\ccases{
			\displaystyle -\frac{\partial}{\partial x^j}\del[3]{a^{ij}\frac{\partial u}{\partial x^i}} = f & \text{in } \Omega,\\
			u = 0 & \text{on } \partial \Omega,
		}
\end{equation}
Suppose $u \in C^\infty(\wbar{\Omega})$ solves (\ref{eq:homogeneous_Dirichlet_problem}). Then integration by parts (see \cite[436]{lee:smooth_manifolds:2013}) yields 
\begin{equation*}
	\int_\Omega f v = -\int_\Omega \frac{\partial}{\partial x^j}\del[3]{a^{ij}\frac{\partial u}{\partial x^i}}v = -\int_\Omega \mathrm{div}(X)v = \int_\Omega \langle X, \nabla v\rangle = \int_\Omega a^{ij} \frac{\partial u}{\partial x^i}\frac{\partial v}{\partial x^j}
\end{equation*}
\noindent for any $v \in C^\infty_c(\Omega)$, where $X := \del[1]{a^{ij}\frac{\partial u}{\partial x^i}}_j$. Thus we say that $u \in H^1_0(\Omega)$ is a \emph{weak solution} of (\ref{eq:homogeneous_Dirichlet_problem}) iff
\begin{equation*}
	\forall v \in C^\infty_c(\Omega): \> \int_\Omega a^{ij} \frac{\partial u}{\partial x^i}\frac{\partial v}{\partial x^j} = \int_\Omega fv.
\end{equation*}
If $(a^{ij})_{ij}$ is \emph{uniformly elliptic}, i.e. there exists $\lambda > 0$ such that
\begin{equation*}
	\forall x \in \Omega\forall \xi \in \mathbb{R}^n : \> a^{ij}(x)\xi_i\xi_j \geq \lambda \abs{\xi}^2,	
\end{equation*}
\noindent then (\ref{eq:homogeneous_Dirichlet_problem}) admits a unique weak solution $u \in H^1_0(\Omega)$ (in fact $u \in C^\infty(\Omega)$ using \emph{regularity theory}, for more details see \cite[175]{struwe:fa:2014}). Indeed, observe that 
\begin{equation*}
	\langle \cdot,\cdot \rangle_a : H^1_0(\Omega) \times H^1_0(\Omega) \to \mathbb{R}
\end{equation*}
\noindent defined by
\begin{equation}
	\label{eq:Sobolev_inner_product}
	\langle u, v \rangle_a := \int_\Omega a^{ij} \frac{\partial u}{\partial x^i}\frac{\partial v}{\partial x^j}
\end{equation}
\noindent is an inner product on $H^1_0(\Omega)$ with induced norm equivalent to the standard one on $H^1_0(\Omega)$ due to Poincar\'e's inequality \cite[107]{struwe:fa:2014}. Applying the Riesz Representation theorem \cite[49--50]{struwe:fa:2014} yields the result. Moreover, this solution can be characterized by a \emph{variational principle}, i.e. if we define the \emph{energy functional} $E : H^1_0(\Omega) \to \mathbb{R}$
\begin{equation*}
	E(v) := \frac{1}{2} \norm{v}^2_a - \int_\Omega fv,
\end{equation*}
\noindent for any $v \in H^1_0(\Omega)$, where $\norm{\cdot}_a$ denotes the norm induced by the inner product (\ref{eq:Sobolev_inner_product}), then $u \in H^1_0(\Omega)$ solves (\ref{eq:homogeneous_Dirichlet_problem}) if and only if
\begin{equation}
	\label{eq:variational_formulation}
	E(u) = \inf_{v \in H^1_0(\Omega)}E(v).
\end{equation}
Indeed, suppose $u \in H^1_0(\Omega)$ is a solution of (\ref{eq:homogeneous_Dirichlet_problem}). Let $v \in H^1_0(\Omega)$. Then $u = v + w$ for $w := u - v \in H^1_0(\Omega)$ and we compute
\begin{equation*}
	E(v) = E(u + w) = \frac{1}{2}\norm{u}^2_a + \langle u,w \rangle_a + \frac{1}{2}\norm{w}^2_a - \int_\Omega f(u + w) = E(u) + \frac{1}{2}\norm{w}^2_a \geq E(u)
\end{equation*}
\noindent with equality if and only if $u = v$ a.e. Conversly, suppose the infimum is attained by some $u \in H^1_0(\Omega)$. Thus by elementary calculus
\begin{equation}
	\label{eq:derivative}
	0 = \frac{d}{dt}\bigg\vert_{t = 0} E(u + tv) = \langle u, v \rangle_a - \int_\Omega f v
\end{equation}
\noindent for all $v \in H^1_0(\Omega)$.\\
Suppose now that $u \in C^\infty(\wbar{\Omega})$ with $u\vert_{\partial \Omega} = 0$ solves the variational formulation (\ref{eq:variational_formulation}). Then again integration by parts yields
\begin{equation*}
	\langle u, v \rangle_a - \int_\Omega f v = -\int_\Omega \mathrm{div}(X) v - \int_\Omega fv = \int_\Omega \del[3]{-\frac{\partial}{\partial x^j}\del[3]{a^{ij}\frac{\partial u}{\partial x^i}} - f} v
\end{equation*}
\noindent for all $v \in C^\infty_c(\Omega)$ and where $X := \del[1]{a^{ij}\frac{\partial u}{\partial x^i}}_j$. Hence (\ref{eq:derivative}) implies 
\begin{equation*}
	\forall v \in C^\infty_c(\Omega): \> \int_\Omega \del[3]{-\frac{\partial}{\partial x^j}\del[3]{a^{ij}\frac{\partial u}{\partial x^i}} - f} v = 0.
\end{equation*}

We might expect that this implies 
\begin{equation*}
	-\frac{\partial}{\partial x^j}\del[3]{a^{ij}\frac{\partial u}{\partial x^i}} = f.
\end{equation*}

That this is indeed the case, is guaranteed by a foundational result in the \emph{calculus of variations} (therefore the name).

\begin{proposition}[Fundamental Lemma of Calculus of Variations]
	\label{prop:fundamental_lemma}
	Let $\Omega \subseteq \mathbb{R}^n$ open and $f \in L^1_{\mathrm{loc}}(\Omega)$. If
	\begin{equation*}
		\forall \varphi \in C^\infty_c(\Omega): \> \int_\Omega f\varphi = 0,
	\end{equation*}
	\noindent then $f = 0$ a.e.
\end{proposition}

\begin{proof}
	See \cite[40]{struwe:fa:2014}.
\end{proof}

Thus we recovered a second order partial differential equation from the variational formulation. In fact, this is exactly the boundary value problem (\ref{eq:homogeneous_Dirichlet_problem}) from the beginning of our exposition. This technique, and in particular the fundamental lemma of calculus of variations \ref{prop:fundamental_lemma} will play an important role in our treatment of classical mechanics. However, since we are concerned with smooth manifolds only, we use a version of the fundamental lemma of calculus of variations \ref{prop:fundamental_lemma}, which is fairly easy to prove and hence really deserves the terminology ``lemma''.

\begin{lemma}[Fundamental Lemma of Calculus of Variations, Smooth Version]
	\label{lem:fundamental_lemma}
	Let $\Omega \subseteq \mathbb{R}^n$ open and $f \in C^\infty(\Omega)$. If
	\begin{equation*}
		\forall \varphi \in C^\infty_c(\Omega): \> \int_\Omega f\varphi = 0,
	\end{equation*}
	\noindent then $f = 0$.
\end{lemma}

\begin{proof}
	Towards a contradiction, assume that $f \neq 0$ on $\Omega$. Thus there exists $x_0 \in \Omega$, such that $f(x_0) \neq 0$. Without loss of generality, we may assume that $f(x_0) > 0$, since otherwise, consider $-f$ instead of $f$.	The smoothness of $f$ implies the continuity of $f$ on $\Omega$. Thus there exists $\delta > 0$, such that $f(x) \in B_{f(x_0)/2}\del[1]{f(x_0)}$ holds for all $x \in B_\delta(x_0)$ or equivalently, $f(x) > f(x_0)/2 > 0$ for all $x \in B_\delta(x_0)$. By lemma $2.22$ \cite[42]{lee:smooth_manifolds:2013}, there exists a smooth bump function $\varphi$ supported in $B_\delta(x_0)$ and $\varphi = 1$ on $\wbar{B}_{\delta/2}(x_0)$. In particular, $\varphi \in C^\infty_c(\Omega)$. Therefore we have
	\begin{equation*}
		0 = \int_\Omega f \varphi = \int_{B_\delta(x_0)} f \varphi \geq \int_{B_{\delta/2}(x_0)} f \varphi > \frac{1}{2}f(x_0) \abs[0]{B_{\delta/2}(x_0)} > 0,
	\end{equation*}
	\noindent which is a contradiction.
\end{proof}

\begin{exercise}\footnote{This is exercise $1.2. (b)$ from exercise sheet $1$ of the course \emph{Functional Analysis II} taught by \emph{Prof. Dr. A. Carlotto} at ETHZ in the spring of 2018, which can be found \href{https://metaphor.ethz.ch/x/2018/fs/401-3462-00L/ex/Problems01-FAII.pdf}{here}.}
	Let $\Omega \subseteq \subseteq \mathbb{R}^n$, $2 \leq p < \infty$ and define $\mathcal{B} := \cbr[0]{v \in C^\infty(\wbar{\Omega}) : v\vert_{\partial\Omega} = 0}$. Moreover, define $E_p : \mathcal{B} \to \mathbb{R}$ by $E_p(v) := \int_\Omega \abs[0]{\nabla v}^p$. Derive the partial differential equation satisfied by minimizers $u \in \mathcal{B}$ of the variational problem $E(u) = \inf_{v \in \mathcal{B}}E(v)$.	
\end{exercise}
