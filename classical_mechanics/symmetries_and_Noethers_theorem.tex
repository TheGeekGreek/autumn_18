\section*{Symmetries and Noether's Theorem}
Solving the Euler-Lagrange equations \ref{thm:EL_equations} in general is a very difficult task. Thus we have to make certain assumptions for deriving general results.

\begin{definition}[Closed Lagrangian System]
	A Lagrangian system $(M,L)$ is said to be \bld{closed}, iff
	\begin{equation*}
		\frac{\partial L}{\partial t} = 0
	\end{equation*}
	\noindent with respect to every chart $U$ of $M$.
\end{definition}

\begin{definition}[Integral of Motion]
	Let $(M,L)$ be a Lagrangian system. An \bld{integral of motion} is defined to be a morphism $I \in C^\infty(TM)$ such that
	\begin{equation*}
		\frac{d}{dt}I\del[1]{\gamma(t),\dot{\gamma}(t)} = 0
	\end{equation*}
	\noindent holds for every extremal $\gamma$ of the action functional.
\end{definition}

Let $(M,L)$ be a Lagrangian system and $(U,q)$ a chart. Denote by $(q,\dot{q})$ the standard coordinates on $TM$ and consider the function $E \in C^\infty(TU \times \mathbb{R})$ defined by
\begin{equation}
	\label{eq:energy}
	E(q,\dot{q},t) := \dot{q}^i \frac{\partial L}{\partial \dot{q}^i}(q,\dot{q},t) - L(q,\dot{q},t).
\end{equation}

\begin{definition}[Energy]
	
\end{definition}
