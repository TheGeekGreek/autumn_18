\section*{Symmetries and Noether's Theorem}
Solving the Euler-Lagrange equations \ref{thm:EL_equations} in general is a very difficult task. Thus we have to make certain assumptions for deriving general results.

\begin{definition}[Integral of Motion]
	Let $(M,L)$ be a Lagrangian system. An \bld{integral of motion} is defined to be a morphism $I \in C^\infty(TM)$ such that
	\begin{equation*}
		\frac{d}{dt}I\del[1]{\gamma(t),\dot{\gamma}(t)} = 0
	\end{equation*}
	\noindent holds for every extremal $\gamma$ of the action functional.
\end{definition}

\begin{definition}[Energy]
	
\end{definition}

\begin{definition}[Closed Lagrangian System]
	A Lagrangian system $(M,L)$ is said to be \bld{closed}, iff
	\begin{equation*}
		\frac{\partial L}{\partial t} = 0
	\end{equation*}
	\noindent with respect to every chart $U$ of $M$.
\end{definition}
