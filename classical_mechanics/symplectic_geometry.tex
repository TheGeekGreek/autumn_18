\section*{Symplectic Geometry}
A profound difference between the tangent bundle $TM$ and the cotangent bundle $T^*M$ of a smooth manifold $M$ is that on the latter there exists a natural $1$-form, the tautological form $\alpha$ defined in definition \ref{def:tautological_form}.  

\subsection*{The Category of Symplectic Manifolds}
Recall that a form $\omega$ on a smooth manifold $M$ is said to be \emph{closed}, iff $d\omega = 0$.

\begin{definition}[Symplectic Manifold]
	A \bld{symplectic manifold} is a tuple $(M,\omega)$ consisting of a smooth manifold $M$ and a closed nondegenerate $2$-form $\omega \in \Omega^2(M)$, called a \bld{symplectic form on $M$}.
\end{definition}

\begin{example}[The Cotangent Bundle]
	Let $M$ be a smooth manifold and consider the tautological form $\alpha \in \Omega^1(T^*M)$ defined by $\alpha := \xi_i dx^i$ on a chart $\del[1]{T^*U, (x^i,\xi^i)}$ on $T^*M$. Define $\omega \in \Omega^2(T^*M)$ by $\omega := -d\alpha$. It is immediate that $\omega$ is closed since $d\omega = -(d \circ d)(\alpha) = 0$. Moreover, we compute locally
	\begin{equation*}
		\omega = -d(\xi_idx^i) = -\frac{\partial \xi_i}{\partial x^j} dx^j \wedge dx^i - \frac{\partial \xi_i}{\partial \xi_j} d\xi_j \wedge dx^i = \delta^j_i dx^i \wedge d\xi_j = \sum_i dx^i \wedge d\xi_i.
	\end{equation*}
	Thus $\omega$ is nondegenerate. 
\end{example}

\begin{definition}
	\label{def:symplectic_morphisms}
	A morphism $F : (M,\omega) \to (\wtilde{M},\wtilde{\omega})$ between two symplectic manifolds $(M,\omega)$ and $(\wtilde{M},\wtilde{\omega})$ is defined to be a morphism $F \in C^\infty(M,\wtilde{M})$ such that $F^*\wtilde{\omega} = \omega$. 
\end{definition}

\begin{exercise}
	Consider as objects symplectic manifolds and as morphisms the ones from definition \ref{def:symplectic_morphisms}. Show that they do form a category, the \bld{category of symplectic manifolds}.
\end{exercise}

\begin{definition}[Symplectomorphism]
	A \bld{symplectomorphism} is defined to be an isomorphism in the category of symplectic manifolds.
\end{definition}

\subsection*{The Tangent-Cotangent Bundle Isomorphism}
One very important feature of a symplectic manifold $(M,\omega)$ is, that there is a canonical identification of the tangent bundle $TM$ and the cotangent bundle $T^*M$. 

%Input tensor fields
\begin{lemma}[{Vector Bundle Chart Lemma \cite[253]{lee:smooth_manifolds:2013}}]
	\label{lem:vector_bundle_chart_lemma}
	Let $M$ be a smooth manifold, $k \in \mathbb{N}$ and suppose that for all $x \in M$ we are given a real vector space $E_x$ of dimension $k$. Let $E := \coprod_{x \in M} E_x$ and let $\pi : E \to M$ be given by $\pi(x,v) := x$. Moreover, suppose that we are given the following data:
	\begin{enumerate}[label = \textup{(\roman*)},leftmargin=*]
		\item An open cover $(U_\alpha)_{\alpha \in A}$ of $M$.
		\item For all $\alpha \in A$ a bijection $\Phi_\alpha : \pi^{-1}(U_\alpha) \to U_\alpha \times \mathbb{R}^k$ such that the restriction $\Phi_\alpha\vert_{E_x} : E_x \to \cbr{x} \times \mathbb{R}^k \cong \mathbb{R}^k$ is an isomorphism of vector spaces for all $x \in M$.
		\item For all $\alpha, \beta \in A$ with $U_\alpha \cap U_\beta \neq \varnothing$, a smooth mapping $\tau_{\alpha\beta} : U_\alpha \cap U_\beta \to \GL(k,\mathbb{R})$  such that the mapping $\Phi_\alpha \circ \Phi_\beta^{-1} : (U_\alpha \cap U_\beta) \times \mathbb{R}^k \to (U_\alpha \cap U_\beta) \times \mathbb{R}^k$ is of the form $\Phi_\alpha \circ \Phi_\beta^{-1}(x,v) = \del[1]{x,\tau_{\alpha\beta}(x)v}$. 
	\end{enumerate}
	Then $E$ admits a unique topology and a smooth structure making it into a smooth manifold and a smooth vector bundle $\pi : E \to M$ of rank $k$ with local trivializations $(U_\alpha,\Phi_\alpha)_{\alpha \in A}$.
\end{lemma}

Let $M^n$ be a smooth manifold and let $k,l \in \mathbb{N}$. For all $x \in M$ define the space of \bld{mixed tensors of type $(k,l)$ on $T_xM$} by
\begin{equation*}
	T^{(k,l)}(T_xM) := \undercbrace{T_xM \otimes \dots \otimes T_xM}_{k} \otimes \undercbrace{T^*_xM \otimes \dots \otimes T^*_xM}_{l}.
\end{equation*}
By proposition 12.10 \cite[311]{lee:smooth_manifolds:2013} we have that 
\begin{equation*}
	T^{(k,l)}(T_xM) \cong L\del[1]{\undercbrace{T^*_xM, \dots, T^*_xM}_{k},\undercbrace{T_xM,\dots,T_xM}_{l};\mathbb{R}}
\end{equation*}
\noindent since $(T^*_xM)^* \cong T_xM$ canonically ($T_xM$ is finite-dimensional) where the latter denotes the space of all $\mathbb{R}$-valued multilinear forms on
\begin{equation*}
	\undercbrace{T^*_xM \times \dots \times T^*_xM}_{k} \times \undercbrace{T_xM \times \dots \times T_xM}_{l}.
\end{equation*}
We will always think of mixed tensors as multilinear forms. Let $(U,x^i)$ be a chart about $x$. Then using corollary 12.12 \cite[313]{lee:smooth_manifolds:2013} we get that a basis for $T^{(k,l)}(T_xM)$ is given by all elements 
\begin{equation*}
	\frac{\partial}{\partial x^{i_1}}\bigg\vert_x \otimes \dots \otimes \frac{\partial}{\partial x^{i_k}}\bigg\vert_x \otimes dx^{j_1}\vert_x \otimes \dots \otimes dx^{j_l}\vert_x
\end{equation*}
\noindent for all $1 \leq i_1,\dots,i_k,j_1,\dots,j_l \leq n$. Consequently, $\dim T^{(k,l)}(T_xM) = n^{k + l}$ and a particular tensor $A \in T^{(k,l)}(T_xM)$ expressed in this basis is given by 
\begin{equation}
	\label{eq:tensor_expression_basis}
	A = A^{i_1\dots i_k}_{j_1\dots j_l}\frac{\partial}{\partial x^{i_1}}\bigg\vert_x \otimes \dots \otimes \frac{\partial}{\partial x^{i_k}}\bigg\vert_x \otimes dx^{j_1}\vert_x \otimes \dots \otimes dx^{j_l}\vert_x
\end{equation}
\noindent where
\begin{equation}
	\label{eq:tensor_components}
	A^{i_1\dots i_k}_{j_1\dots j_l} := A\del[3]{dx^{i_1}\vert_x,\dots,dx^{i_k}\vert_x,\frac{\partial}{\partial x^{j_1}}\bigg\vert_x,\dots\frac{\partial}{\partial x^{j_l}}\bigg\vert_x}.
\end{equation}
 Next we want to ``glue'' together the different spaces of mixed tensors.

\begin{proposition}
	\label{prop:tensor_bundle}
	Let $M$ be a smooth manifold and let $k,l \in \mathbb{N}$. Then
	\begin{equation*}
		T^{(k,l)}TM := \coprod_{x \in M} T^{(k,l)}(T_xM)
	\end{equation*}
	\noindent admits a unique topology and a smooth structure making it into a smooth manifold and a smooth vector bundle $\pi : T^{(k,l)}TM \to M$ of rank $n^{k + l}$. This smooth vector bundle is called the \bld{bundle of mixed tensors of type $(k,l)$ on $M$}. 
\end{proposition}

\begin{proof}
	This is an application of the vector bundle chart lemma \ref{lem:vector_bundle_chart_lemma}. For all $x \in M$ define $E_x := T^{(k,l)}(T_xM)$. By the preceeding discussion, $\dim E_x = n^{k + l}$. Let $(U_\alpha,\varphi_\alpha)_{\alpha \in A}$ denote the smooth structure on $M$. Then clearly $(U_\alpha)_{\alpha \in A}$ is an open cover for $M$. For each $\alpha \in A$, define
\begin{equation*}
	\Phi_\alpha:\ccases{
		\pi^{-1}(U_\alpha) \to U_\alpha \times \mathbb{R}^{n^{k + l}}\\
		(x,A) \mapsto \del[1]{x,(A^{i_1\dots i_k}_{j_1\dots j_l})}
	}
\end{equation*}
\noindent where we expressed $A$ as in (\ref{eq:tensor_expression_basis}). Observe, that this map strongly depends on the coordinate functions. Clearly, the inverse is given by
\begin{equation*}
	\Phi^{-1}_\alpha:\ccases{
		U_\alpha \times\mathbb{R}^{n^{k + l}} \to \pi^{-1}(U_\alpha)\\
		\del[1]{x,(A^{i_1\dots i_k}_{j_1\dots j_l})} \mapsto (x,A)
	}.
\end{equation*}
	Hence each $\Phi_\alpha$ is bijective. Now we have to check, that $\Phi_\alpha\vert_{E_x}$ is an isomorphism for all $x \in M$. By elementary linear algebra it is enough to show that $\Phi_\alpha$ is linear. So let $\lambda \in \mathbb{R}$ and $A,B \in E_x$. Then
	\begin{align*}
		\Phi_\alpha\vert_{E_x}(x,A + \lambda B) &= \del[1]{x,(A + \lambda B)^{i_1\dots i_k}_{j_1\dots j_l})}\\
		&= \del[1]{x,(A^{i_1\dots i_k}_{j_1\dots j_l}) + \lambda (B^{i_1\dots i_k}_{j_1\dots j_l})}\\
		&= \Phi_\alpha\vert_{E_x}(x,A) + \lambda\Phi_\alpha\vert_{E_x}(x,B).
	\end{align*}
	Lastly, let $\alpha, \beta \in A$ such that $U_\alpha \cap U_\beta \neq \varnothing$ and coordinates $(x^i_\alpha)$ and $(x^i_\beta)$, respectively. Then for $x \in U_\alpha \cap U_\beta$ we have that
	\begin{equation*}
		\frac{\partial}{\partial x_\alpha^i}\bigg\vert_x = \frac{\partial x^j_\beta}{\partial x_\alpha^i}(x)\frac{\partial}{\partial x_\beta^j}\bigg\vert_x \qquad \text{and} \qquad dx_\alpha^i\vert_x = \frac{\partial x_\alpha^i}{\partial x_\beta^j}(x)dx^j_\beta\vert_x.
	\end{equation*}
	So if $A^{i_1\dots i_k}_{j_1\dots j_l}$ are coordinates of a mixed tensor with respect to the basis induced by $(x^i_\alpha)$, we compute
	\begin{align*}
		A^{i_1\dots i_k}_{j_1\dots j_l} &= A\del[3]{dx_\alpha^{i_1}\vert_x,\dots,dx_\alpha^{i_k}\vert_x,\frac{\partial}{\partial x_\alpha^{j_1}}\bigg\vert_x,\dots\frac{\partial}{\partial x_\alpha^{j_l}}\bigg\vert_x}\\
		&= \frac{\partial x_\alpha^{i_1}}{\partial x^{p_1}_\beta}(x)\cdots\frac{\partial x_\alpha^{i_k}}{\partial x_\beta^{p_k}}(x)\frac{\partial x^{q_1}_\beta}{\partial x_\alpha^{j_1}}(x) \cdots \frac{\partial x^{q_l}_\beta}{\partial x_\alpha^{j_l}}(x)A^{p_1\dots p_k}_{q_1\dots q_l}
	\end{align*}
	Thus define $\tau_{\alpha\beta}: U_\alpha \cap U_\beta \to \mathrm{GL}(n^{k + l},\mathbb{R})$ by
	\begin{equation*}
		\tau_{\alpha\beta}(x) := \del[4]{\frac{\partial x_\alpha^{i_1}}{\partial x^{p_1}_\beta}(x)\cdots\frac{\partial x_\alpha^{i_k}}{\partial x_\beta^{p_k}}(x)\frac{\partial x^{q_1}_\beta}{\partial x_\alpha^{j_1}}(x) \cdots \frac{\partial x^{q_l}_\beta}{\partial x_\alpha^{j_l}}(x)}.
	\end{equation*}
	Then $\tau_{\alpha\beta}$ is clearly smooth and moreover
	\begin{equation*}
		\Phi_\alpha \circ \Phi_\beta^{-1}\del[1]{x,(A^{p_1\dots p_k}_{q_1\dots q_l})} = \del[1]{x, (A^{i_1\dots i_k}_{j_1\dots j_l})} = \del[1]{x,\tau_{\alpha\beta}(x)(A^{p_1\dots p_k}_{q_1\dots q_l})}. 
	\end{equation*}
	Therefore, conditions (i)-(iii) in the vector bundle chart lemma \ref{lem:vector_bundle_chart_lemma} are satisfied and the statement follows.
\end{proof}

Recall, that in a category $\mathcal{C}$, a \emph{section} of a morphism $f : X \to Y$ is a morphism $\sigma : Y \to X$ such that $f \circ \sigma = \id_Y$.

\begin{definition}[Tensor Field]
	Let $M$ be a smooth manifold and $k,l \in \mathbb{N}$. A \bld{smooth tensor field of type $(k,l)$ on $M$} is defined to be a section of $\pi : T^{(k,l)}TM \to M$. The space of all smooth tensor fields of type $(k,l)$ on $M$ is denoted by $\Gamma\del[1]{T^{(k,l)}TM}$.
\end{definition}

\begin{example}[Vector Field and Covector Field]
	\label{ex:vector_and_covector_fields}
	Let $M$ be a smooth manifold. Of particular importance are the tensor fields such that $k + l = 1$. If $k = 1$, such tensor fields are called \bld{vector fields} and we write $\mathfrak{X}(M) := \Gamma\del[1]{T^{(1,0)}TM}$. Likewise, if $l = 1$, we call such tensor fields \bld{covector fields} and write $\mathfrak{X}^*(M) := \Gamma\del[1]{T^{(0,1)}TM}$.
\end{example}

Let $\del[1]{U,(x^i)}$ be a chart on $M$ and $A : M \to T^{(k,l)}TM$ such that $A_x \in T^{(k,l)}(T_xM)$ for all $x \in M$. From (\ref{eq:tensor_expression_basis}) we get that
\begin{equation*}
	A_x = A^{i_1\dots i_k}_{j_1\dots j_l}(x)\frac{\partial}{\partial x^{i_1}}\bigg\vert_x \otimes \dots \otimes \frac{\partial}{\partial x^{i_k}}\bigg\vert_x \otimes dx^{j_1}\vert_x \otimes \dots \otimes dx^{j_l}\vert_x
\end{equation*}
\noindent for all $x \in U$ where $A^{i_1\dots i_k}_{j_1\dots j_l} : U \to \mathbb{R}$ are given as in (\ref{eq:tensor_components}). We will call these functions the \bld{component functions of $A$}. Recall, that if $M$ is a smooth manifold, $A \subseteq U \subseteq M$, where $U$ is open and $A$ is closed in $M$, a function $\psi \in C^\infty(M)$ is said to be a \emph{smooth bump function for $A$ supported in $U$}, iff $0 \leq \psi \leq 1$, $\psi \vert_A = 1$ and $\supp \psi \subseteq U$. The paracompactness condition guarantees that smooth bump functions exist in great abundance.

\begin{proposition}[{Existence of Smooth Bump Functions \cite[44]{lee:smooth_manifolds:2013}}]
	\label{prop:existence_smooth_bump_functions}
	Let $M$ be a smooth manifold and $A \subseteq U \subseteq M$, where $U$ is open and $A$ is closed in $M$. Then there exists a smooth bump function for $A$ supported in $U$. 	
\end{proposition}

\begin{proposition}[{Smoothness Criteria for Tensor Fields \cite[317]{lee:smooth_manifolds:2013}}]
	\label{prop:smoothness_criteria_for_tensor_fields}
	Let $M$ be smooth manifold, $k,l \in \mathbb{N}$ and $A : M \to T^{(k,l)}TM$ such that $A_x \in T^{(k,l)}T_xM$ for all $x \in M$. Then the following conditions are equivalent:
	\begin{enumerate}[label = \textup{(\alph*\textup)},leftmargin=*]
		\item $A \in \Gamma\del[1]{T^{(k,l)}TM}$.
		\item In every smooth coordinate chart, the component functions of $A$ are smooth.
		\item Each point of $M$ is contained in a chart in which $A$ has smooth component functions.
		\item For all $\omega^1,\dots,\omega^k \in \mathfrak{X}^*(M)$ and $X_1,\dots,X_l \in \mathfrak{X}(M)$, the function 
			\begin{equation*}
				\mathcal{A}(\omega^1,\dots,\omega^k,X_1,\dots,X_l) : M \to \mathbb{R}
			\end{equation*}
			\noindent defined by
			\begin{equation}
				\label{eq:curly_A}
				\mathcal{A}\del[1]{\omega^1,\dots,\omega^k,X_1,\dots,X_l}(x) := A_x\del[1]{\omega^1_x,\dots,\omega^k_x, X_1\vert_x,\dots,X_l\vert_x}
			\end{equation}
			\noindent is smooth.
		\item Let $U \subseteq M$ be open. If $\omega^1,\dots,\omega^k \in \mathfrak{X}^*(U)$ and $X_1,\dots,X_l \in \mathfrak{X}(U)$, then $\mathcal{A}$ defined by \textup{(\ref{eq:curly_A})} belongs to $C^\infty(U)$.
	\end{enumerate}
\end{proposition}

\begin{proof}
We prove (a) $\Leftrightarrow$ (b) and (b) $\Rightarrow$ (c) $\Rightarrow$ (d) $\Rightarrow$ (e) $\Rightarrow$ (b).\\
To prove (a) $\Leftrightarrow$ (b), let $x \in M$ and $\del[1]{U,(x^i)}$ be a smooth chart on $M$ about $x$. Proposition \ref{prop:tensor_bundle} yields a map $\Phi_U : \pi^{-1}(U) \to U \times \mathbb{R}^{n^{k + l}}$, and the proof of the vector bundle chart lemma implies, that the corresponding chart on $T^{(k,l)}TM$ is given by $\del[1]{\pi^{-1}(U),\wtilde{\varphi}}$, where 
\begin{equation*}
	\wtilde{\varphi}: \pi^{-1}(U) \to \varphi(U) \times \mathbb{R}^{n^{k+l}}
\end{equation*}
\noindent is defined by
\begin{equation*}
	\wtilde{\varphi} := \del[1]{\varphi \times \id_{\mathbb{R}^{n^{k+l}}}} \circ \Phi_U.
\end{equation*}
Since $A_x \in T^{(k,l)}T_xM$ for all $x \in M$, we have that 
\begin{equation*}
A^{-1}\del[1]{\pi^{-1}(U)} = (\pi \circ A)^{-1}(U) = \id_M(U) = U.
\end{equation*}
Hence $U \cap A^{-1}\del[1]{\pi^{-1}(U)} = U$, which is open in $M$, and 
\begin{equation*}
	\wtilde{\varphi} \circ A \circ \varphi^{-1} : \varphi(U) \to \wtilde{\varphi}\del[1]{\pi^{-1}(U)} 
\end{equation*}
\noindent is given by
\begin{align*}
	\del[1]{\wtilde{\varphi} \circ A \circ \varphi^{-1}}\del[1]{\varphi(y)} &= \del[1]{\varphi \times \id_{\mathbb{R}^{n^{k+l}}}}\del[1]{\Phi_U(A_y)}\\
	&= \del[1]{\varphi(y),(A^{i_1\dots i_k}_{j_1\dots j_l})(y)}\\
	&= \del[1]{\varphi(y),\del[1]{(A^{i_1\dots i_k}_{j_1\dots j_l}) \circ {\varphi^{-1}}}\del[1]{\varphi(y)}}
\end{align*}
\noindent for all $y \in U$. Thus $\wtilde{\varphi} \circ A \circ \varphi^{-1}$ is smooth if and only if $(A^{i_1\dots i_k}_{j_1\dots j_l}) \circ {\varphi^{-1}}$ is smooth, which is equivalent to $A^{i_1\dots i_k}_{j_1\dots j_l}$ being smooth.\\
The implication (b) $\Rightarrow$ (c) is immediate.\\
To prove (c) $\Rightarrow$ (d), suppose $x \in M$ and let $(U,(x^i))$ be a chart about $x$ such that the component functions of $A$ are smooth. By example \ref{ex:vector_and_covector_fields} and the equivalence (a) $\Leftrightarrow$ (b) we have
\begin{equation*}
	\omega^i = \omega_j^i dx^j \qquad \text{and} \qquad X_i = X^j_i \frac{\partial}{\partial x^j}
\end{equation*}
\noindent on $U$ for smooth functions $\omega_j^i$ and $X^j_i$. Thus for any $y \in U$ we compute
\begin{align*}
	\mathcal{A}\del[1]{\omega^1,\dots,\omega^k,X_1,\dots,X_l}(y) &= A_x\del[1]{\omega^1_x,\dots,\omega^k_x, X_1\vert_x,\dots,X_l\vert_x}\\
	&= \omega^1_{i_1}(y) \cdots \omega^k_{i_k}(y)X_1^{j_1}(y)\cdots X_l^{j_l}(y) A^{i_1\dots i_k}_{j_1\dots j_l}(y)
\end{align*}
\noindent and so $\mathcal{A}\del[1]{\omega^1,\dots,\omega^k,X_1,\dots,X_l}$ is smooth.\\
To prove (d) $\Rightarrow$ (e), we use the fact that smoothness is a local property. Let $x \in U$ and suppose $(V,\varphi)$ is a chart on $U$ centered at $x$. Then $\varphi(V) \subseteq \mathbb{R}^n$ is open and so we find $\varepsilon > 0$ such that $B_\varepsilon(0) \subseteq \varphi(V)$. Set $A := \varphi^{-1}\del[1]{\wbar{B}_{\varepsilon/2}(0)} \subseteq U$. Then $A$ is closed in $U$ and by proposition \ref{prop:existence_smooth_bump_functions} there exists a smooth bump function $\psi \in C^\infty(U)$ for $A$ supported in $U$. Define $\wtilde{\omega}^i : M \to T^*M$ and $\wtilde{X}_i : M \to TM$ by
\begin{equation*}
	\wtilde{\omega}^i_x := \ccases{
		\psi(x)\omega^i_x & x \in U,\\
		0_x & x \in M \setminus \supp \psi,
	} \quad \text{and} \quad \wtilde{X}_i\vert_x := \ccases{
		\psi(x)X_i\vert_x & x \in U,\\
		0_x & x \in M \setminus \supp \psi.
	}
\end{equation*}
Then $\wtilde{\omega}^i \in \mathfrak{X}^*(M)$ and $\wtilde{X}_i \in \mathfrak{X}(M)$ by the gluing lemma for smooth maps (see \cite[35]{lee:smooth_manifolds:2013}). Moreover, on $\varphi^{-1}\del[1]{B_{\varepsilon/2}(0)}$ we have that $\wtilde{\omega}^i = \omega^i$ and $\wtilde{X}_i = X_i$. But then also	
\begin{equation*}
	\mathcal{A}\del[1]{\wtilde{\omega}^1,\dots,\wtilde{\omega}^k,\wtilde{X_1},\dots,\wtilde{X}_l} = \mathcal{A}\del[1]{\omega^1,\dots,\omega^k,X_1,\dots,X_l}
\end{equation*}
\noindent on this neighbourhood, and so since the former is smooth by assumption, so is the latter.
Finally, to prove (e) $\Rightarrow$ (b), let $(U,(x^i))$ be a chart about $x \in M$. Consider $\omega^i \in \mathfrak{X}^*(U)$ and $X_i \in \mathfrak{X}(U)$ defined by
\begin{equation*}
	\omega^i := \delta^i_j dx^j \qquad \text{and} \qquad X_i := \delta^j_i \frac{\partial}{\partial x^j}.
\end{equation*}
Then it is easy to verify that
\begin{equation*}
	\mathcal{A}\del[1]{\omega^{i_1},\dots,\omega^{i_k},X_{j_1},\dots,X_{j_l}} = A^{i_1\dots i_k}_{j_1\dots j_l}
\end{equation*}
\noindent holds on $U$. Thus by assumption, each component function is smooth.
\end{proof}

Part (d) of the smoothness criteria for tensor fields \ref{prop:smoothness_criteria_for_tensor_fields} implies that for any tensor field $A \in \Gamma\del[1]{T^{(k,l)}TM}$ there is a mapping 
\begin{equation*}
	\mathcal{A} : \undercbrace{\mathfrak{X}^*(M) \times \dots \times \mathfrak{X}^*(M)}_{k} \times \undercbrace{\mathfrak{X}(M) \times \dots \times \mathfrak{X}(M)}_{l} \to C^\infty(M)
\end{equation*}
\noindent defined by 
\begin{equation*}
	\del[1]{\omega^1,\dots,\omega^k,X_1,\dots,X_l} \mapsto \mathcal{A}\del[1]{\omega^1,\dots,\omega^k,X_1,\dots,X_l}.
\end{equation*}
We will call this mapping the \bld{map induced by the tensor field $A$}.

\begin{theorem}[{Tensor Field Characterisation Lemma \cite[318]{lee:smooth_manifolds:2013}}]
	\label{thm:tensor_field_characterisation_lemma}
	Let $M$ be a smooth manifold and $k,l \in \mathbb{N}$. A mapping
\begin{equation*}
	\mathcal{A}: \undercbrace{\mathfrak{X}^*(M) \times \dots \times \mathfrak{X}^*(M)}_{k} \times \undercbrace{\mathfrak{X}(M) \times \dots \times \mathfrak{X}(M)}_{l} \to C^\infty(M) 
	\end{equation*}
	\noindent is induced by a $(k,l)$-tensor field if and only if $\mathcal{A}$ is multilinear over $C^\infty(M)$.
\end{theorem}

\begin{proof}
	Suppose $\mathcal{A}$ is induced by a $(k,l)$-tensor field $A$. Let $\omega^1,\dots,\omega^k,\wtilde{\omega}^i \in \mathfrak{X}^*(M)$ and $X_1,\dots,X_l \in \mathfrak{X}(M)$ as well as $f \in C^\infty(M)$. Then for any $x \in M$ we compute
	\begin{align*}
		\mathcal{A}\del[1]{\dots,\omega^i + f\wtilde{\omega}^i,\dots}(x) &= A_x\del[1]{\dots,\omega^i_x + f(x)\wtilde{\omega}^i_x,\dots}\\
		&= A_x\del[1]{\dots,\omega^i_x,\dots} + f(x)A_x\del[1]{\dots,\wtilde{\omega}^i_x,\dots}\\
		&= \mathcal{A}\del[1]{\dots,\omega^i,\dots}(x) + f(x)\mathcal{A}\del[1]{\dots,\wtilde{\omega}^i,\dots}(x)\\
		&= \del[1]{\mathcal{A}\del[1]{\dots,\omega^i,\dots} + f\mathcal{A}\del[1]{\dots,\wtilde{\omega}^i,\dots}}(x).
	\end{align*}
	Thus $\mathcal{A}$ is $C^\infty(M)$-multilinear with respect to the first $k$ arguments. Similarly, $\mathcal{A}$ is $C^\infty(M)$-multilinear with repect to the last $l$ arguments.\\
Conversly, suppose that
	\begin{equation*}
		\mathcal{A}: \undercbrace{\mathfrak{X}^*(M) \times \dots \times \mathfrak{X}^*(M)}_{k} \times \undercbrace{\mathfrak{X}(M) \times \dots \times \mathfrak{X}(M)}_{l} \to C^\infty(M)
	\end{equation*}
	\noindent is $C^\infty(M)$-multilinear. We wish to define a $(k,l)$-tensor field $A$ that induces $\mathcal{A}$. That this is indeed possible, is the observation that $\mathcal{A}\del[1]{\omega^1,\dots,\omega^k,X_1,\dots,X_l}(x)$ only depends on $\omega^1_x,\dots,\omega^k_x,X_1\vert_x,\dots,X_l\vert_x$. Thus we divide the remaining proof into three steps.\\
	\emph{Step 1: $\mathcal{A}\del[1]{\omega^1,\dots,\omega^k,X_1,\dots,X_l}$ acts locally.} That is, if either some $\omega^i$ or $X_i$ vanish on an open set $U$, then so does $\mathcal{A}\del[1]{\omega^1,\dots,\omega^k,X_1,\dots,X_l}$. Let $x \in U$ and $\psi \in C^\infty(M)$ be a smooth bump function for $\cbr{x}$ supported in $U$. Then $\psi\omega^i = 0$ on $M$ and by $C^\infty(M)$-multilinearity 
	\begin{equation*}
		0 = \mathcal{A}\del[1]{\dots,\psi\omega^i,\dots} = \psi(x)\mathcal{A}\del[1]{\dots,\omega^i,\dots}(x) = \mathcal{A}\del[1]{\dots,\omega^i,\dots}(x).
	\end{equation*}
	An analogous argument works if some $X_i$ vanishes on $U$.\\
	\emph{Step 2: $\mathcal{A}\del[1]{\omega^1,\dots,\omega^k,X_1,\dots,X_l}$ acts pointwise.} Thats is, if $\omega^i_x$ or $X_i\vert_x$ vanish for some $x \in M$, then so does $\mathcal{A}\del[1]{\omega^1,\dots,\omega^k,X_1,\dots,X_l}$. Let $(U,(x^i))$ be a chart about $x$. Then $\omega^i = \omega_j^i dx^j$ on $U$. Let $\psi \in C^\infty(U)$ denote the smooth bump function used in the proof of part (d) $\Rightarrow$ (e) of the smoothness criteria for tensor fields \ref{prop:smoothness_criteria_for_tensor_fields}. Define
	\begin{equation*}
		\varepsilon^j := \ccases{
			\psi(x)dx^j\vert_x & x \in U,\\
			0_x & x \in M \setminus \supp \psi,
		} \quad \text{and} \quad 
		f^i_j := \ccases{
			\psi(x)\omega^i_j(x) & x \in U,\\
			0_x & x \in M \setminus \supp \psi.
		}
	\end{equation*} 
	Then $\omega^i = f^i_j \varepsilon^j$ on a neighbourhood of $x$ and so by multilinearity and step 1, we have that
	\begin{equation*}
		\mathcal{A}\del[1]{\dots,\omega^i,\dots} = f^i_j\mathcal{A}\del[1]{\dots,\varepsilon^j,\dots}
	\end{equation*}
	\noindent on a neighbourhood of $x$. But since $\omega^i_x$ vanishes so does each $\omega^i_j(x)$. Hence 
	\begin{equation*}
		\mathcal{A}\del[1]{\dots,\omega^i,\dots}(x) = f^i_j(x)\mathcal{A}\del[1]{\dots,\varepsilon^j,\dots}(x) = \omega^i_j(x)\mathcal{A}\del[1]{\dots,\varepsilon^j,\dots}(x) = 0.
	\end{equation*}
	An analogous argument works if some $X_i\vert_x$.\\
	\emph{Step 3: Definition of the $(k,l)$-tensor field $A$ inducing $\mathcal{A}$.} Let $x \in M$, $\omega^1,\dots,\omega^k \in T_x^*M$ and $v_1,\dots,v_l \in T_xM$. Suppose that $\wtilde{\omega}^1,\dots,\wtilde{\omega}^k \in \mathfrak{X}^*(M)$ and $\wtilde{X}_1,\dots,\wtilde{X}_l \in \mathfrak{X}(M)$ are any extensions, respectively. That is, $\wtilde{\omega}^i_x = \omega^i$ and $\wtilde{X}_i\vert_x = v_i$. They do always exist, since in a chart $(U,(x^i))$ we may write
	\begin{equation*}
		\omega^i = \omega^i_j dx^j\vert_x \qquad \text{and} \qquad v_i = v_i^j \frac{\partial}{\partial x^j}\bigg\vert_x
	\end{equation*}
	\noindent and so using a smooth bump function for $\cbr{x}$ supported in $U$ we can construct global maps as in step 2 if we consider the components as constant functions. Now define
	\begin{equation}
		\label{eq:def_A}
		A_x\del[1]{\omega^1,\dots,\omega^k,v_1,\dots,v_l} := \mathcal{A}\del[1]{\wtilde{\omega}^1,\dots,\wtilde{\omega}^k,\wtilde{X}_1,\dots,\wtilde{X}_l}(x).
	\end{equation}
	This is well-defined by step 2. Now if $\omega^1,\dots,\omega^k \in \mathfrak{X}^*(M)$ and $X_1,\dots,X_l \in \mathfrak{X}(M)$, we have that 
	\begin{equation*}
		\mathcal{A}\del[1]{\omega^1,\dots,\omega^k,X_1,\dots,X_l}(x) = A_x\del[1]{\omega^1_x,\dots,\omega^k_x,X_1\vert_x,\dots,X_l\vert_x},
	\end{equation*}
	\noindent since $\omega^i$ and $X_i$ are extensions of $\omega^i_x$ and $X_i\vert_x$, respectively, for all $x \in M$. So the assumption that $\mathcal{A}$ takes values in the space of smooth functions $C^\infty(M)$ together with part (d) of the smoothness criteria for tensor fields \ref{prop:smoothness_criteria_for_tensor_fields} yields that $A$ is a smooth $(k,l)$-tensor field which moreover induces $\mathcal{A}$.
\end{proof}


\begin{theorem}[{Bundle Homomorphism Characterisation Lemma \cite[262]{lee:smooth_manifolds:2013}}]
	\label{thm:bundle_homomorphism_characterisation_lemma}
	Let $\pi : E \to M$ and $\wtilde{\pi} : \wtilde{E} \to M$ be smooth vector bundles over a smooth manifold $M$. A map $\mathcal{F} : \Gamma(E) \to \Gamma(\wtilde{E})$ is linear over $C^\infty(M)$ if and only if there exists a smooth bundle homomorphism $F : E \to \wtilde{E}$ over $M$ such that $\mathcal{F}(\sigma) = F \circ \sigma$ for all $\sigma \in \Gamma(E)$.
\end{theorem}

\begin{proposition}[Tangent-Cotangent Bundle Isomorphism]
	\label{prop:tangent-cotangent_bundle_isomorphism}
	Let $(M,\omega)$ be a symplectic manifold. Define $\Omega : TM \to T^*M$ by
	\begin{equation}
		\Omega_{(x,v)}(w) := \omega_x(v,w)
	\end{equation}
	\noindent for all $w \in T_xM$. Then $\Omega$ is a well-defined smooth bundle isomorphism.
\end{proposition}

\begin{proof}
	
\end{proof}
