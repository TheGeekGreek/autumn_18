\chapter{Review of Differential Topology}
We follow the treatment as provided by \emph{Will J. Merry} in the year course \emph{Differential Geometry I and II} at the \emph{ETH Zurich} in the autumn semester 2018 and spring semester 2019, respectively. The course notes are available at
\begin{center}
	\href{https://www.merry.io/differential-geometry/}{https://www.merry.io/differential-geometry/}.
\end{center}
Additionally, we rely on \cite{lee:smooth_manifolds:2013}.

\section*{The Category of Smooth Manifolds}

\begin{definition}[Topological Manifold]
	Let $n \in \mathbb{N}$. A topological space $M$ is said to be a \bld{topological manifold of dimension $n$}, iff
	\begin{enumerate}[label = \textup{(\roman*)},leftmargin=*]
		\item $M$ is locally Euclidean of dimension $n$, that is, for every $x \in M$ there exist an open subset $U \subseteq M$ and a function $\varphi : U \to \mathbb{R}^n$ such that $\varphi(U) \subseteq \mathbb{R}^n$ is open and $\varphi : U \to \varphi(U)$ is a homeomorphism. Every such pair $(U,\varphi)$ is called a \bld{chart on $M$ about $x$}.
		\item $M$ is Hausdorff and has at most countably many connected components.
		\item $M$ is paracompact, that is, every open cover of $M$ admits a locally finite open refinement.
	\end{enumerate}
\end{definition}

\begin{definition}[Smooth Atlas]
	A \bld{smooth atlas for a topological manifold $M$} is a collection $(U_\alpha,\varphi_\alpha)_{\alpha \in A}$ of charts on $M$ such that
	\begin{enumerate}[label = \textup{(\roman*)},leftmargin = *]
		\item $(U_\alpha)_{\alpha \in A}$ is an open cover for $M$.
		\item For all $\alpha,\beta \in A$ such that $U_\alpha \cap U_\beta \neq \varnothing$, the function 
			\begin{equation*}
				\varphi_\alpha \circ \varphi_\beta^{-1} : \varphi_\beta(U_\alpha \cap U_\beta) \to \varphi_\alpha(U_\alpha \cap U_\beta)
			\end{equation*}
			\noindent is smooth. The function $\varphi_\alpha \circ \varphi_\beta^{-1}$ is called a \bld{transition function}.
	\end{enumerate}
\end{definition}

Let $\mathcal{A}$ and $\mathcal{A}'$ be two smooth atlases on a topological manifold $M$. Define a relation on the set of all smooth atlases on $M$ (this is a subset of the power set $2^M$) by
\begin{equation*}
	\mathcal{A} \sim \mathcal{A}' \quad :\Leftrightarrow \quad \mathcal{A} \cup \mathcal{A}' \text{ is an atlas for } M.
\end{equation*}

\begin{exercise}
	Show that above relation is actually an equivalence relation on the set of all smooth atlases on a topological manifold $M$.
\end{exercise}

\begin{definition}[Smooth Structure]
	A \bld{smooth structure on a topological manifold $M$} is an equivalence class $\sbr[0]{\mathcal{A}}$ where $\mathcal{A}$ is a smooth atlas for $M$.
\end{definition}

\begin{definition}[Maximal Smooth Atlas]
	Let $\sbr[0]{\mathcal{A}}$ be a smooth structure on a topological manifold $M$. Define the \bld{maximal smooth atlas on $M$} by $\bigcup_{\mathcal{A}' \in \sbr[0]{\mathcal{A}}} \mathcal{A}'$.
\end{definition}

\begin{definition}[Smooth Manifold]
	Let $n \in \mathbb{N}$. A \bld{smooth manifold of dimension $n$} is defined to be a pair $(M,\mathcal{A})$, where $M$ is a topological manifold of dimension $n$ and $\mathcal{A}$ is a maximal smooth atlas on $M$.
\end{definition}

\begin{example}[$n$-Spheres]
	Let $n \in \mathbb{N}$. If $n = 0$, we have that $\mathbb{S}^0 = \cbr{\pm 1}$. It is easily seen that $\mathbb{S}^0$ is a smooth manifold of dimension $0$. Let $n \geq 1$. Define $N := e_{n + 1}$ and $S := -e_{n + 1}$, where $e_{n + 1}$ denotes the $n + 1$-th standard basis vector of $\mathbb{R}^{n + 1}$. Moreover, set
	\begin{equation*}
		U_+ := \mathbb{S}^n \setminus S \qquad \text{and} \qquad U_- := \mathbb{S}^n \setminus N.
	\end{equation*}
	Then $U_+$ and $U_-$ are open subsets of $\mathbb{S}^n$, the upper and lower hemisphere, respectively. Then the functions $\varphi_\pm : U_\pm \to \mathbb{R}^n$ defined by
	\begin{equation*}
		\varphi_\pm(x) := \frac{1}{1 \pm x_{n + 1}}(x_1,\dots,x_n),
	\end{equation*}
	\noindent are homeomorphisms. Indeed, one can check that $\psi_\pm : \mathbb{R}^n \to U_\pm$ defined by
	\begin{equation*}
		\psi_\pm(x) := \del[4]{\frac{2x}{1 + \abs[0]{x}^2}, \frac{\pm(1 - \abs[0]{x}^2)}{1 + \abs[0]{x}^2}} 
	\end{equation*}
	\noindent is a continuous inverse for $\varphi_+$ and $\varphi_-$, respectively. We claim that $\cbr[0]{(U_\pm,\varphi_\pm)}$ is a smooth atlas for $\mathbb{S}^n$. Clearly, $\mathbb{S}^n$ is covered by the two charts. Next we have to calculate the transition functions $\varphi_\mp \circ \varphi^{-1}_\pm = \varphi_\mp \circ \psi_\pm : \varphi_\pm(U_+ \cap U_-) \to \varphi_\mp(U_+ \cap U_-)$. It is easy to see that $\varphi_\pm(U_+ \cap U_-) = \mathbb{R}^n \setminus \cbr{0}$ and that
	\begin{equation*}
		\varphi_\mp \circ \psi_\pm = \frac{x}{\abs{x}^2},
	\end{equation*}
	\noindent which is smooth. Since $\mathbb{S}^n$ is Hausdorff as a metric space and as a subspace of a second countable space, itself second countable, $\mathbb{S}^n$ equipped with the smooth structure induced by the smooth atlas constructed above, is a smooth manifold of dimension $n$.
\end{example}

\begin{definition}[Smooth Map]
	Let $M$ and $N$ be smooth manifolds and $F : M \to N$ a map. We say that $F$ is \bld{smooth}, iff for all $x \in M$, there exists a chart $(U,\varphi)$ on $M$ about $x$ and a chart $(V,\psi)$ on $N$ about $F(x)$ such that
	\begin{enumerate}[label = \textup{(\roman*)},leftmargin=*]
		\item $U \cap F^{-1}(V)$ is open in $M$.
		\item $\psi \circ F \circ \psi^{-1} : \varphi\del[1]{U \cap F^{-1}(V)} \to \psi(V)$ is smooth.
	\end{enumerate}
	The \bld{set of all smooth maps from $M$ to $N$} is denoted by $C^\infty(M,N)$ and the \bld{set of all smooth functions on $M$} is denoted by $C^\infty(M)$.
\end{definition}

\begin{exercise}
	Let $M$ be a smooth manifold. Show that $C^\infty(M)$ is an $\mathbb{R}$-algebra under pointwise defined operations.
\end{exercise}

\begin{example}[Coordinate Functions]
	Let $M^n$ be a smooth manifold and $(U,\varphi)$ be a chart about some $x \in M$. Let $\pi^i : \mathbb{R}^n \to \mathbb{R}$ be defined by $\pi^i(x^1,\dots,x^n) := x^i$ for $i = 1,\dots,n$. Define $x^i : U \to \mathbb{R}$ by $x^i := \pi^i \circ \varphi$. Then $x^i \in C^\infty(U)$ and we call $x^i$ a \bld{coordinate function}. Moreover, we may denote the chart $(U,\varphi)$ by $\del[1]{U,(x^i)}$ and say that $(x^i)$ are \bld{local coordinates about $x$}. 
\end{example}

\section*{Tangent Spaces}
Let $M$ be a smooth manifold and let $x \in M$. Define a binary relation on the set
\begin{equation*}
	X := \cbr[1]{(U,f) : U \subseteq M \text{ neighbourhood of } x, f \in C^\infty(U)}
\end{equation*}
\noindent by
\begin{equation*}
	(U,f)\sim(V,g) \quad :\Leftrightarrow \quad \exists W \subseteq U \cap V \text{ neighbourhood of $x$, such that } f\vert_W = g\vert_W.
\end{equation*}

\begin{exercise}
	Show that the above relation is actually an equivalence relation.
\end{exercise}

\begin{definition}[Germ]
	Let $M$ be a smooth manifold and let $x \in M$. The set of \bld{germs at $p$}, written $C^\infty_x(M)$ is defined to be $C^\infty_x(M) := X/{\sim}$.
\end{definition}

\begin{exercise}
	Show that $C^\infty_x(M)$ is an $\mathbb{R}$-algebra under the obvious operations.
\end{exercise}

\begin{remark}
	Note that if $f \in C^\infty(M)$, then $\sbr[0]{(M,f)} \sim \sbr[0]{(U,f\vert_U)}$ for any neighbourhood $U$ of $x$. Thus any germ at $p$ contains a representant which is defined on the whole manifold and we thus may simply write $\sbr[0]{f}$ for a germ at $p$. 
\end{remark}

\begin{remark}
	Let $\sbr[0]{f}$ be a germ at $x \in M$. Then $f(x)$ is well-defined. Indeed, if $f\vert_U = g\vert_U$ on some neighbourhood of $x$, then in particular $f(x) = g(x)$.
\end{remark}

\begin{definition}[Tangent Space]
	Let $M$ be a smooth manifold and let $x \in M$. The \bld{tangent space of $M$ at $x$}, written $T_xM$, is defined to be the vector space $\del[1]{C^\infty_x(M)}^*$ such that
	\begin{equation*}
		v(\sbr[0]{f}\sbr[0]{g}) = v\sbr[0]{f}g(x) + f(x)v\sbr[0]{g}
	\end{equation*}
	\noindent holds.
\end{definition}

\begin{definition}[Derivation]
	Let $M$ be a smooth manifold, $x \in M$ and $U$ a neighbourhood of $x$. The \bld{space of derivations of $C^\infty(U)$ at $x$}, written $\mathcal{D}_x(U)$, is defined to be the vector space $\del[1]{C^\infty(U)}^*$ such that
	\begin{equation*}
		v(fg) = v(f)g(x) + f(x)v(g)
	\end{equation*}
	\noindent holds.
\end{definition}

\begin{proposition}
	\label{prop:isomorphism_tangent_space}
	Let $M$ be a a smooth manifold, $x \in M$ and $U$ be a neighbourhood of $x$. Then
	\begin{equation*}
		T_xM \cong \mathcal{D}_x(U).
	\end{equation*}
\end{proposition}

\begin{proof}
	Let $\Phi : T_xM \to \mathcal{D}_x(U)$ be defined by
	\begin{equation*}
		\Phi(v)(f) := v\sbr[0]{f}
	\end{equation*}
	\noindent for all $f \in C^\infty(U)$. Clearly $\Phi$ is well-defined and linear. We want to construct an inverse $\Psi : \mathcal{D}_x(U) \to T_xM$ for $\Phi$. This implies, that we should define
	\begin{equation*}
		\Psi(v)\sbr[0]{f} = v\del[1]{\wtilde{f}}
	\end{equation*}
	\noindent where $\wtilde{f} \in C^\infty(U)$ such that $\sbr[1]{\wtilde{f}} = \sbr[0]{f}$. 
	\begin{enumerate}[label=\textit{Step \arabic*:},leftmargin=*,wide=0pt]
		\item \textit{Existence of $\wtilde{f}$.} Let $(V,f)$ be a representant of $\sbr[0]{f}$. As in the proof of the smoothness criteria for tensor fields \ref{prop:smoothness_criteria_for_tensor_fields}, we find a neighbourhood $W$ about $x$ such that $\wwbar{W} \subseteq U \cap V$. Then there exists a smooth bump function $\psi \in C^\infty(U \cap V)$ such that $\psi\vert_W = 1$ and $\supp \psi \subseteq U\cap V$. Let $\wtilde{f} := \psi f$ extended to be zero on $U$. Then clearly $\sbr[1]{\wtilde{f}} = \sbr[0]{f}$ since $\wtilde{f} = f$ on $W$.
		\item \textit{$\Psi$ is well-defined.} Suppose that $\sbr[0]{f} = \sbr[0]{g}$ in $C_x^\infty(M)$. Then $f = g$ on some neighbourhood $V$ of $x$. We claim that $v(f) = v(g)$ on $U \cap V$. Indeed, let $\psi$ be a smooth bump function for $\cbr{x}$ supported in $U \cap V$. Then $\psi(f - g) = 0$ on $U$ and we compute
			\begin{equation*}
				0 = v\del[1]{\psi(f - g)} = v(\psi)(f - g)(x) + \psi(x)v(f - g) = v(f - g).
			\end{equation*}
	\end{enumerate}
\end{proof}

\begin{example}[Coordinate Derivation]
	Let $M^n$ be a smooth manifold and $(U,\varphi)$ be a chart on $M$. For every $x \in U$ and every $i = 1,\dots,n$ define 
	\begin{equation*}
		\frac{\partial}{\partial x^i}\bigg\vert_x : C^\infty(U) \to \mathbb{R}
	\end{equation*}
	\noindent by
	\begin{equation*}
		\frac{\partial}{\partial x^i}\bigg\vert_x(f) := D_i(f \circ \varphi^{-1})\del[1]{\varphi(x)}.
	\end{equation*}
	Then clearly $\frac{\partial}{\partial x^i}\big\vert_x$ is a derivation of $C^\infty(U)$ at $x$. Thus by proposition \ref{prop:isomorphism_tangent_space}, $\frac{\partial}{\partial x^i}\big\vert_x \in T_xM$. 
\end{example}

One of the profound features of tangent spaces to a smooth manifold are that they are finite dimensional. In fact, they admit the same dimension as the manifold.

\begin{lemma}
	\label{lem:star_shaped}
	Let $\upOmega \subseteq \mathbb{R}^n$ be open, and star-shaped about $0 \in \upOmega$. Suppose $f \in C^\infty(\upOmega)$. Then there exists $\varphi_1,\dots,\varphi_n \in C^\infty(\upOmega)$ such that $\varphi_i(0) = D_if(0)$ and
	\begin{equation*}
		f = f(0) + \pi^i \varphi_i.
	\end{equation*}
\end{lemma}

\begin{proof}
	For $x \in \upOmega$ define $\gamma : \intcc{0,1} \to \upOmega$ by $\gamma_x(t) := tx$ (note that this is only possible since $\upOmega$ is assumed to be star-shaped with centre $0$). Then
	\begin{align*}
		f(x) - f(0) &= \int_0^1 (f \circ \gamma_x)'(t) dt\\
		&= \int_0^1 D_if\del[1]{\gamma_x(t)} \dot{\gamma}^i_x(t) dt\\
		&= \int_0^1 D_if\del[1]{\gamma_x(t)} \pi^i(x)dt\\
		&= \pi^i(x)\varphi_i(x)
	\end{align*}
	\noindent where
	\begin{equation*}
		\varphi_i(x) := \int_0^1 D_if\del[1]{\gamma_x(t)} dt.
	\end{equation*}
\end{proof}

\begin{proposition}[Basis for the Tangent Space]
	Let $M^n$ be asmooth manifold and $x \in M$. Then 
	\begin{equation*}
		\cbr[3]{\frac{\partial}{\partial x^i}\bigg\vert_x : i = 1,\dots,n}
	\end{equation*}
	\noindent is a basis for $T_xM$, where $\del[1]{U,(x^i)}$ is any chart about $x$.
\end{proposition}

\begin{proof}

\end{proof}
