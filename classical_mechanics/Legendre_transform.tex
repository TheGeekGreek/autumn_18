\section*{Legendre Transform}
In this section we \emph{dualize} the notion of a Lagrangian function, that is, to each Lagrangian function $L \in C^\infty(TM)$ we will associate a \emph{dual function} $L^* \in C^\infty(T^*M)$. It turns out, that in this dual formulation, the equations of motion take a very symmetric form. To simplify the notation and illuminating the main concept, we consider Lagrangian functions of a special type.

\begin{definition}[Autonomous System]
	A Lagrangian system $(M,L)$ is said to be an \bld{autonomous Lagrangian system}\index{Lagrangian!system!autonomous}, iff $L \in C^\infty(TM)$.
\end{definition}

Let $(M^n,L)$ be an autonomous Lagrangian system and $(U,x^i)$ a chart on $M$. Moreover, let $(x^i,v^i)$ denote standard coordinates on $TM$, that is $v^i := dx^i$ for all $i = 1,\dots,n$. Expanding the Euler-Lagrange equations (\ref{eq:EL_equations}) yields
\begin{align*}
	\frac{\partial L}{\partial x^j}\del[1]{\gamma(t),\dot{\gamma}(t)} &= \frac{d}{dt}\frac{\partial L}{\partial v^j}\del[1]{\gamma(t),\dot{\gamma}(t)}\\
	&= \frac{\partial^2 L}{\partial x^i\partial v^j}\del[1]{\gamma(t),\dot{\gamma}(t)}\dot{\gamma}^i(t) + \frac{\partial^2 L}{\partial v^i\partial v^j}\del[1]{\gamma(t),\dot{\gamma}(t)}\ddot{\gamma}^i(t)
\end{align*}
\noindent for all $j = 1,\dots,n$. In order to solve above system of second order ordinary differential equations for $\ddot{\gamma}^i(t)$ and all initial conditions in the chart on $TU$, the matrix $\mathcal{H}_L(x,v)$ defined by
\begin{equation}
	\label{eq:H_L}
	\mathcal{H}_L(x,v) := \del[3]{\frac{\partial^2 L}{\partial v^i\partial v^j}(x,v)}_j^i
\end{equation}
\noindent must be invertible on $TU$.

\begin{definition}[Nondegenrate System]
	An autonomous Lagrangian system $(M,L)$ is said to be \bld{nondegenerate}\index{Lagrangian!system!nondegenerate}, iff for all coordinate charts $U$ on $M$, $\det \mathcal{H}_L(x,v) \neq 0$ holds on $TU$. 
\end{definition}

\begin{example}[Nondegenrate System on a Riemannian Manifold]
	\label{ex:nondegenerate_Lagrangian_system}
	Let $(M,g)$ be a Riemannian manifold. Consider the Lagrangian $T - V$ with kinetic energy $T \in C^\infty(TM)$ defined by $T(v) := \frac{1}{2}\abs{v}^2$ and potential energy $V \in C^\infty(M)$. Then the computation performed in example \ref{ex:motions_on_Riemannian_manifolds} yields
	\begin{equation*}
		\mathcal{H}_{T - V}(x,v) = \del[1]{g_{ij}(x)}^i_j
	\end{equation*} 
	\noindent on every chart since $\frac{\partial V}{\partial v^i} = 0$ for every $i$, and so this Lagrangian system is nondegenerate.
\end{example}

The nondegeneracy of an autonomous Lagrangian system is intrinsically connected to a certain differential form in $\Omega^1(TM)$, which we will construct now. For every $(x,v) \in TM$ we can define a covector $D^\mathcal{F}_{(x,v)} L \in T^*_xM$ by setting
\begin{equation}
	D^\mathcal{F}_{(x,v)}L := \frac{\partial}{\partial v^i}\bigg\vert_{(x,v)}(L) dx^i\vert_x = \frac{\partial L}{\partial v^i}dx^i.
\end{equation}
Let $(\wtilde{U},\wtilde{x}^i)$ be another chart on $M$ such that $U \cap \wtilde{U} \neq \varnothing$. Denote the induced coordinates on $TM$ by $(\wtilde{x}^i,\wtilde{v}^i)$. Then on $U \cap \wtilde{U}$ we have that 
\begin{equation*}
	\frac{\partial}{\partial \wtilde{v}^i} = \frac{\partial x^j}{\partial \wtilde{v}^i}\frac{\partial}{\partial x^j} + \frac{\partial v^j}{\partial \wtilde{v}^i}\frac{\partial}{\partial v^j} = \frac{\partial v^j}{\partial \wtilde{v}^i}\frac{\partial}{\partial v^j}.
\end{equation*}
Moreover
\begin{equation*}
	\frac{\partial}{\partial x^j} = \frac{\partial \wtilde{x}^k}{\partial x^j} \frac{\partial}{\partial \wtilde{x}^k}
\end{equation*}
\noindent which implies
\begin{equation*}
	d\wtilde{x}^i \del[3]{\frac{\partial}{\partial x^j}} = \frac{\partial \wtilde{x}^k}{\partial x^j} d\wtilde{x}^i\del[3]{\frac{\partial}{\partial \wtilde{x}^k}} = \frac{\partial \wtilde{x}^k}{\partial x^j}\delta^i_k = \frac{\partial \wtilde{x}^i}{\partial x^j}.
\end{equation*}
Thus
\begin{equation*}
	d\wtilde{x}^i = \frac{\partial \wtilde{x}^i}{\partial x^j} dx^j
\end{equation*}
\noindent or equivalently 
\begin{equation*}
	v^j = \frac{\partial x^j}{\partial \wtilde{x}^i}\wtilde{v}^i,
\end{equation*}
\noindent and so we compute
\begin{equation*}
	D^\mathcal{F}L = \frac{\partial L}{\partial \wtilde{v}^i}d\wtilde{x}^i = \frac{\partial v^j}{\partial \wtilde{v}^i}\frac{\partial L}{\partial v^j}\frac{\partial \wtilde{x}^i}{\partial x^k} dx^k = \frac{\partial x^j}{\partial \wtilde{x}^i}\frac{\partial L}{\partial v^j}\frac{\partial \wtilde{x}^i}{\partial x^k} dx^k = \frac{\partial L}{\partial v^j}\delta^j_k dx^k = \frac{\partial L}{\partial v^j}dx^j.
\end{equation*}
Therefore, $D^\mathcal{F}L$ is independent of the choice of coordinates.

\begin{definition}[Fibrewise Differential\footnote{\unboldmath This terminology is adapted from exercise C.3. on problem sheet C of the lecture \emph{Differential geometry I} taught by \emph{Will J. Merry} at \emph{ETH Z\"urich} in the autumn semester $2018$, which can be found \href{https://www.merry.io/differential-geometry-i-problem-sheets/problem-sheet-c}{here}. See also \cite[2]{mazzucchelli:CPT:2012}.}]
	\label{def:fibrewise_differential}
	Let $(M,L)$ be an autonomous Lagrangian system. The form $D^\mathcal{F}L \in \Omega^1(TM)$ defined on a chart $(U,x^i)$ of $M$ by
	\begin{equation}
		\label{eq:fibrewise_differential}
		D^\mathcal{F}L := \frac{\partial L}{\partial v^i}dx^i
	\end{equation}
	\noindent where $(x^i,v^i)$ denotes the induced standard coordinates on $TM$, is called the \bld{fibrewise differential of $L$}\index{Differential!fibrewise}.
\end{definition}

\begin{remark}
	The preceeding discussion showed, that the fibrewise differential $D^\mathcal{F}L$ is well-defined.
\end{remark}

\begin{example}[Fibrewise Differential on a Riemannian Manifold]
	\label{ex:fibrewise_differential_Riemannian_manifold}
	Consider the autonomous Lagrangian system as defined in example \ref{ex:nondegenerate_Lagrangian_system}. Then the computation performed in example \ref{ex:motions_on_Riemannian_manifolds} yields
	\begin{equation*}
		D_{(x,v)}^\mathcal{F}(T - V) = g_{ij}(x)v^idx^j
	\end{equation*}
	\noindent on every chart since $\frac{\partial V}{\partial v^j} = 0$ for all $j$.
\end{example}

Recall, that a $2$-covector on a finite-dimensional real vector space is said to be \emph{nondegenrate}, iff the matrix representation with respect to some basis is invertible. Moreover, a \emph{nondegenerate $2$-form} on a smooth manifold $M$ is defined to be a $2$-form $\omega$, such that $\omega_x$ is a nondegenrate $2$-covector for all $x \in M$ (see \cite[565,567]{lee:smooth_manifolds:2013}). 

\begin{proposition}
	An autonomous Lagrangian system $(M,L)$ is nondegenerate if and only if $d(D^\mathcal{F}L)$ is nondegenerate.
\end{proposition}

\begin{proof}
	Using the computation performed in \cite[363]{lee:smooth_manifolds:2013}, we get
	\begin{equation*}
		d(D^\mathcal{F}L) = d\del[3]{\frac{\partial L}{\partial v^j}dx^j} = \frac{\partial^2 L}{\partial x^i\partial v^j}dx^i \wedge dx^j + \frac{\partial^2L}{\partial v^i\partial v^j}dv^i \wedge dx^j.
	\end{equation*}
	Moreover, using part (e) of properties of the wedge product \cite[356]{lee:smooth_manifolds:2013}, we compute
	\begin{align*}
		d(D^\mathcal{F}L)\del[3]{\frac{\partial}{\partial x^k},\frac{\partial}{\partial x^l}} =& \frac{\partial^2 L}{\partial x^i\partial v^j}\det\begin{pmatrix}
			\displaystyle dx^i\del[3]{\frac{\partial}{\partial x^k}} & \displaystyle dx^j\del[3]{\frac{\partial}{\partial x^k}}\\[4mm]
			\displaystyle dx^i\del[3]{\frac{\partial}{\partial x^l}} & \displaystyle dx^j\del[3]{\frac{\partial}{\partial x^l}}
		\end{pmatrix}\\ 
		&+ \frac{\partial^2L}{\partial v^i\partial v^j}\det\begin{pmatrix}
			\displaystyle dv^i\del[3]{\frac{\partial}{\partial x^k}} & \displaystyle dx^j\del[3]{\frac{\partial}{\partial x^k}}\\[4mm]
			\displaystyle dv^i\del[3]{\frac{\partial}{\partial x^l}} & \displaystyle dx^j\del[3]{\frac{\partial}{\partial x^l}}
		\end{pmatrix}\\
		=& \frac{\partial^2 L}{\partial x^i\partial v^j}(\delta^i_k\delta^j_l - \delta^i_l\delta^j_k)\\
		=& \frac{\partial^2 L}{\partial x^k\partial v^l} - \frac{\partial^2 L}{\partial x^l\partial v^k}
	\end{align*}
	\noindent for all $k,l = 1,\dots,n$. Similarly, we compute
	\begin{equation*}
		d(D^\mathcal{F}L)\del[3]{\frac{\partial}{\partial v^k},\frac{\partial}{\partial x^l}} = \frac{\partial^2 L}{\partial v^k\partial v^l} \qquad \text{and} \qquad d(D^\mathcal{F}L)\del[3]{\frac{\partial}{\partial v^k},\frac{\partial}{\partial v^l}} = 0,
	\end{equation*}
	\noindent and using skew-symmetry, we also deduce
	\begin{equation*}
		d(D^\mathcal{F}L)\del[3]{\frac{\partial}{\partial x^k},\frac{\partial}{\partial v^l}} = -\frac{\partial^2 L}{\partial v^k\partial v^l}.
	\end{equation*}
	Therefore, the matrix representing $d(D^\mathcal{F}L)$ with respect to the standard basis is given by the block matrix
	\begin{equation*}
		d(D^\mathcal{F}L) = \del{\begin{array}{c|c}
			* & -\mathcal{H}_L\\\hline
			\mathcal{H}_L & 0
		\end{array}},
	\end{equation*}
	\noindent where $\mathcal{H}_L$ is the matrix defined in (\ref{eq:H_L}). Thus 
	\begin{equation*}
		\det\del[1]{d(D^\mathcal{F}L)} = (-1)^n(\det \mathcal{H}_L)^2
	\end{equation*}
	Hence the matrix representation of $d(D^\mathcal{F}L)$ is invertible if and only if $\mathcal{H}_L$ is invertible, and the conclusion follows.
\end{proof}

So far, we have associated to each Lagrangian system $(M,L)$ a $1$-form on $TM$, the fibrewise differential $D^\mathcal{F}L$. In order to get closer to our goal of dualizing the concept of a Lagrangian function, we need also a $1$-form on $T^*M$. Suppose $(U,x^i)$ is a chart on $M$. The induced standard coordinates on the cotangent bundle $T^*M$ of $M$ are given by $(x^i,\xi_i)$, where $\xi_i := \frac{\partial}{\partial x^i}$, considered as an element of the double dual $T^{**}U$. On this chart, define a one $1$-form $\alpha$ by $\alpha := \xi_i dx^i$. Suppose $(\wtilde{x}^i,\wtilde{\xi}_i)$ are other coordinates. Then from the computations performed at the beginning of the previous section, we have that
\begin{equation*}
	\wtilde{\xi}_i = \frac{\partial x^j}{\partial \wtilde{x}^i}\xi_j \qquad \text{and} \qquad d\wtilde{x}^i = \frac{\partial \wtilde{x}^i}{\partial x^k}dx^k.
\end{equation*}
Thus
\begin{equation*}
	\alpha = \wtilde{\xi}_i d\wtilde{x}^i = \frac{\partial x^j}{\partial \wtilde{x}^i}\xi_j\frac{\partial \wtilde{x}^i}{\partial x^k}dx^k = \xi_j \delta^j_k dx^k = \xi_j dx^j,
\end{equation*}
\noindent and so, $\alpha$ is independen of the choice of coordinates.

\begin{definition}[Tautological Form]
	\label{def:tautological_form}
	Let $M$ be a smooth manifold. The \bld{tautological form on $T^*M$}\index{Form!tautological}, denoted by $\alpha$, is the form  $\alpha \in \Omega^1(T^*M)$ defined locally by
	\begin{equation*}
		\alpha := \xi_idx^i,
	\end{equation*}
	\noindent where $(x^i,\xi_i)$ denotes the standard coordinates on $T^*M$.
\end{definition}

\begin{remark}
	The preceeding discussion showed, that the tautological form $\alpha$ is well-defined.
\end{remark}

\begin{remark}
	The tautological form $\alpha$ as well as the fibrewise derivative $D^\mathcal{F}L$ on an autonomous Lagrangian system $(M,L)$ admit invariant definitions, that is a coordinate free definition. For the invariant definition of $\alpha$ see \cite[569]{lee:smooth_manifolds:2013} or \cite[10--11]{silva:SG:2008}, and for the invariant definition of $D^\mathcal{F}L$ see \cite[31]{takhtajan:QM:2008}. 
\end{remark}

\begin{definition}[Legendre Transform]
	\label{def:Legendre_transform}
	A \bld{Legendre transform of an autonomous Lagrangian system $(M,L)$}\index{Transform!Legendre} is defined to be a fibrewise mapping $\tau_L \in C^\infty(TM,T^*M)$ such that
	\begin{equation*}
		D^\mathcal{F}L = \tau_L^*(\alpha).
	\end{equation*}
\end{definition}

\begin{example}[Legendre Transform on a Riemannian Manifold]
	\label{ex:Legendre_transform_Riemannian_manifold}
	Let $(M,L)$ be a Lagrangian system. Then the morphism $\tau_L : TM \to T^*M$ defined by
	\begin{equation}
		\label{eq:Legendre_transform}
		\tau_L(x,v) := \del[1]{x,D^\mathcal{F}_{(x,v)}L}
	\end{equation}
	\noindent is a Legendre transform. In particular, if we consider the Lagrangian system defined in example \ref{ex:nondegenerate_Lagrangian_system}, we get that the above defined Legendre transform is a diffeomorphism. Indeed, suppose that $\tau_{T - V}(x,v) = \tau_{T - V}(\wtilde{x},\wtilde{v})$. Then $x = \wtilde{x}$ and
	\begin{equation*}
		g_{ij}(x)v^idx^j = g_{ij}(x)\wtilde{v}^idx^j	
	\end{equation*}
	\noindent using example \ref{ex:fibrewise_differential_Riemannian_manifold}. So we must have
	\begin{equation*}
		g_{ij}(x)v^i = g_{ij}(x)\wtilde{v}^i
	\end{equation*}
	\noindent for all $j$. Multiplying both sides by $g^{kj}(x)$ yields $v^k = \wtilde{v}^k$ for every $k$ and hence $v = \wtilde{v}$. Thus $\tau_{T - V}$ is injective. Let $\xi \in T^*_xM$ be given by $\xi_i dx^i\vert_x$. Then $\tau_{T - V}(x,v) = (x,\xi)$, where $v$ is given in coordinates by $v^k := g^{ki}(x)\xi_i$.
\end{example} 

Since the nondegenracy of a Lagragian system $(M,L)$ is inherently connected to the nondegenracy of the form $d(D^\mathcal{F}L)$ and the definition of the Legendre transform invokes the form $D^\mathcal{F}L$, one would expect a connection between the nondegeneracy of the Lagrangian system and a local property of Legendre transform.

\begin{lemma}
	\label{lem:local_diffeomorphism}
	A Legendre transform on a Lagrangian system is a local diffeomorphism if and only if the Lagrangian system is nondegenrate.	
\end{lemma}

\begin{proof}
	Denote the Lagrangian system by $(M,L)$. Let $(U,x^i)$ be a chart on $M$ and denote by $(x^i,v^i)$ and $(x^i,\xi_i)$ the induced standard coordinates on $TM$ and $T^*M$, respectively. Then we compute
	\begin{equation*}
		\tau^*_L(\alpha) = \tau^*_\alpha(\xi_j dx^j) = (\xi_j \circ \tau_L)d\del[1]{x^j \circ \tau_L},
	\end{equation*}
	\noindent which must coincide with
	\begin{equation*}
		D^\mathcal{F}L = \frac{\partial L}{\partial v^j}dx^j.
	\end{equation*}
	Thus in coordinates
	\begin{equation}
		\label{eq:Legendre_in_coordinates}
		\tau_L(x,v) = \del[3]{x,\frac{\partial L}{\partial v}}
	\end{equation}
	\noindent and so
	\begin{equation*}
		d_{(x,v)}\tau_L = \del{\begin{array}{c|c}
			I & 0\\\hline
			0 & \mathcal{H}_L
		\end{array}}
	\end{equation*}
	\noindent at every $(x,v) \in TM$. Hence 
	\begin{equation*}
		\det \del[1]{d_{(x,v)}\tau_L} = \det \mathcal{H}_L.
	\end{equation*}
	If $\tau_L$ is a local diffeomorphism, by definition, we have that some restriction of $\tau_L$ to some neighbourhood of $(x,v)$ is a diffeomorphism, and so, by properties of differentials (d) \cite[55]{lee:smooth_manifolds:2013}, we have that $d_{(x,v)}\tau_L$ is an isomorphism. Conversly, if the Lagrangian system is nondegenerate, we conclude using the inverse function theorem for manifolds \cite[79]{lee:smooth_manifolds:2013}, that $\tau_L$ is a local diffeomorphism.
\end{proof}

\begin{definition}[Energy]
	The \bld{energy of an autonomous Lagrangian system $(M,L)$}\index{Energy!of an autonomous Lagrangian system} is defined to be the function $E_L \in C^\infty(TM)$ given by
	\begin{equation*}
		E_L(x,v) := D^\mathcal{F}_{(x,v)}L(v) - L(x,v),
	\end{equation*}
	\noindent in standard coordinates $(x^i,v^i)$ of $TM$.
\end{definition}

\begin{example}[Energy on a Riemannian Manifold]
	\label{ex:energy_Riemannian_manifold}
	Consider the Lagrangian system defined in example \ref{ex:nondegenerate_Lagrangian_system}. Then the computation performed in example \ref{ex:fibrewise_differential_Riemannian_manifold} yields
	\begin{align*}
		E_{T - V}(x,v) &= \frac{\partial T}{\partial v^k}v^k - \frac{\partial V}{\partial v^k}v^k - T(v) + V(x)\\
		&= \frac{1}{2}g_{ij}\delta^i_k v^jv^k + \frac{1}{2}g_{ij}v^i \delta^j_k v^k - T(v) + V(x)\\
		&= g_{ij}v^i v^j - T(v) + V(x)\\
		&= T(v) + V(x)
	\end{align*}
	\noindent for every $(x,v) \in TM$. Hence the energy of this Lagrangian system is given by \emph{kinetic energy plus potential energy}.
\end{example}

\begin{definition}[Hamiltonian Function]
	\label{def:Hamiltonian_function}
	Let $(M,L)$ be an autonomous Lagrangian system and $\tau_L$ a diffeomorphic Legendre transform. The morphism $H_L \in C^\infty(T^*M)$ defined by
	\begin{equation*}
		H_L := E_L \circ \tau_L^{-1}
	\end{equation*}
	\noindent is called the \bld{Hamiltonian function associated to the Lagrangian function $L$}\index{Hamiltonian!function}.
\end{definition}

\begin{example}[Hamiltonian function on a Riemannian Manifold]
	Consider the Lagrangian system defined in example \ref{ex:nondegenerate_Lagrangian_system}. By example \ref{ex:Legendre_transform_Riemannian_manifold} the Legendre transform $\tau_{T - V}$ is a diffeomorphism. Using example \ref{ex:energy_Riemannian_manifold}, we compute
	\begin{align*}
		H_{T - V}(x,\xi) &= E_{T - V}\del[1]{\tau^{-1}_{T - V}(x,\xi)}\\
		&= E_{T - V}\del[1]{x,v}\\
		&= T(v) + V(x)\\
		&= \frac{1}{2}g_{ij}(x)v^i v^j + V(x)\\
		&= \frac{1}{2}g_{ij}(x)g^{ik}(x)\xi_k g^{jl}(x)\xi_l + V(x)\\
		&= \frac{1}{2}\delta^k_j\xi_j g^{jl}(x)\xi_l + V(x)\\
		&= \frac{1}{2}g^{kl}(x)\xi_k\xi_l + V(x)
	\end{align*}
	\noindent where $v = \del[1]{g^{ki}}^k_i \xi$.
\end{example}

\begin{theorem}[Hamilton's Equations]
	\label{thm:Hamiltons_equations}
	Let $\gamma$ be a motion on an autonomous Lagrangian system $(M^n,L)$ and suppose that $\tau_L$ is a diffeomorphic Legendre transform. Then $\gamma$ satisfies the Euler-Lagrange equations in every chart if and only if the path 
	\begin{equation*}
			\del[1]{\gamma(t),\xi(t)} := \tau_L\del[1]{\gamma(t),\dot{\gamma}(t)}
	\end{equation*}
	\noindent satisfies the following system of first order ordinary differential equations in every chart:
	\begin{equation}
		\label{eq:Hamiltons_equations}
		\dot{\gamma}(t) = \frac{\partial H_L}{\partial \xi}\del[1]{\gamma(t),\xi(t)} \qquad \text{and} \qquad \dot{\xi}(t) = -\frac{\partial H_L}{\partial x}\del[1]{\gamma(t),\xi(t)}
	\end{equation}
	The equations \textup{(}\ref{eq:Hamiltons_equations}\textup{)} are called \bld{Hamilton's equations}\index{Equations!Hamilton's}.
\end{theorem}

\begin{proof}
	First we compute $H_L$ in standard coordinates $(x^i,\xi_i)$ on $T^*M$. By (\ref{eq:Legendre_in_coordinates}), the Legendre transform is given by
	\begin{equation}
		\tau_L(x,v) = \del[3]{x,\frac{\partial L}{\partial v}(x,v)}
	\end{equation}
	\noindent in standard coordinates on $TM$. Since $\tau_L$ is a diffeomorphism by assumption, in particular it is a local diffeomorphism (see \cite[80]{lee:smooth_manifolds:2013}). Hence by lemma \ref{lem:local_diffeomorphism}, the Lagrangian system $(M,L)$ is nondegenerate. So considering $\tau^{-1}_L(x,\xi)$, we can apply the implicit function theorem \cite[661]{lee:smooth_manifolds:2013} to obtain $v$ implicitely from the equation
	\begin{equation*}
		\xi = \frac{\partial L}{\partial v}(x,v).
	\end{equation*}
	Hence in coordinates
	\begin{equation*}
		H_L(x,\xi) = \del[3]{\frac{\partial L}{\partial v^i}v^i - L(x,v)}\bigg\vert_{\xi = \frac{\partial L}{\partial v}}.
	\end{equation*}
	Therefore
	\begin{equation*}
		\frac{\partial H_L}{\partial \xi^j} = \frac{\partial}{\partial \xi_j} \del[1]{\xi_i v^i - L(x,v)}\big\vert_{\xi = \frac{\partial L}{\partial v}} = \delta^j_i v^i = v^j.
	\end{equation*}
	Hence
	\begin{equation*}
		\frac{\partial H_L}{\partial \xi^j}\del[1]{\gamma(t),\xi(t)} = \dot{\gamma}^j(t),
	\end{equation*}
	\noindent for all $j = 1,\dots,n$. Moreover, we have that
	\begin{equation*}
		\frac{\partial H_L}{\partial x^j} = \frac{\partial}{\partial x^j}\del[3]{\frac{\partial L}{\partial v^i}v^i - L(x,v)}\bigg\vert_{\xi = \frac{\partial L}{\partial v}} = - \frac{\partial L}{\partial x^j}(x,v)\bigg\vert_{\xi = \frac{\partial L}{\partial v}}, 
	\end{equation*}
	\noindent and so
	\begin{equation*}
		\frac{\partial H_L}{\partial x^j}\del[1]{\gamma(t),\xi(t)} = -\frac{\partial L}{\partial x^j}\del[1]{\gamma(t),\dot{\gamma}(t)},
	\end{equation*}
	\noindent for all $j = 1,\dots,n$. If the Euler-Lagrange equations (\ref{eq:EL_equations}) hold, then we get
	\begin{equation*}
		\frac{\partial H_L}{\partial x^j}\del[1]{\gamma(t),\xi(t)} = -\frac{d}{dt}\frac{\partial L}{\partial v^j}\del[1]{\gamma(t),\dot{\gamma}(t)} = -\dot{\xi}_j(t),
	\end{equation*}
	\noindent and thus the Hamilton's equations (\ref{eq:Hamiltons_equations}) hold. Conversly, if we suppose that Hamilton's equations (\ref{eq:Hamiltons_equations}) hold, we get that
	\begin{equation*}
		-\frac{d}{dt}\frac{\partial L}{\partial v^j}\del[1]{\gamma(t),\dot{\gamma}(t)} = -\dot{\xi}_j(t) = \frac{\partial H_L}{\partial x^j}\del[1]{\gamma(t),\xi(t)} = -\frac{\partial L}{\partial x^j}\del[1]{\gamma(t),\dot{\gamma}(t)},
	\end{equation*}
	\noindent and so the Euler-Lagrange equations (\ref{eq:EL_equations}) are satisfied.
\end{proof}

\begin{remark}
	Under some reasonable assumptions on the Lagrangian system it can be shown that the Legendre transform (\ref{eq:Legendre_transform}) defined in example \ref{ex:Legendre_transform_Riemannian_manifold} is always a diffeomorphism. For more details see \cite[8]{mazzucchelli:CPT:2012}.	
\end{remark}
