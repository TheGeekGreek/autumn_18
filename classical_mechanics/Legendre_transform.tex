\section*{Legendre Transform}
In this section we \emph{dualize} the notion of a Lagrangian function, that is, to each Lagrangian function $L \in C^\infty(TM)$ we will associate a \emph{dual function} $L^* \in C^\infty(T^*M)$. It turns out, that in this dual formulation, the equations of motion take a very symmetric form. To simplify the notation and illuminating the main concept, we consider Lagrangian functions of a special type.

\begin{definition}[Autonomous System]
	A Lagrangian system $(M,L)$ is said to be an \bld{autonomous Lagrangian system}, iff $L \in C^\infty(TM)$.
\end{definition}

\subsection*{The Energy of a Lagrangian System}
Let $M \in \mathsf{Diff}$ of dimension $n$ and $(U,(x^i))$ a chart on $M$. Let $(x^i,v^i)$ denote standard coordinates on $TM$, that is $v^i := dx^i$ for all $i = 1,\dots,n$. Suppose $L \in C^\infty(TM)$. For every $(x,v) \in TM$ we can define a covector $d^\mathcal{F}_{(x,v)} L \in T^*_xM$ by setting
\begin{equation}
	d^\mathcal{F}_{(x,v)}L := \frac{\partial}{\partial v^i}\bigg\vert_{(x,v)}(L) dx^i\vert_x = \frac{\partial L}{\partial v^i}dx^i.
\end{equation}
Let $(\wtilde{x}^i,\wtilde{v}^i)$ denote another pair of coordinates on $TM$. Then we have that 
\begin{equation*}
	\frac{\partial}{\partial \wtilde{v}^i} = \frac{\partial x^j}{\partial \wtilde{v}^i}\frac{\partial}{\partial x^j} + \frac{\partial v^j}{\partial \wtilde{v}^i}\frac{\partial}{\partial v^j} = \frac{\partial v^j}{\partial \wtilde{v}^i}\frac{\partial}{\partial v^j}.
\end{equation*}
Moreover
\begin{equation*}
	\frac{\partial}{\partial x^j} = \frac{\partial \wtilde{x}^k}{\partial x^j} \frac{\partial}{\partial \wtilde{x}^k}
\end{equation*}
\noindent which implies
\begin{equation*}
	d\wtilde{x}^i \del[3]{\frac{\partial}{\partial x^j}} = \frac{\partial \wtilde{x}^k}{\partial x^j} d\wtilde{x}^i\del[3]{\frac{\partial}{\partial \wtilde{x}^k}} = \frac{\partial \wtilde{x}^k}{\partial x^j}\delta^i_k = \frac{\partial \wtilde{x}^i}{\partial x^j}.
\end{equation*}
Thus
\begin{equation*}
	d\wtilde{x}^i = \frac{\partial \wtilde{x}^i}{\partial x^j} dx^j
\end{equation*}
\noindent or equivalently 
\begin{equation*}
	v^j = \frac{\partial x^j}{\partial \wtilde{x}^i}\wtilde{v}^i,
\end{equation*}
\noindent and so we compute
\begin{equation*}
	d^\mathcal{F}L = \frac{\partial L}{\partial \wtilde{v}^i}d\wtilde{x}^i = \frac{\partial v^j}{\partial \wtilde{v}^i}\frac{\partial L}{\partial v^j}\frac{\partial \wtilde{x}^i}{\partial x^k} dx^k = \frac{\partial x^j}{\partial \wtilde{x}^i}\frac{\partial L}{\partial v^j}\frac{\partial \wtilde{x}^i}{\partial x^k} dx^k = \frac{\partial L}{\partial v^j}\delta^j_k dx^k = \frac{\partial L}{\partial v^j}dx^j.
\end{equation*}
Therefore, $d^\mathcal{F}$ is independent of the choice of coordinates.

\begin{definition}[Fibrewise Differential\footnote{\unboldmath This terminology is adapted from exercise C.3. on problem sheet C of the lecture \emph{Differential geometry I} taught by \emph{Will J. Merry} at \emph{ETH Z\"urich} in the autumn semester $2018$, which can be found \href{https://www.merry.io/differential-geometry-i-problem-sheets/problem-sheet-c}{here}.}]
	Let $(M,L)$ be an autonomous Lagrangian system. The map $d^\mathcal{F}L : TM \to T^*M$ defined on a chart $(U,x^i)$ of $M$ by
	\begin{equation}
		\label{eq:fibrewise_differential}
		d^\mathcal{F}L := \frac{\partial L}{\partial v^i}dx^i
	\end{equation}
	\noindent where $(x^i,v^i)$ denotes the induced standard coordinates on $TM$, is called the \bld{fibrewise differential of $L$}.
\end{definition}

\begin{remark}
	The preceeding discussion showed, that the fibrewise differential $d^\mathcal{F}L$ is well-defined.
\end{remark}

\begin{definition}[Energy]
	The \bld{energy} of an autonomous Lagrangian system $(M,L)$ is defined to be the function $E_L \in C^\infty(TM)$ given by
	\begin{equation*}
		E_L(x,v) := d^\mathcal{F}_{(x,v)}L(v) - L(x,v),
	\end{equation*}
	\noindent in standard coordinates $(x^i,v^i)$ of $TM$.
\end{definition}

\begin{example}
	Let $(M,g)$ be a Riemannian manifold and consider the Lagrangian $T - U$, with kinetic energy $T \in C^\infty(TM)$ defined by $T(v) := \frac{1}{2}\abs{v}^2$ and potential energy $U \in C^\infty(M)$. Then the computation performed in example \ref{ex:motions_on_Riemannian_manifolds} yields
	\begin{align*}
		E_{T - U}(x,v) &= \frac{\partial T}{\partial v^k}v^k - \frac{\partial U}{\partial v^k}v^k - T(v) + U(x)\\
		&= \frac{1}{2}g_{ij}\delta^i_k v^jv^k + \frac{1}{2}g_{ij}v^i \delta^j_k v^k - T(v) + U(x)\\
		&= g_{ij}v^i v^j - T(v) + U(x)\\
		&= T(v) + U(x)
	\end{align*}
	\noindent for every $(x,v) \in TM$. Hence the energy of this Lagrangian system is given by \emph{kinetic energy plus potential energy}.
\end{example}

\subsection*{Hamilton's Equations}
