%%%%%%%%%%%%%%%%%%%%%%%%%%%%%%%%%%%%%%%%%%%%%%%%%%%%%%%%%%%%%%%%%%%%%%%%%%
%Author:																 %
%-------																 %
%Yannis Baehni at University of Zurich									 %
%baehni.yannis@uzh.ch													 %
%																		 %
%Version log:															 %
%------------															 %
%06/02/16 . Basic structure												 %
%04/08/16 . Layout changes including section, contents, abstract.		 %
%%%%%%%%%%%%%%%%%%%%%%%%%%%%%%%%%%%%%%%%%%%%%%%%%%%%%%%%%%%%%%%%%%%%%%%%%%

%Page Setup
\documentclass[
	12pt, 
	oneside, 
	a4paper,
	reqno,
	final
]{amsart}

\usepackage[
	left = 3cm, 
	right = 3cm, 
	top = 3cm, 
	bottom = 3cm
]{geometry}

\newcommand\hmmax{0}
\newcommand\bmmax{0}

%Headers and footers
\usepackage{fancyhdr}
	\pagestyle{fancy}
	%Clear fields
	\fancyhf{}
	%Header right
	\fancyhead[R]{
		\footnotesize
		Yannis B\"{a}hni\\
		\href{mailto:yannis.baehni@uzh.ch}{yannis.baehni@uzh.ch}
	}
	%Header left
	\fancyhead[L]{
		\footnotesize
		Algebraic Topology II\\
		Spring 2018
	}
	%Page numbering in footer
	\fancyfoot[C]{\thepage}
	%Separation line header and footer
	\renewcommand{\headrulewidth}{0.4pt}
	%\renewcommand{\footrulewidth}{0.4pt}
	
	\setlength{\headheight}{19pt} 

\usepackage[utf8]{inputenc}
\usepackage[T1]{fontenc}

%Title
\usepackage[foot]{amsaddr}
\usepackage{newtxtext}
\usepackage[subscriptcorrection,nofontinfo,mtpcal,mtphrb]{mtpro2}
\usepackage{mathtools}
\usepackage{bm}
\usepackage{xspace}
\makeatletter
\def\@textbottom{\vskip \z@ \@plus 1pt}
\let\@texttop\relax
\usepackage{etoolbox}
\patchcmd{\abstract}{\scshape\abstractname}{\textbf{\abstractname}}{}{}

%Section, subsection and subsubsection font
%------------------------------------------
	\renewcommand{\@secnumfont}{\bfseries}
	\renewcommand\section{\@startsection{section}{1}%
  	\z@{.7\linespacing\@plus\linespacing}{.5\linespacing}%
  	{\normalfont\boldmath\bfseries\centering}}
	\renewcommand\subsection{\@startsection{subsection}{2}%
    	\z@{.5\linespacing\@plus.7\linespacing}{-.5em}%
    	{\normalfont\boldmath\bfseries}}%
	\renewcommand\subsubsection{\@startsection{subsubsection}{3}%
    	\z@{.5\linespacing\@plus.7\linespacing}{-.5em}%
    	{\normalfont\boldmath\bfseries}}%
%Formatting title of TOC
\renewcommand{\contentsnamefont}{\bfseries}
%Table of Contents
\setcounter{tocdepth}{3}

% Add bold to \section titles in ToC and remove . after numbers
\renewcommand{\tocsection}[3]{%
	\indentlabel{\@ifnotempty{#2}{\bfseries\ignorespaces#1 #2\quad}}\boldmath\bfseries#3}
\renewcommand{\tocappendix}[3]{%
  	\indentlabel{\@ifnotempty{#2}{\bfseries\ignorespaces#1 #2: }}\bfseries#3}
% Remove . after numbers in \subsection
\renewcommand{\tocsubsection}[3]{%
  \indentlabel{\@ifnotempty{#2}{\ignorespaces#1 #2\quad}}#3}
%\let\tocsubsubsection\tocsubsection% Update for \subsubsection
%...

\newcommand\@dotsep{4.5}
\def\@tocline#1#2#3#4#5#6#7{\relax
  \ifnum #1>\c@tocdepth % then omit
  \else
    \par \addpenalty\@secpenalty\addvspace{#2}%
    \begingroup \hyphenpenalty\@M
    \@ifempty{#4}{%
      \@tempdima\csname r@tocindent\number#1\endcsname\relax
    }{%
      \@tempdima#4\relax
    }%
    \parindent\z@ \leftskip#3\relax \advance\leftskip\@tempdima\relax
    \rightskip\@pnumwidth plus1em \parfillskip-\@pnumwidth
    #5\leavevmode\hskip-\@tempdima{#6}\nobreak
    \leaders\hbox{$\m@th\mkern \@dotsep mu\hbox{.}\mkern \@dotsep mu$}\hfill
    \nobreak
    \hbox to\@pnumwidth{\@tocpagenum{\ifnum#1=1\bfseries\fi#7}}\par% <-- \bfseries for \section page
    \nobreak
    \endgroup
  \fi}
\AtBeginDocument{%
\expandafter\renewcommand\csname r@tocindent0\endcsname{0pt}
}
\def\l@subsection{\@tocline{2}{0pt}{2.5pc}{5pc}{}}
\def\l@subsubsection{\@tocline{2}{0pt}{4.5pc}{5pc}{}}
\makeatother

\advance\footskip0.4cm
\textheight=54pc    %a4paper
\textheight=50.5pc %letterpaper
\advance\textheight-0.4cm
\calclayout
%Font settings
%\usepackage{anyfontsize}
%Footnote settings
\usepackage{footmisc}
%	\renewcommand*{\thefootnote}{\fnsymbol{footnote}}
\usepackage{commath}
%Further math environments
%Further math fonts (loads amsfonts implicitely)
%Redefinition of \text
%\usepackage{amstext}
\usepackage{upref}
%Graphics
%\usepackage{graphicx}
%\usepackage{caption}
%\usepackage{subcaption}
%Frames
\usepackage{mdframed}
\allowdisplaybreaks
%\usepackage{interval}
\newcommand{\toup}{%
  \mathrel{\nonscript\mkern-1.2mu\mkern1.2mu{\uparrow}}%
}
\newcommand{\todown}{%
  \mathrel{\nonscript\mkern-1.2mu\mkern1.2mu{\downarrow}}%
}
\AtBeginDocument{\renewcommand*\d{\mathop{}\!\mathrm{d}}}
\renewcommand{\Re}{\operatorname{Re}}
\renewcommand{\Im}{\operatorname{Im}}
\renewcommand{\div}{\operatorname{div}}
\DeclareMathOperator\Log{Log}
\DeclareMathOperator\Arg{Arg}
\DeclareMathOperator\id{id}
\DeclareMathOperator\sech{sech}
\DeclareMathOperator\Aut{Aut}
\DeclareMathOperator\h{h}
\DeclareMathOperator\sgn{sgn}
\DeclareMathOperator\arctanh{arctanh}
\DeclareMathOperator*\esssup{ess.sup}
\DeclareMathOperator\ob{ob}
\DeclareMathOperator\coker{coker}
\DeclareMathOperator\im{im}
\DeclareMathOperator\Ch{Ch}
\DeclareMathOperator\AugCh{AugCh}
\DeclareMathOperator\Ext{Ext}
\DeclareMathOperator\Hom{Hom}
\DeclareMathOperator\tr{tr}
\DeclareMathOperator\dist{dist}
\DeclareMathOperator\supp{supp}
\DeclareMathOperator\grad{grad}
\DeclareMathOperator\Tor{Tor}
\DeclareMathOperator\twist{twist}
\DeclareMathOperator\colim{colim}
%\usepackage{hhline}
%\usepackage{booktabs} 
%\usepackage{array}
%\usepackage{xfrac} 
%\everymath{\displaystyle}
%Enumerate
\usepackage{tikz-cd}
\usepackage{enumitem} 
%\renewcommand{\labelitemi}{$\bullet$}
%\renewcommand{\labelitemii}{$\ast$}
%\renewcommand{\labelitemiii}{$\cdot$}
%\renewcommand{\labelitemiv}{$\circ$}
%Colors
%\usepackage{color}
%\usepackage[cmtip, all]{xy}
%Main style theorem environment
\newtheoremstyle{main} 		             	 		%Stylename
  	{}	                                     		%Space above
  	{}	                                    		%Space below
  	{\itshape}			                     		%Body font
  	{}        	                             		%Indent
  	{\boldmath\bfseries}   	                         		%Head font
  	{.}            	                        		%Head punctuation
  	{ }           	                         		%Head space 
  	{\thmname{#1}\thmnumber{ #2}\thmnote{ (#3)}}	%Head specification
\theoremstyle{main}
\newtheorem{definition}{Definition}[section]
\newtheorem{proposition}{Proposition}[section]
\newtheorem{corollary}{Corollary}[section]
\newtheorem{theorem}{Theorem}[section]
\newtheorem{lemma}{Lemma}[section]
\newtheoremstyle{nonit} 		             	 		%Stylename
  	{}	                                     		%Space above
  	{}	                                    		%Space below
  	{}			                     		%Body font
  	{}        	                             		%Indent
	{\boldmath\bfseries}   	                         		%Head font
  	{.}            	                        		%Head punctuation
  	{ }           	                         		%Head space 
  	{\thmname{#1}\thmnumber{ #2}\thmnote{ (#3)}}	%Head specification
\theoremstyle{nonit}
\newtheorem{remark}{Remark}[section]
\newtheorem{example}{Example}[section]
\newtheorem{examples}{Examples}[section]
\newtheoremstyle{ex} 		             	 		%Stylename
  	{}	                                     		%Space above
  	{}	                                    		%Space below
  	{\small}			                     		%Body font
  	{}        	                             		%Indent
  	{\bfseries\boldmath}   	                         		%Head font
  	{.}            	                        		%Head punctuation
  	{ }           	                         		%Head space 
  	{\thmname{#1}\thmnumber{ #2}\thmnote{ (#3)}}	%Head specification
\theoremstyle{ex}
\newtheorem{exercise}[theorem]{Exercise}
%German non-ASCII-Characters
%Graphics-Tool
%\usepackage{tikz}
%\usepackage{tikzscale}
%\usepackage{bbm}
%\usepackage{bera}
%Listing-Setup
%Bibliographie
\usepackage[backend=bibtex, style=alphabetic]{biblatex}
%\usepackage[babel, german = swiss]{csquotes}
\bibliography{bibliography}
%PDF-Linking
%\usepackage[hyphens]{url}
\usepackage[bookmarksopen=true,bookmarksnumbered=true]{hyperref}
%\PassOptionsToPackage{hyphens}{url}\usepackage{hyperref}
\urlstyle{rm}
\hypersetup{
  colorlinks   = true, %Colours links instead of ugly boxes
  urlcolor     = blue, %Colour for external hyperlinks
  linkcolor    = blue, %Colour of internal links
  citecolor    = blue %Colour of citations
}
\newcommand{\bld}[1]{\boldmath\textit{\textbf{#1}}\unboldmath}


\begin{document}

%Roman page numbering
\frontmatter

\renewcommand*{\thefootnote}{\fnsymbol{footnote}}

%Half-Title Page
\thispagestyle{empty}

\begin{center}
    \rule{\linewidth}{1mm}\\
	\fontsize{35}{35}{\textsc{Mathematical Aspects of}}\\
	\fontsize{85}{85}{\textsc{Classical Mechanics}}\\
	\rule{\linewidth}{1mm} \\[.5cm]
	\fontsize{25}{25}{\textsc{Yannis B\"ahni}}\\
	\vspace{.5cm}
	\begin{figure}[h!tb]
		\centering
		\includegraphics[width = \textwidth]{variation.pdf}
	\end{figure}
	\begin{figure}[h!tb]
		\includegraphics[width = .4\textwidth,left]{ETHZ_Logo.pdf}
	\end{figure}
\end{center}
\clearpage

\chapter*{Preface}
This notes are the product of a semester project done at the ETH Zurich in the autumn semester of $2018$ under the supervision of \emph{Dr. Ana Cannas da Silva}\footnote{Departement of Mathematics, ETH Z\"urich, R\"amistrasse 101, 8092 Z\"urich, Switzerland.}. I will roughly follow the first chapter of the book \emph{Quantum Mechanics for Mathematicians} by \emph{Leon A. Takhtajan} \cite{takhtajan:QM:2008}, which serves as a brief introduction to classical mechanics. Since this introduction is very brief, understandable by considering its purpose, I additionally rely on the classic \emph{Mathematical Methods of Classical Mechanics} by \emph{Vladimir I. Arnold} \cite{arnold:CM:1989}. As the title already suggests, this is not a treatment of the physical part of classical mechanics, but a mathematical one. Hence the aim of these notes is to give a thoughtful introduction to the mathematical methods used in the realm of classical mechanics and their strong connection to differential topology and differential geometry, especially \emph{symplectic geometry}.
\newline\\
\noindent Winterthur, \hfill Yannis B\"ahni\\
September 8, 2018
\tableofcontents

\mainmatter

\renewcommand*{\thefootnote}{\arabic{footnote}}

\chapter{Lagrangian Mechanics}

\section*{Introduction}
Classical mechanics deals with ordinary differential equations originating from extremals of \bld{functionals}\index{Functional}, that is functions defined on an infinite-dimensional function space. The study of such extremality properties of functionals is known as the \bld{calculus of variations}\index{Calculus!of variations}. To illustrate this fundamental principle, let us consider the \emph{variational formulation} of second order elliptic operators in divergence form based on \cite[167--168]{struwe:fa:2014}.\\ 
For convention, unless explicietly stated otherwise, we will assume that all manifolds are smooth, that is of class $C^\infty$, finite-dimensional, Hausdorff and paracompact with at most countably many connected components.\\ 
Let $n \in \mathbb{N}$, $n \geq 1$, and $\upOmega \subseteq \subseteq \mathbb{R}^n$ such that $\wbar{\upOmega}$ is a smooth manifold with boundary. Moreover, let $H^1_0(\upOmega)$ denote the Sobolev space $W^{1,2}_0(\upOmega)$ with inner product
\begin{equation*}
	\langle u,v \rangle_{H^1_0(\upOmega)} = \int_\upOmega uv + \int_\upOmega \nabla u \nabla v.
\end{equation*}
Suppose $a^{ij} \in C^\infty(\wbar{\upOmega})$ symmetric, $f \in C^\infty(\wbar{\upOmega})$ and consider the second order homogenous Dirichlet problem
\begin{equation}
	\label{eq:homogeneous_Dirichlet_problem}
	\ccases{
			\displaystyle -\frac{\partial}{\partial x^j}\del[3]{a^{ij}\frac{\partial u}{\partial x^i}} = f & \text{in } \upOmega,\\
			u = 0 & \text{on } \partial \upOmega,
		}
\end{equation}
Suppose $u \in C^\infty(\wbar{\upOmega})$ solves (\ref{eq:homogeneous_Dirichlet_problem}). Then integration by parts (see \cite[436]{lee:smooth_manifolds:2013}) yields 
\begin{equation*}
	\int_\upOmega f v = -\int_\upOmega \frac{\partial}{\partial x^j}\del[3]{a^{ij}\frac{\partial u}{\partial x^i}}v = -\int_\upOmega \mathrm{div}(X)v = \int_\upOmega \langle X, \nabla v\rangle = \int_\upOmega a^{ij} \frac{\partial u}{\partial x^i}\frac{\partial v}{\partial x^j}
\end{equation*}
\noindent for any $v \in C^\infty_c(\upOmega)$, where $X := \del[1]{a^{ij}\frac{\partial u}{\partial x^i}}_j$. Thus we say that $u \in H^1_0(\upOmega)$ is a \emph{weak solution} of (\ref{eq:homogeneous_Dirichlet_problem}) iff
\begin{equation*}
	\forall v \in C^\infty_c(\upOmega): \> \int_\upOmega a^{ij} \frac{\partial u}{\partial x^i}\frac{\partial v}{\partial x^j} = \int_\upOmega fv.
\end{equation*}
If $(a^{ij})_{ij}$ is \emph{uniformly elliptic}, i.e. there exists $\lambda > 0$ such that
\begin{equation*}
	\forall x \in \upOmega\forall \xi \in \mathbb{R}^n : \> a^{ij}(x)\xi_i\xi_j \geq \lambda \abs{\xi}^2,	
\end{equation*}
\noindent then (\ref{eq:homogeneous_Dirichlet_problem}) admits a unique weak solution $u \in H^1_0(\upOmega)$ (in fact $u \in C^\infty(\upOmega)$ using \emph{regularity theory}, for more details see \cite[175]{struwe:fa:2014}). Indeed, observe that 
\begin{equation*}
	\langle \cdot,\cdot \rangle_a : H^1_0(\upOmega) \times H^1_0(\upOmega) \to \mathbb{R}
\end{equation*}
\noindent defined by
\begin{equation}
	\label{eq:Sobolev_inner_product}
	\langle u, v \rangle_a := \int_\upOmega a^{ij} \frac{\partial u}{\partial x^i}\frac{\partial v}{\partial x^j}
\end{equation}
\noindent is an inner product on $H^1_0(\upOmega)$ with induced norm equivalent to the standard one on $H^1_0(\upOmega)$ due to Poincar\'e's inequality \cite[107]{struwe:fa:2014}. Applying the Riesz Representation theorem \cite[49--50]{struwe:fa:2014} yields the result. Moreover, this solution can be characterized by a \emph{variational principle}, i.e. if we define the \emph{energy functional} $E : H^1_0(\upOmega) \to \mathbb{R}$
\begin{equation*}
	E(v) := \frac{1}{2} \norm{v}^2_a - \int_\upOmega fv,
\end{equation*}
\noindent for any $v \in H^1_0(\upOmega)$, where $\norm{\cdot}_a$ denotes the norm induced by the inner product (\ref{eq:Sobolev_inner_product}), then $u \in H^1_0(\upOmega)$ solves (\ref{eq:homogeneous_Dirichlet_problem}) if and only if
\begin{equation}
	\label{eq:variational_formulation}
	E(u) = \inf_{v \in H^1_0(\upOmega)}E(v).
\end{equation}
Indeed, suppose $u \in H^1_0(\upOmega)$ is a solution of (\ref{eq:homogeneous_Dirichlet_problem}). Let $v \in H^1_0(\upOmega)$. Then $u = v + w$ for $w := u - v \in H^1_0(\upOmega)$ and we compute
\begin{equation*}
	E(v) = E(u + w) = \frac{1}{2}\norm{u}^2_a + \langle u,w \rangle_a + \frac{1}{2}\norm{w}^2_a - \int_\upOmega f(u + w) = E(u) + \frac{1}{2}\norm{w}^2_a \geq E(u)
\end{equation*}
\noindent with equality if and only if $u = v$ a.e. Conversly, suppose the infimum is attained by some $u \in H^1_0(\upOmega)$. Thus by elementary calculus
\begin{equation}
	\label{eq:derivative}
	0 = \frac{d}{dt}\bigg\vert_{t = 0} E(u + tv) = \langle u, v \rangle_a - \int_\upOmega f v
\end{equation}
\noindent for all $v \in H^1_0(\upOmega)$.\\
Suppose now that $u \in C^\infty(\wbar{\upOmega})$ with $u\vert_{\partial \upOmega} = 0$ solves the variational formulation (\ref{eq:variational_formulation}). Then again integration by parts yields
\begin{equation*}
	\langle u, v \rangle_a - \int_\upOmega f v = -\int_\upOmega \mathrm{div}(X) v - \int_\upOmega fv = \int_\upOmega \del[3]{-\frac{\partial}{\partial x^j}\del[3]{a^{ij}\frac{\partial u}{\partial x^i}} - f} v
\end{equation*}
\noindent for all $v \in C^\infty_c(\upOmega)$ and where $X := \del[1]{a^{ij}\frac{\partial u}{\partial x^i}}_j$. Hence (\ref{eq:derivative}) implies 
\begin{equation*}
	\forall v \in C^\infty_c(\upOmega): \> \int_\upOmega \del[3]{-\frac{\partial}{\partial x^j}\del[3]{a^{ij}\frac{\partial u}{\partial x^i}} - f} v = 0.
\end{equation*}

We might expect that this implies 
\begin{equation*}
	-\frac{\partial}{\partial x^j}\del[3]{a^{ij}\frac{\partial u}{\partial x^i}} = f.
\end{equation*}

That this is indeed the case, is guaranteed by a foundational result in the \emph{calculus of variations} (therefore the name).

\begin{proposition}[{Fundamental Lemma of Calculus of Variations \cite[40]{struwe:fa:2014}}]
	\label{prop:fundamental_lemma}
	Let $\upOmega \subseteq \mathbb{R}^n$ open and $f \in L^1_{\mathrm{loc}}(\upOmega)$. If
	\begin{equation*}
		\forall \varphi \in C^\infty_c(\upOmega): \> \int_\upOmega f\varphi = 0,
	\end{equation*}
	\noindent then $f = 0$ a.e.
\end{proposition}

Thus we recovered a second order partial differential equation from the variational formulation. In fact, this is exactly the boundary value problem (\ref{eq:homogeneous_Dirichlet_problem}) from the beginning of our exposition. This technique, and in particular the fundamental lemma of calculus of variations \ref{prop:fundamental_lemma} will play an important role in our treatment of classical mechanics. However, since we are concerned with smooth manifolds only, we use a version of the fundamental lemma of calculus of variations \ref{prop:fundamental_lemma}, which is fairly easy to prove and hence really deserves the terminology ``lemma''.

\begin{lemma}[Fundamental Lemma of Calculus of Variations, Smooth Version\index{Calculus!of variations!fundamental lemma of}]
	\label{lem:fundamental_lemma}
	Let $\upOmega \subseteq \mathbb{R}^n$ open and $f \in C^\infty(\upOmega)$. If
	\begin{equation*}
		\forall \varphi \in C^\infty_c(\upOmega): \> \int_\upOmega f\varphi = 0,
	\end{equation*}
	\noindent then $f = 0$.
\end{lemma}

\begin{proof}
	Towards a contradiction, assume that $f \neq 0$ on $\upOmega$. Thus there exists $x_0 \in \upOmega$, such that $f(x_0) \neq 0$. Without loss of generality, we may assume that $f(x_0) > 0$, since otherwise, consider $-f$ instead of $f$.	The smoothness of $f$ implies the continuity of $f$ on $\upOmega$. Thus there exists $\delta > 0$, such that $f(x) \in B_{f(x_0)/2}\del[1]{f(x_0)}$ holds for all $x \in B_\delta(x_0)$ or equivalently, $f(x) > f(x_0)/2 > 0$ for all $x \in B_\delta(x_0)$. By lemma $2.22$ \cite[42]{lee:smooth_manifolds:2013}, there exists a smooth bump function $\varphi$ supported in $B_\delta(x_0)$ and $\varphi = 1$ on $\wbar{B}_{\delta/2}(x_0)$. In particular, $\varphi \in C^\infty_c(\upOmega)$. Therefore we have
	\begin{equation*}
		0 = \int_\upOmega f \varphi = \int_{B_\delta(x_0)} f \varphi \geq \int_{B_{\delta/2}(x_0)} f \varphi > \frac{1}{2}f(x_0) \abs[0]{B_{\delta/2}(x_0)} > 0,
	\end{equation*}
	\noindent which is a contradiction.
\end{proof}

\begin{exercise}\footnote{This is exercise $1.2. (b)$ from exercise sheet $1$ of the course \emph{Functional Analysis II} taught by \emph{Prof. Dr. A. Carlotto} at ETHZ in the spring of 2018, which can be found \href{https://metaphor.ethz.ch/x/2018/fs/401-3462-00L/ex/Problems01-FAII.pdf}{here}.}
	Let $\upOmega \subseteq \subseteq \mathbb{R}^n$, $2 \leq p < \infty$ and define $\mathcal{B} := \cbr[0]{v \in C^\infty(\wbar{\upOmega}) : v\vert_{\partial\upOmega} = 0}$. Moreover, define $E_p : \mathcal{B} \to \mathbb{R}$ by $E_p(v) := \int_\upOmega \abs[0]{\nabla v}^p$. Derive the partial differential equation satisfied by minimizers $u \in \mathcal{B}$ of the variational problem $E(u) = \inf_{v \in \mathcal{B}}E(v)$.	
\end{exercise}

\section*{Lagrangian Systems and the Principle of Least Action}
Mechanical systems, i.e. a pendulum, are modelled using the language of differential geometry. Thus it is necessary to introduce the relevant physical terminology. 

\begin{definition}[Configuration Space]
	A finite-dimensional manifold in $\mathsf{Diff}$ is said to be a \bld{configuration space}.
\end{definition}

As easily imaginable in the three dimensional Euclidean space $\mathbb{R}^3$, the motion of a pendulum covered over time is the image of a path in a configuration space.

\begin{definition}[Motion]
	A motion of a configuration space $M$ is defined to be a path $\gamma \in \mathsf{Diff}(J, M)$, where $J \subseteq \mathbb{R}$ is an interval.
\end{definition}

\begin{definition}[State]
	A \bld{state in the configuration space} is defined to be an element of the tangent bundle of the configuration space.
\end{definition}

Also in classical mechanics, one has to rely on basic principles, which are to some extent experimentally verified.

\begin{axiom}[Newton-Laplace Determinacy Principle]
	\label{ax:NL_determinacy_principle}
	A motion in a configuration space is completely determined by a state at some instant.
\end{axiom}

The Newton-Laplace determinacy principle \ref{ax:NL_determinacy_principle} motivates our main definition of this chapter.

\begin{definition}[Lagrangian System]
	A \bld{Lagrangian system}\index{Lagrangian!system} is defined to be a tuple $(M,L)$ consisting of an object $M \in \mathsf{Diff}$ and a morphism $L \in \mathsf{Diff}(TM \times \mathbb{R},\mathbb{R})$, called a \bld{Lagrangian function}\index{Lagrangian!function}.
\end{definition}

\begin{definition}[Path Space]
	\label{def:path_space}
	Let $M \in \mathsf{Diff}$, $q_0,q_1 \in M$ and $t_0, t_1 \in \mathbb{R}$ with $t_0 \leq t_1$. Define the \bld{path space of $M$ connecting $(q_0,t_0)$ and $(q_1,t_1)$}\index{Path space} to be the set
	\begin{equation}
		\label{eq:path_space}
		\mathcal{P}(M)^{q_0,t_0}_{q_1,t_1} := \cbr[0]{\gamma \in \mathsf{Diff}(\intcc[0]{t_0,t_1},M) : \gamma(t_0) = q_0 \text{ and } \gamma(t_1) = q_1}.
			\end{equation}
\end{definition}

\begin{remark}
	For the sake of simplicity, we will just use the terminology \emph{path space} for $\mathcal{P}(M)^{q_0,t_0}_{q_1,t_1}$ and simply write $\mathcal{P}(M)$. We implicitely assume the conditions of definition \ref{def:path_space}, however.
\end{remark}

\begin{definition}[Variation]
	\label{def:variation}
	Let $\mathcal{P}(M)$ be a path space and $\gamma \in \mathcal{P}(M)$. A \bld{variation of $\gamma$}\index{Variation} is defined to be a morphism $\Gamma \in \mathsf{Diff}(\intcc[0]{t_0,t_1} \times \intcc[0]{-\varepsilon_0,\varepsilon_0},M)$ for some $\varepsilon_0 > 0$ and such that
	\begin{itemize}[wide=0pt]
		\item $\Gamma(t,0) = \gamma$ for all $t \in \intcc[0]{t_1,t_0}$.
		\item $\Gamma(t_0,\varepsilon) = q_0$ for all $\varepsilon \in \intcc[0]{-\varepsilon_0,\varepsilon_0}$.
		\item $\Gamma(t_1,\varepsilon) = q_1$ for all $\varepsilon \in \intcc[0]{-\varepsilon_0,\varepsilon_0}$.
	\end{itemize}
\end{definition}

\begin{remark}
	If $\Gamma$ is a variation of $\gamma \in \mathcal{P}(M)$, we write $\gamma_\varepsilon(-) := \Gamma(-,\varepsilon)$ for all $\varepsilon \in \intcc[0]{-\varepsilon_0,\varepsilon_0}$.
\end{remark}

\begin{example}[Perturbation of a Path along a Single Direction]
	\label{ex:perturbation_along_single_direction}
	Let $M \in \mathsf{Diff}$ of dimension $n$, $(U,\varphi)$ a chart and suppose that $\gamma$ is a path in $U$. With respect to this chart, we can write the coordinate representation of $\gamma$ as
	\begin{equation*}
		\gamma(t) = \del[1]{\gamma^1(t),\dots,\gamma^n(t)}
	\end{equation*}
	\noindent for any $t \in \intcc[0]{t_0,t_1}$. Let $f \in C^\infty_c\intoo[0]{t_0,t_1}$. Consider the family $\Gamma : \intcc[0]{t_0,t_1} \times \intcc[0]{-\varepsilon_0,\varepsilon_0} \to M$ defined by
	\begin{equation*}
		\Gamma(t,\varepsilon) := (\iota \circ \varphi^{-1})\del[1]{\gamma^1(t),\dots,\gamma^i(t) + \varepsilon f(t),\dots,\gamma^n(t)}
	\end{equation*}
	\noindent where $\iota : U \hookrightarrow M$ denotes inclusion and $\varepsilon_0 > 0$ is to be determined. By exercise \ref{ex:U_delta_neighbourhood}, there exists $\delta > 0$ such that
	\begin{equation*}
		U_\delta := \cbr[0]{x \in \mathbb{R}^n : \dist(x,\gamma(\intcc[0]{t_0,t_1})) < \delta} \subseteq \varphi(U).
	\end{equation*}
	Choose $\varepsilon_0 > 0$ such that $0 < \varepsilon_0 < \delta/\norm{f}_\infty$. Then in coordinates
	\begin{equation*}
		\dist\del[1]{\gamma_\varepsilon(t), \gamma(\intcc[0]{t_0,t_1})} \leq \abs[0]{\gamma_\varepsilon(t) - \gamma(t)} = \abs{\varepsilon} \norm{f}_\infty \leq \varepsilon_0 \norm{f}_\infty < \delta 
	\end{equation*}
	\noindent for all $t \in \intcc[0]{t_0,t_1}$. Hence $\gamma_\varepsilon(t) \in U_\delta$ and thus $\gamma_\varepsilon(t) \in \varphi(U)$. Therefore, $\Gamma$ is indeed well-defined. Moreover, it is easy to show that the properties of definition \ref{def:variation} holds, therefore, $\Gamma$ is a variation of $\gamma$. In fact, this example shows, that any path $\gamma$ contained in a single chart admits infinitely many variations. An example of such a variation is shown in figure \ref{fig:variation}.

	\begin{figure}[h!tb]
		\centering
		\includegraphics[width = .7\textwidth]{variation.pdf}
		\caption{Example of a variation along the second coordinate using a smooth bump function as in \cite[42]{lee:smooth_manifolds:2013}.}
		\label{fig:variation}
	\end{figure}
\end{example}

\begin{exercise}
	\label{ex:U_delta_neighbourhood}
	Let $U \subseteq \mathbb{R}^n$ open and $A \subseteq U$ closed. Then there exists $\delta > 0$ such that
	\begin{equation*}
		U_\delta := \cbr[0]{x \in \mathbb{R}^n : \dist(x,A) < \delta} \subseteq U.
	\end{equation*}
\end{exercise}

\begin{definition}[Action Functional]
	Let $(M,L)$ be a Lagrangian system and $\mathcal{P}(M)$ be a path space. The morphism $S : \mathcal{P}(M) \to \mathbb{R}$ defined by
	\begin{equation*}
		S(\gamma) := \int_{t_0}^{t_1} L\del[0]{\gamma(t),\dot{\gamma}(t),t} dt
	\end{equation*}
	\noindent is called the \bld{action functional}\index{Action functional}.
\end{definition}

\begin{definition}[Motion of a Lagrangian System]
	A \bld{motion of a Lagrangian system $(M,L)$ between $(q_0,t_0)$ and $(q_1,t_1)$} is defined to be a morphism $\gamma \in \mathsf{Diff}(\intcc[0]{t_0,t_1},M)$.
\end{definition}

Motions of Lagrangian systems are characterized by an axiom.

\begin{axiom}[Hamilton's Principle of Least Action]\index{Hamilton!'s principle of least action}
	\label{ax:Hamilton_least_action}
	Let $(M,L)$ be a Lagrangian system and $\mathcal{P}(M)$ be a path space. A path $\gamma \in \mathsf{Diff}(\intcc[0]{t_0,t_1},M)$ describes a motion of $(M,L)$ between $(q_0,t_0)$ and $(q_1,t_1)$ if and only if 
	\begin{equation}
		\frac{d}{d\varepsilon}\bigg\vert_{\varepsilon = 0} S(\gamma_\varepsilon) = 0
	\end{equation}
	\noindent for all variations $\gamma_\varepsilon$ of $\gamma$.
\end{axiom}

\begin{definition}[Extremal]
	A motion of a Lagrangian system between two points is called an \bld{extremal of the action functional $S$}.
\end{definition}

The Newton-Laplace determinacy principle \ref{ax:NL_determinacy_principle} implies that motions of mechanical systems can be described as solutions of second order differential equations. That this is indeed the case, is shown by the next theorem.

\begin{theorem}[Euler-Lagrange Equations]
	\label{thm:EL_equations}
	Let $(M,L)$ be a Lagrangian system. If a path $\gamma \in \mathsf{Diff}(\intcc[0]{t_0,t_1},M)$ describes a motion of $(M,L)$ between $(q_0,t_0)$ and $(q_1,t_1)$ then for all charts $(U,q^i)$
	\begin{equation}
		\label{eq:EL_equations}
		\pd{L}{q}\del[1]{\gamma(t),\dot{\gamma}(t),t} - \frac{d}{dt} \pd{L}{\dot{q}}\del[1]{\gamma(t),\dot{\gamma}(t),t} = 0
	\end{equation}
	\noindent holds, where $(q,\dot{q})$ denotes the standard coordinates on $TM$. The system of equations \textup{(}\ref{eq:EL_equations}\textup{)} is referred to as the \bld{Euler-Lagrange equations}\index{Euler-Lagrange equations}.
\end{theorem}

\begin{proof}
	By Hamilton's principle of least action \ref{ax:Hamilton_least_action}, we may assume that $\gamma$ is an extremal of the action functional $S$. The proof is divided into two steps.
	\begin{enumerate}[label = \textit{Step \arabic*:},wide=0pt]
		\item \textit{Suppose that the extremal $\gamma$ of $S$ is conatined in a chart domain $U$.} Let $t \in \intcc[0]{t_0,t_1}$ and abreviate $p := (\gamma(t),\dot{\gamma}(t),t)$ Using the formula for the derivative of a function along a curve \cite[283]{lee:smooth_manifolds:2013}, we compute
			\begin{align*}
				\frac{d}{d\varepsilon}\bigg\vert_{\varepsilon = 0} L\del[0]{\gamma_\varepsilon(t),\dot{\gamma}_\varepsilon(t),t} &= dL_p\del[3]{\frac{d}{d\varepsilon}\bigg\vert_{\varepsilon = 0}\gamma_\varepsilon(t),\frac{d}{d\varepsilon}\bigg\vert_{\varepsilon = 0}\dot{\gamma}_\varepsilon(t),0}\\
				&= dL_p\del[3]{\frac{d\gamma_\varepsilon^j(t)}{d\varepsilon}(0)\frac{\partial}{\partial q^j}\bigg\vert_{\gamma(t)},\frac{d\dot{\gamma}_\varepsilon^j(t)}{d\varepsilon}(0)\frac{\partial}{\partial \dot{q}^j}\bigg\vert_{\dot{\gamma}(t)},0}.
			\end{align*}
			Using the formula for the differential of a function in coordinates yields
			\begin{equation*}
				dL_p = \frac{\partial L}{\partial q^i}(p) dq^i\vert_p + \frac{\partial L}{\partial \dot{q}^i}(p) d\dot{q}^i\vert_p + \frac{\partial L}{\partial t}(p)dt\vert_p.
			\end{equation*}
			So
			\begin{align*}
				0 &= \frac{d}{d\varepsilon}\bigg\vert_{\varepsilon = 0} S(\gamma_\varepsilon)\\
				&= \int_{t_0}^{t_1} \frac{d}{d\varepsilon}\bigg\vert_{\varepsilon = 0} L\del[0]{\gamma_\varepsilon(t),\dot{\gamma}_\varepsilon(t),t} dt\\
				&= \int_{t_0}^{t_1} dL_p\del[3]{\frac{d\gamma_\varepsilon^j(t)}{d\varepsilon}(0)\frac{\partial}{\partial q^j}\bigg\vert_{\gamma(t)},\frac{d\dot{\gamma}_\varepsilon^j(t)}{d\varepsilon}(0)\frac{\partial}{\partial \dot{q}^j}\bigg\vert_{\dot{\gamma}(t)},0}\\
				&= \int_{t_0}^{t_1} \frac{\partial L}{\partial q^i}(p)\frac{d\gamma_\varepsilon^i(t)}{d\varepsilon}(0)dt + \int_{t_0}^{t_1}\frac{\partial L}{\partial \dot{q}^i}(p)\frac{d\dot{\gamma}_\varepsilon^i(t)}{d\varepsilon}(0)dt\\
				&= \int_{t_0}^{t_1} \frac{\partial L}{\partial q^i}(p)\frac{d\gamma_\varepsilon^i(t)}{d\varepsilon}(0)dt + \int_{t_0}^{t_1}\frac{\partial L}{\partial \dot{q}^i}(p)\del[3]{\frac{d\gamma_\varepsilon^i(t)}{d\varepsilon}(0)}'dt\\
				&= \int_{t_0}^{t_1} \frac{\partial L}{\partial q^i}(p)\frac{d\gamma_\varepsilon^i(t)}{d\varepsilon}(0)dt + \frac{\partial L}{\partial \dot{q}^i}(p)\frac{d\gamma_\varepsilon^i(t)}{d\varepsilon}(0)\bigg\vert_{t_0}^{t_1} - \int_{t_0}^{t_1} \frac{d}{dt}\frac{\partial L}{\partial \dot{q}^i}(p)\frac{d\gamma_\varepsilon^i(t)}{d\varepsilon}(0)dt.	
			\end{align*}
			\noindent Let $f \in C^\infty_c\intoo[0]{t_0,t_1}$, $j = 1,\dots,n$ and $\gamma_\varepsilon$ be the variation of $\gamma$ defined in example \ref{ex:perturbation_along_single_direction} along the $j$-th direction. Above computation therefore yields
			\begin{align*}
				0 &= \int_{t_0}^{t_1} \frac{\partial L}{\partial q^j}(p)f(t)dt + \frac{\partial L}{\partial \dot{q}^j}(p)f(t)\bigg\vert_{t_0}^{t_1} - \int_{t_0}^{t_1} \frac{d}{dt}\frac{\partial L}{\partial \dot{q}^j}(p)f(t)dt\\
				&= \int_{t_0}^{t_1} \del[3]{\frac{\partial L}{\partial q^j}(p) - \frac{d}{dt}\frac{\partial L}{\partial \dot{q}^j}(p)}f(t) dt
			\end{align*}
			\noindent for all $f \in C^\infty_c\intoo[0]{t_0,t_1}$. Hence the fundamental lemma of calculus of variations \ref{prop:fundamental_lemma} implies
			\begin{equation*}
				\frac{\partial L}{\partial q^j}(p) - \frac{d}{dt}\frac{\partial L}{\partial \dot{q}^j}(p) = 0
			\end{equation*}
			\noindent for all $j = 1,\dots,n$.
		\item \textit{Suppose that $\gamma$ is an arbitrary extremal of $S$.} The key technical result used here is the following lemma.
			\begin{lemma}[Lebesgue Number Lemma]
				\label{lem:Lebesgue_number_lemma}
				Every open cover of a compact metric space admits a Lebesgue number, i.e. a number $\delta > 0$ such that every subset of the metric space with diameter less than $\delta$ is contained in one member of the family.
			\end{lemma}

			\begin{proof}
				See \cite[194]{lee:topological_manifolds:2011}.
			\end{proof}

			Let $(U_\alpha)_{\alpha \in A}$ be the smooth structure on $M$, i.e. the maximal smooth atlas. Since $\gamma$ is continuous, $\del[1]{\gamma^{-1}(U_\alpha)}_{\alpha \in A}$ is an open cover for $\intcc[0]{t_0,t_1}$. By the Lebesgue number lemma \ref{lem:Lebesgue_number_lemma}, this open cover admits a Lebesgue number $\delta > 0$. Let $k \in \mathbb{N}$ such that $(t_1 - t_0)/k < \delta$ and define
			\begin{equation*}
				x_i := \frac{i}{k}(t_1 - t_0) + t_0
			\end{equation*}
			\noindent for all $i = 0,\dots,k$. Then for all $i = 1,\dots,k$, $\gamma\vert_{\intcc[0]{x_{i - 1},x_i}}$ is contained in $U_\alpha$ for some $\alpha \in A$. Hence applying step $1$ yields the result.
	\end{enumerate}
\end{proof}

\begin{definition}[Equations of Motion]
	The Euler-Lagrange equations of a Lagrangian system are also called \bld{equations of motion}.
\end{definition}

\section*{Symmetries and Noether's Theorem}
Solving the Euler-Lagrange equations \ref{thm:EL_equations} in general is a very difficult task. Thus we have to make certain assumptions for deriving general results.

\begin{definition}[Closed Lagrangian System]
	A Lagrangian system $(M,L)$ is said to be \bld{closed}, iff
	\begin{equation*}
		\frac{\partial L}{\partial t} = 0
	\end{equation*}
	\noindent with respect to every chart $U$ of $M$.
\end{definition}

\begin{definition}[Integral of Motion]
	Let $(M,L)$ be a Lagrangian system. An \bld{integral of motion} is defined to be a morphism $I \in C^\infty(TM)$ such that
	\begin{equation*}
		\frac{d}{dt}I\del[1]{\gamma(t),\dot{\gamma}(t)} = 0
	\end{equation*}
	\noindent holds for every extremal $\gamma$ of the action functional.
\end{definition}

Let $(M,L)$ be a Lagrangian system and $(U,q)$ a chart. Denote by $(q,\dot{q})$ the standard coordinates on $TM$ and consider the function $E \in C^\infty(TU \times \mathbb{R})$ defined by
\begin{equation}
	\label{eq:energy}
	E(q,\dot{q},t) := \dot{q}^i \frac{\partial L}{\partial \dot{q}^i}(q,\dot{q},t) - L(q,\dot{q},t).
\end{equation}

\begin{definition}[Energy]
	
\end{definition}


\chapter{Hamiltonian Mechanics}

\section*{Symplectic Geometry}
A profound difference between the tangent bundle $TM$ and the cotangent bundle $T^*M$ of a smooth manifold $M$ is that on the latter there exists a natural $1$-form, the tautological form $\alpha$ defined in definition \ref{def:tautological_form}.  

\subsection*{The Category of Symplectic Manifolds}
Recall that a form $\omega$ on a smooth manifold $M$ is said to be \emph{closed}, iff $d\omega = 0$.

\begin{definition}[Symplectic Manifold]
	A \bld{symplectic manifold}\index{Manifold!symplectic} is defined to be a tuple $(M,\omega)$ consisting of a smooth manifold $M$ and a closed nondegenerate $2$-form $\omega \in \upOmega^2(M)$, called a \bld{symplectic form on $M$}.
\end{definition}

\begin{example}[The Cotangent Bundle]
	\label{ex:cotangent_bundle}
	Let $M$ be a smooth manifold and consider the tautological form $\alpha \in \Omega^1(T^*M)$ defined by $\alpha := \xi_i dx^i$ on a chart $\del[1]{T^*U, (x^i,\xi^i)}$ on $T^*M$. Define $\omega \in \Omega^2(T^*M)$ by $\omega := -d\alpha$. It is immediate that $\omega$ is closed since $d\omega = -(d \circ d)(\alpha) = 0$. Moreover, we compute locally
	\begin{equation*}
		\omega = -d(\xi_idx^i) = -\frac{\partial \xi_i}{\partial x^j} dx^j \wedge dx^i - \frac{\partial \xi_i}{\partial \xi_j} d\xi_j \wedge dx^i = \delta^j_i dx^i \wedge d\xi_j = \sum_i dx^i \wedge d\xi_i.
	\end{equation*}
	Thus $\omega$ is nondegenerate. 
\end{example}

\begin{definition}
	\label{def:symplectic_morphisms}
	A morphism $F : (M,\omega) \to (\wtilde{M},\wtilde{\omega})$ between two symplectic manifolds $(M,\omega)$ and $(\wtilde{M},\wtilde{\omega})$ is defined to be a morphism $F \in C^\infty(M,\wtilde{M})$ such that $F^*\wtilde{\omega} = \omega$. 
\end{definition}

\begin{exercise}
	Consider as objects symplectic manifolds and as morphisms the ones from definition \ref{def:symplectic_morphisms}. Show that they do form a category, the \bld{category of symplectic manifolds}.
\end{exercise}

\begin{definition}[Symplectomorphism]
	A \bld{symplectomorphism}\index{Symplectomorphism} is defined to be an isomorphism in the category of symplectic manifolds.
\end{definition}

\subsection*{The Tangent-Cotangent Bundle Isomorphism}
As in Riemannian geometry, one very important feature of a symplectic manifold $(M,\omega)$ is that there is a canonical identification of the tangent bundle $TM$ and the cotangent bundle $T^*M$ (for the Riemannian case see \cite[341]{lee:smooth_manifolds:2013}). But first we recall some basic facts from the tensor calculus on smooth manifolds. 

%Input tensor fields
\begin{lemma}[{Vector Bundle Chart Lemma \cite[253]{lee:smooth_manifolds:2013}}]
	\label{lem:vector_bundle_chart_lemma}
	Let $M$ be a smooth manifold, $k \in \mathbb{N}$ and suppose that for all $x \in M$ we are given a real vector space $E_x$ of dimension $k$. Let $E := \coprod_{x \in M} E_x$ and let $\pi : E \to M$ be given by $\pi(x,v) := x$. Moreover, suppose that we are given the following data:
	\begin{enumerate}[label = \textup{(\roman*)},leftmargin=*]
		\item An open cover $(U_\alpha)_{\alpha \in A}$ of $M$.
		\item For all $\alpha \in A$ a bijection $\Phi_\alpha : \pi^{-1}(U_\alpha) \to U_\alpha \times \mathbb{R}^k$ such that the restriction $\Phi_\alpha\vert_{E_x} : E_x \to \cbr{x} \times \mathbb{R}^k \cong \mathbb{R}^k$ is an isomorphism of vector spaces for all $x \in M$.
		\item For all $\alpha, \beta \in A$ with $U_\alpha \cap U_\beta \neq \varnothing$, a smooth mapping $\tau_{\alpha\beta} : U_\alpha \cap U_\beta \to \GL(k,\mathbb{R})$  such that the mapping $\Phi_\alpha \circ \Phi_\beta^{-1} : (U_\alpha \cap U_\beta) \times \mathbb{R}^k \to (U_\alpha \cap U_\beta) \times \mathbb{R}^k$ is of the form $\Phi_\alpha \circ \Phi_\beta^{-1}(x,v) = \del[1]{x,\tau_{\alpha\beta}(x)v}$. 
	\end{enumerate}
	Then $E$ admits a unique topology and a smooth structure making it into a smooth manifold and a smooth vector bundle $\pi : E \to M$ of rank $k$ with local trivializations $(U_\alpha,\Phi_\alpha)_{\alpha \in A}$.
\end{lemma}

Let $M^n$ be a smooth manifold and let $k,l \in \mathbb{N}$. For all $x \in M$ define the space of \bld{mixed tensors of type $(k,l)$ on $T_xM$} by
\begin{equation*}
	T^{(k,l)}(T_xM) := \undercbrace{T_xM \otimes \dots \otimes T_xM}_{k} \otimes \undercbrace{T^*_xM \otimes \dots \otimes T^*_xM}_{l}.
\end{equation*}
By proposition 12.10 \cite[311]{lee:smooth_manifolds:2013} we have that 
\begin{equation*}
	T^{(k,l)}(T_xM) \cong L\del[1]{\undercbrace{T^*_xM, \dots, T^*_xM}_{k},\undercbrace{T_xM,\dots,T_xM}_{l};\mathbb{R}}
\end{equation*}
\noindent since $(T^*_xM)^* \cong T_xM$ canonically ($T_xM$ is finite-dimensional) where the latter denotes the space of all $\mathbb{R}$-valued multilinear forms on
\begin{equation*}
	\undercbrace{T^*_xM \times \dots \times T^*_xM}_{k} \times \undercbrace{T_xM \times \dots \times T_xM}_{l}.
\end{equation*}
We will always think of mixed tensors as multilinear forms. Let $(U,x^i)$ be a chart about $x$. Then using corollary 12.12 \cite[313]{lee:smooth_manifolds:2013} we get that a basis for $T^{(k,l)}(T_xM)$ is given by all elements 
\begin{equation*}
	\frac{\partial}{\partial x^{i_1}}\bigg\vert_x \otimes \dots \otimes \frac{\partial}{\partial x^{i_k}}\bigg\vert_x \otimes dx^{j_1}\vert_x \otimes \dots \otimes dx^{j_l}\vert_x
\end{equation*}
\noindent for all $1 \leq i_1,\dots,i_k,j_1,\dots,j_l \leq n$. Consequently, $\dim T^{(k,l)}(T_xM) = n^{k + l}$ and a particular tensor $A \in T^{(k,l)}(T_xM)$ expressed in this basis is given by 
\begin{equation}
	\label{eq:tensor_expression_basis}
	A = A^{i_1\dots i_k}_{j_1\dots j_l}\frac{\partial}{\partial x^{i_1}}\bigg\vert_x \otimes \dots \otimes \frac{\partial}{\partial x^{i_k}}\bigg\vert_x \otimes dx^{j_1}\vert_x \otimes \dots \otimes dx^{j_l}\vert_x
\end{equation}
\noindent where
\begin{equation}
	\label{eq:tensor_components}
	A^{i_1\dots i_k}_{j_1\dots j_l} := A\del[3]{dx^{i_1}\vert_x,\dots,dx^{i_k}\vert_x,\frac{\partial}{\partial x^{j_1}}\bigg\vert_x,\dots\frac{\partial}{\partial x^{j_l}}\bigg\vert_x}.
\end{equation}
Next we want to ``glue'' together the different spaces of mixed tensors. 

\begin{proposition}
	\label{prop:tensor_bundle}
	Let $M$ be a smooth manifold and let $k,l \in \mathbb{N}$. Then
	\begin{equation*}
		T^{(k,l)}TM := \coprod_{x \in M} T^{(k,l)}(T_xM)
	\end{equation*}
	\noindent admits a unique topology and a smooth structure making it into a smooth manifold and a smooth vector bundle $\pi : T^{(k,l)}TM \to M$ of rank $n^{k + l}$. This smooth vector bundle is called the \bld{bundle of mixed tensors of type $(k,l)$ on $M$}. 
\end{proposition}

\begin{proof}
	This is an application of the vector bundle chart lemma \ref{lem:vector_bundle_chart_lemma}. For all $x \in M$ define $E_x := T^{(k,l)}(T_xM)$. By the preceeding discussion, $\dim E_x = n^{k + l}$. Let $(U_\alpha,\varphi_\alpha)_{\alpha \in A}$ denote the smooth structure on $M$. Then clearly $(U_\alpha)_{\alpha \in A}$ is an open cover for $M$. For each $\alpha \in A$, define
\begin{equation*}
	\Phi_\alpha:\ccases{
		\pi^{-1}(U_\alpha) \to U_\alpha \times \mathbb{R}^{n^{k + l}}\\
		(x,A) \mapsto \del[1]{x,(A^{i_1\dots i_k}_{j_1\dots j_l})}
	}
\end{equation*}
\noindent where we expressed $A$ as in (\ref{eq:tensor_expression_basis}). Observe, that this map strongly depends on the coordinate functions. Clearly, the inverse is given by
\begin{equation*}
	\Phi^{-1}_\alpha:\ccases{
		U_\alpha \times\mathbb{R}^{n^{k + l}} \to \pi^{-1}(U_\alpha)\\
		\del[1]{x,(A^{i_1\dots i_k}_{j_1\dots j_l})} \mapsto (x,A)
	}.
\end{equation*}
	Hence each $\Phi_\alpha$ is bijective. Now we have to check, that $\Phi_\alpha\vert_{E_x}$ is an isomorphism for all $x \in M$. By elementary linear algebra it is enough to show that $\Phi_\alpha$ is linear. So let $\lambda \in \mathbb{R}$ and $A,B \in E_x$. Then
	\begin{align*}
		\Phi_\alpha\vert_{E_x}(x,A + \lambda B) &= \del[1]{x,(A + \lambda B)^{i_1\dots i_k}_{j_1\dots j_l})}\\
		&= \del[1]{x,(A^{i_1\dots i_k}_{j_1\dots j_l}) + \lambda (B^{i_1\dots i_k}_{j_1\dots j_l})}\\
		&= \Phi_\alpha\vert_{E_x}(x,A) + \lambda\Phi_\alpha\vert_{E_x}(x,B).
	\end{align*}
	Lastly, let $\alpha, \beta \in A$ such that $U_\alpha \cap U_\beta \neq \varnothing$ and coordinates $(x^i_\alpha)$ and $(x^i_\beta)$, respectively. Then for $x \in U_\alpha \cap U_\beta$ we have that
	\begin{equation*}
		\frac{\partial}{\partial x_\alpha^i}\bigg\vert_x = \frac{\partial x^j_\beta}{\partial x_\alpha^i}(x)\frac{\partial}{\partial x_\beta^j}\bigg\vert_x \qquad \text{and} \qquad dx_\alpha^i\vert_x = \frac{\partial x_\alpha^i}{\partial x_\beta^j}(x)dx^j_\beta\vert_x.
	\end{equation*}
	So if $A^{i_1\dots i_k}_{j_1\dots j_l}$ are coordinates of a mixed tensor with respect to the basis induced by $(x^i_\alpha)$, we compute
	\begin{align*}
		A^{i_1\dots i_k}_{j_1\dots j_l} &= A\del[3]{dx_\alpha^{i_1}\vert_x,\dots,dx_\alpha^{i_k}\vert_x,\frac{\partial}{\partial x_\alpha^{j_1}}\bigg\vert_x,\dots\frac{\partial}{\partial x_\alpha^{j_l}}\bigg\vert_x}\\
		&= \frac{\partial x_\alpha^{i_1}}{\partial x^{p_1}_\beta}(x)\cdots\frac{\partial x_\alpha^{i_k}}{\partial x_\beta^{p_k}}(x)\frac{\partial x^{q_1}_\beta}{\partial x_\alpha^{j_1}}(x) \cdots \frac{\partial x^{q_l}_\beta}{\partial x_\alpha^{j_l}}(x)A^{p_1\dots p_k}_{q_1\dots q_l}
	\end{align*}
	Thus define $\tau_{\alpha\beta}: U_\alpha \cap U_\beta \to \mathrm{GL}(n^{k + l},\mathbb{R})$ by
	\begin{equation*}
		\tau_{\alpha\beta}(x) := \del[4]{\frac{\partial x_\alpha^{i_1}}{\partial x^{p_1}_\beta}(x)\cdots\frac{\partial x_\alpha^{i_k}}{\partial x_\beta^{p_k}}(x)\frac{\partial x^{q_1}_\beta}{\partial x_\alpha^{j_1}}(x) \cdots \frac{\partial x^{q_l}_\beta}{\partial x_\alpha^{j_l}}(x)}.
	\end{equation*}
	Then $\tau_{\alpha\beta}$ is clearly smooth and moreover
	\begin{equation*}
		\Phi_\alpha \circ \Phi_\beta^{-1}\del[1]{x,(A^{p_1\dots p_k}_{q_1\dots q_l})} = \del[1]{x, (A^{i_1\dots i_k}_{j_1\dots j_l})} = \del[1]{x,\tau_{\alpha\beta}(x)(A^{p_1\dots p_k}_{q_1\dots q_l})}. 
	\end{equation*}
	Therefore, conditions (i)-(iii) in the vector bundle chart lemma \ref{lem:vector_bundle_chart_lemma} are satisfied and the statement follows.
\end{proof}

Recall, that in a category $\mathcal{C}$, a \emph{section} of a morphism $f : X \to Y$ is a morphism $\sigma : Y \to X$ such that $f \circ \sigma = \id_Y$.

\begin{definition}[Tensor Field]
	Let $M$ be a smooth manifold and $k,l \in \mathbb{N}$. A \bld{smooth tensor field of type $(k,l)$ on $M$} is defined to be a section of $\pi : T^{(k,l)}TM \to M$. The space of all smooth tensor fields of type $(k,l)$ on $M$ is denoted by $\Gamma\del[1]{T^{(k,l)}TM}$.
\end{definition}

\begin{example}[Vector Field and Covector Field]
	\label{ex:vector_and_covector_fields}
	Let $M$ be a smooth manifold. Of particular importance are the tensor fields such that $k + l = 1$. If $k = 1$, such tensor fields are called \bld{vector fields} and we write $\mathfrak{X}(M) := \Gamma\del[1]{T^{(1,0)}TM}$. Likewise, if $l = 1$, we call such tensor fields \bld{covector fields} and write $\mathfrak{X}^*(M) := \Gamma\del[1]{T^{(0,1)}TM}$.
\end{example}

Let $\del[1]{U,(x^i)}$ be a chart on $M$ and $A : M \to T^{(k,l)}TM$ such that $A_x \in T^{(k,l)}(T_xM)$ for all $x \in M$. From (\ref{eq:tensor_expression_basis}) we get that
\begin{equation*}
	A_x = A^{i_1\dots i_k}_{j_1\dots j_l}(x)\frac{\partial}{\partial x^{i_1}}\bigg\vert_x \otimes \dots \otimes \frac{\partial}{\partial x^{i_k}}\bigg\vert_x \otimes dx^{j_1}\vert_x \otimes \dots \otimes dx^{j_l}\vert_x
\end{equation*}
\noindent for all $x \in U$ where $A^{i_1\dots i_k}_{j_1\dots j_l} : U \to \mathbb{R}$ are given as in (\ref{eq:tensor_components}). We will call these functions the \bld{component functions of $A$}. Recall, that a map $F : M \to N$ between two smooth manifolds $M$ and $N$ is said to be \emph{smooth}, iff for every $x \in M$ there exists a chart $(U,\varphi)$ about $x$ on $M$ and a chart $(V,\psi)$ about $F(x)$ on $N$ such that $U \cap F^{-1}(V)$ is open in $M$ and $\psi \circ F \circ \varphi^{-1} : \varphi\del[1]{U \cap F^{-1}(V)} \to \psi(V)$ is smooth. Moreover, if $A \subseteq U \subseteq M$, where $U$ is open and $A$ is closed in $M$, a function $\psi \in C^\infty(M)$ is said to be a \emph{smooth bump function for $A$ supported in $U$}, iff $0 \leq \psi \leq 1$, $\psi \vert_A = 1$ and $\supp \psi \subseteq U$. The paracompactness condition guarantees that smooth bump functions exist in great abundance.

\begin{proposition}[{Existence of Smooth Bump Functions \cite[44]{lee:smooth_manifolds:2013}}]
	\label{prop:existence_smooth_bump_functions}
	Let $M$ be a smooth manifold and $A \subseteq U \subseteq M$, where $U$ is open and $A$ is closed in $M$. Then there exists a smooth bump function for $A$ supported in $U$. 	
\end{proposition}

\begin{proposition}[{Smoothness Criteria for Tensor Fields \cite[317]{lee:smooth_manifolds:2013}}]
	\label{prop:smoothness_criteria_for_tensor_fields}
	Let $M$ be smooth manifold, $k,l \in \mathbb{N}$ and $A : M \to T^{(k,l)}TM$ such that $A_x \in T^{(k,l)}T_xM$ for all $x \in M$. Then the following conditions are equivalent:
	\begin{enumerate}[label = \textup{(\alph*\textup)},leftmargin=*]
		\item $A \in \Gamma\del[1]{T^{(k,l)}TM}$.
		\item In every smooth coordinate chart, the component functions of $A$ are smooth.
		\item Each point of $M$ is contained in a chart in which $A$ has smooth component functions.
		\item For all $\omega^1,\dots,\omega^k \in \mathfrak{X}^*(M)$ and $X_1,\dots,X_l \in \mathfrak{X}(M)$, the function 
			\begin{equation*}
				\mathcal{A}(\omega^1,\dots,\omega^k,X_1,\dots,X_l) : M \to \mathbb{R}
			\end{equation*}
			\noindent defined by
			\begin{equation}
				\label{eq:curly_A}
				\mathcal{A}\del[1]{\omega^1,\dots,\omega^k,X_1,\dots,X_l}(x) := A_x\del[1]{\omega^1_x,\dots,\omega^k_x, X_1\vert_x,\dots,X_l\vert_x}
			\end{equation}
			\noindent is smooth.
		\item Let $U \subseteq M$ be open. If $\omega^1,\dots,\omega^k \in \mathfrak{X}^*(U)$ and $X_1,\dots,X_l \in \mathfrak{X}(U)$, then $\mathcal{A}$ defined by \textup{(\ref{eq:curly_A})} belongs to $C^\infty(U)$.
	\end{enumerate}
\end{proposition}

\begin{proof}
We prove (a) $\Leftrightarrow$ (b) and (b) $\Rightarrow$ (c) $\Rightarrow$ (d) $\Rightarrow$ (e) $\Rightarrow$ (b).\\
To prove (a) $\Leftrightarrow$ (b), let $x \in M$ and $\del[1]{U,(x^i)}$ be a smooth chart on $M$ about $x$. Proposition \ref{prop:tensor_bundle} yields a map $\Phi_U : \pi^{-1}(U) \to U \times \mathbb{R}^{n^{k + l}}$, and the proof of the vector bundle chart lemma implies, that the corresponding chart on $T^{(k,l)}TM$ is given by $\del[1]{\pi^{-1}(U),\wtilde{\varphi}}$, where 
\begin{equation*}
	\wtilde{\varphi}: \pi^{-1}(U) \to \varphi(U) \times \mathbb{R}^{n^{k+l}}
\end{equation*}
\noindent is defined by
\begin{equation*}
	\wtilde{\varphi} := \del[1]{\varphi \times \id_{\mathbb{R}^{n^{k+l}}}} \circ \Phi_U.
\end{equation*}
Since $A_x \in T^{(k,l)}T_xM$ for all $x \in M$, we have that 
\begin{equation*}
A^{-1}\del[1]{\pi^{-1}(U)} = (\pi \circ A)^{-1}(U) = \id_M(U) = U.
\end{equation*}
Hence $U \cap A^{-1}\del[1]{\pi^{-1}(U)} = U$, which is open in $M$, and 
\begin{equation*}
	\wtilde{\varphi} \circ A \circ \varphi^{-1} : \varphi(U) \to \wtilde{\varphi}\del[1]{\pi^{-1}(U)} 
\end{equation*}
\noindent is given by
\begin{align*}
	\del[1]{\wtilde{\varphi} \circ A \circ \varphi^{-1}}\del[1]{\varphi(y)} &= \del[1]{\varphi \times \id_{\mathbb{R}^{n^{k+l}}}}\del[1]{\Phi_U(A_y)}\\
	&= \del[1]{\varphi(y),(A^{i_1\dots i_k}_{j_1\dots j_l})(y)}\\
	&= \del[1]{\varphi(y),\del[1]{(A^{i_1\dots i_k}_{j_1\dots j_l}) \circ {\varphi^{-1}}}\del[1]{\varphi(y)}}
\end{align*}
\noindent for all $y \in U$. Thus $\wtilde{\varphi} \circ A \circ \varphi^{-1}$ is smooth if and only if $(A^{i_1\dots i_k}_{j_1\dots j_l}) \circ {\varphi^{-1}}$ is smooth, which is equivalent to $A^{i_1\dots i_k}_{j_1\dots j_l}$ being smooth.\\
The implication (b) $\Rightarrow$ (c) is immediate.\\
To prove (c) $\Rightarrow$ (d), suppose $x \in M$ and let $(U,(x^i))$ be a chart about $x$ such that the component functions of $A$ are smooth. By example \ref{ex:vector_and_covector_fields} and the equivalence (a) $\Leftrightarrow$ (b) we have
\begin{equation*}
	\omega^i = \omega_j^i dx^j \qquad \text{and} \qquad X_i = X^j_i \frac{\partial}{\partial x^j}
\end{equation*}
\noindent on $U$ for smooth functions $\omega_j^i$ and $X^j_i$. Thus for any $y \in U$ we compute
\begin{align*}
	\mathcal{A}\del[1]{\omega^1,\dots,\omega^k,X_1,\dots,X_l}(y) &= A_x\del[1]{\omega^1_x,\dots,\omega^k_x, X_1\vert_x,\dots,X_l\vert_x}\\
	&= \omega^1_{i_1}(y) \cdots \omega^k_{i_k}(y)X_1^{j_1}(y)\cdots X_l^{j_l}(y) A^{i_1\dots i_k}_{j_1\dots j_l}(y)
\end{align*}
\noindent and so $\mathcal{A}\del[1]{\omega^1,\dots,\omega^k,X_1,\dots,X_l}$ is smooth.\\
To prove (d) $\Rightarrow$ (e), we use the fact that smoothness is a local property. Let $x \in U$ and suppose $(V,\varphi)$ is a chart on $U$ centered at $x$. Then $\varphi(V) \subseteq \mathbb{R}^n$ is open and so we find $\varepsilon > 0$ such that $B_\varepsilon(0) \subseteq \varphi(V)$. Set $A := \varphi^{-1}\del[1]{\wbar{B}_{\varepsilon/2}(0)} \subseteq U$. Then $A$ is closed in $U$ and by proposition \ref{prop:existence_smooth_bump_functions} there exists a smooth bump function $\psi \in C^\infty(U)$ for $A$ supported in $U$. Define $\wtilde{\omega}^i : M \to T^*M$ and $\wtilde{X}_i : M \to TM$ by
\begin{equation*}
	\wtilde{\omega}^i_x := \ccases{
		\psi(x)\omega^i_x & x \in U,\\
		0_x & x \in M \setminus \supp \psi,
	} \quad \text{and} \quad \wtilde{X}_i\vert_x := \ccases{
		\psi(x)X_i\vert_x & x \in U,\\
		0_x & x \in M \setminus \supp \psi.
	}
\end{equation*}
Then $\wtilde{\omega}^i \in \mathfrak{X}^*(M)$ and $\wtilde{X}_i \in \mathfrak{X}(M)$ by the gluing lemma for smooth maps (see \cite[35]{lee:smooth_manifolds:2013}). Moreover, on $\varphi^{-1}\del[1]{B_{\varepsilon/2}(0)}$ we have that $\wtilde{\omega}^i = \omega^i$ and $\wtilde{X}_i = X_i$. But then also	
\begin{equation*}
	\mathcal{A}\del[1]{\wtilde{\omega}^1,\dots,\wtilde{\omega}^k,\wtilde{X_1},\dots,\wtilde{X}_l} = \mathcal{A}\del[1]{\omega^1,\dots,\omega^k,X_1,\dots,X_l}
\end{equation*}
\noindent on this neighbourhood, and so since the former is smooth by assumption, so is the latter.
Finally, to prove (e) $\Rightarrow$ (b), let $(U,(x^i))$ be a chart about $x \in M$. Consider $\omega^i \in \mathfrak{X}^*(U)$ and $X_i \in \mathfrak{X}(U)$ defined by
\begin{equation*}
	\omega^i := \delta^i_j dx^j \qquad \text{and} \qquad X_i := \delta^j_i \frac{\partial}{\partial x^j}.
\end{equation*}
Then it is easy to verify that
\begin{equation*}
	\mathcal{A}\del[1]{\omega^{i_1},\dots,\omega^{i_k},X_{j_1},\dots,X_{j_l}} = A^{i_1\dots i_k}_{j_1\dots j_l}
\end{equation*}
\noindent holds on $U$. Thus by assumption, each component function is smooth.
\end{proof}

Part (d) of the smoothness criteria for tensor fields \ref{prop:smoothness_criteria_for_tensor_fields} implies that for any tensor field $A \in \Gamma\del[1]{T^{(k,l)}TM}$ there is a mapping 
\begin{equation*}
	\mathcal{A} : \undercbrace{\mathfrak{X}^*(M) \times \dots \times \mathfrak{X}^*(M)}_{k} \times \undercbrace{\mathfrak{X}(M) \times \dots \times \mathfrak{X}(M)}_{l} \to C^\infty(M)
\end{equation*}
\noindent defined by 
\begin{equation*}
	\del[1]{\omega^1,\dots,\omega^k,X_1,\dots,X_l} \mapsto \mathcal{A}\del[1]{\omega^1,\dots,\omega^k,X_1,\dots,X_l}.
\end{equation*}
We will call this mapping the \bld{map induced by the tensor field $A$}.

\begin{theorem}[{Tensor Field Characterisation Lemma \cite[318]{lee:smooth_manifolds:2013}}]
	\label{thm:tensor_field_characterisation_lemma}
	Let $M$ be a smooth manifold and $k,l \in \mathbb{N}$. A mapping
\begin{equation*}
	\mathcal{A}: \undercbrace{\mathfrak{X}^*(M) \times \dots \times \mathfrak{X}^*(M)}_{k} \times \undercbrace{\mathfrak{X}(M) \times \dots \times \mathfrak{X}(M)}_{l} \to C^\infty(M) 
	\end{equation*}
	\noindent is induced by a $(k,l)$-tensor field if and only if $\mathcal{A}$ is multilinear over $C^\infty(M)$.
\end{theorem}

\begin{proof}
	Suppose $\mathcal{A}$ is induced by a $(k,l)$-tensor field $A$. Let $\omega^1,\dots,\omega^k,\wtilde{\omega}^i \in \mathfrak{X}^*(M)$ and $X_1,\dots,X_l \in \mathfrak{X}(M)$ as well as $f \in C^\infty(M)$. Then for any $x \in M$ we compute
	\begin{align*}
		\mathcal{A}\del[1]{\dots,\omega^i + f\wtilde{\omega}^i,\dots}(x) &= A_x\del[1]{\dots,\omega^i_x + f(x)\wtilde{\omega}^i_x,\dots}\\
		&= A_x\del[1]{\dots,\omega^i_x,\dots} + f(x)A_x\del[1]{\dots,\wtilde{\omega}^i_x,\dots}\\
		&= \mathcal{A}\del[1]{\dots,\omega^i,\dots}(x) + f(x)\mathcal{A}\del[1]{\dots,\wtilde{\omega}^i,\dots}(x)\\
		&= \del[1]{\mathcal{A}\del[1]{\dots,\omega^i,\dots} + f\mathcal{A}\del[1]{\dots,\wtilde{\omega}^i,\dots}}(x).
	\end{align*}
	Thus $\mathcal{A}$ is $C^\infty(M)$-multilinear with respect to the first $k$ arguments. Similarly, $\mathcal{A}$ is $C^\infty(M)$-multilinear with repect to the last $l$ arguments.\\
Conversly, suppose that
	\begin{equation*}
		\mathcal{A}: \undercbrace{\mathfrak{X}^*(M) \times \dots \times \mathfrak{X}^*(M)}_{k} \times \undercbrace{\mathfrak{X}(M) \times \dots \times \mathfrak{X}(M)}_{l} \to C^\infty(M)
	\end{equation*}
	\noindent is $C^\infty(M)$-multilinear. We wish to define a $(k,l)$-tensor field $A$ that induces $\mathcal{A}$. That this is indeed possible, is the observation that $\mathcal{A}\del[1]{\omega^1,\dots,\omega^k,X_1,\dots,X_l}(x)$ only depends on $\omega^1_x,\dots,\omega^k_x,X_1\vert_x,\dots,X_l\vert_x$. Thus we divide the remaining proof into three steps.\\
	\emph{Step 1: $\mathcal{A}\del[1]{\omega^1,\dots,\omega^k,X_1,\dots,X_l}$ acts locally.} That is, if either some $\omega^i$ or $X_i$ vanish on an open set $U$, then so does $\mathcal{A}\del[1]{\omega^1,\dots,\omega^k,X_1,\dots,X_l}$. Let $x \in U$ and $\psi \in C^\infty(M)$ be a smooth bump function for $\cbr{x}$ supported in $U$. Then $\psi\omega^i = 0$ on $M$ and by $C^\infty(M)$-multilinearity 
	\begin{equation*}
		0 = \mathcal{A}\del[1]{\dots,\psi\omega^i,\dots} = \psi(x)\mathcal{A}\del[1]{\dots,\omega^i,\dots}(x) = \mathcal{A}\del[1]{\dots,\omega^i,\dots}(x).
	\end{equation*}
	An analogous argument works if some $X_i$ vanishes on $U$.\\
	\emph{Step 2: $\mathcal{A}\del[1]{\omega^1,\dots,\omega^k,X_1,\dots,X_l}$ acts pointwise.} Thats is, if $\omega^i_x$ or $X_i\vert_x$ vanish for some $x \in M$, then so does $\mathcal{A}\del[1]{\omega^1,\dots,\omega^k,X_1,\dots,X_l}$. Let $(U,(x^i))$ be a chart about $x$. Then $\omega^i = \omega_j^i dx^j$ on $U$. Let $\psi \in C^\infty(U)$ denote the smooth bump function used in the proof of part (d) $\Rightarrow$ (e) of the smoothness criteria for tensor fields \ref{prop:smoothness_criteria_for_tensor_fields}. Define
	\begin{equation*}
		\varepsilon^j := \ccases{
			\psi(x)dx^j\vert_x & x \in U,\\
			0_x & x \in M \setminus \supp \psi,
		} \quad \text{and} \quad 
		f^i_j := \ccases{
			\psi(x)\omega^i_j(x) & x \in U,\\
			0_x & x \in M \setminus \supp \psi.
		}
	\end{equation*} 
	Then $\omega^i = f^i_j \varepsilon^j$ on a neighbourhood of $x$ and so by multilinearity and step 1, we have that
	\begin{equation*}
		\mathcal{A}\del[1]{\dots,\omega^i,\dots} = f^i_j\mathcal{A}\del[1]{\dots,\varepsilon^j,\dots}
	\end{equation*}
	\noindent on a neighbourhood of $x$. But since $\omega^i_x$ vanishes so does each $\omega^i_j(x)$. Hence 
	\begin{equation*}
		\mathcal{A}\del[1]{\dots,\omega^i,\dots}(x) = f^i_j(x)\mathcal{A}\del[1]{\dots,\varepsilon^j,\dots}(x) = \omega^i_j(x)\mathcal{A}\del[1]{\dots,\varepsilon^j,\dots}(x) = 0.
	\end{equation*}
	An analogous argument works if some $X_i\vert_x$.\\
	\emph{Step 3: Definition of the $(k,l)$-tensor field $A$ inducing $\mathcal{A}$.} Let $x \in M$, $\omega^1,\dots,\omega^k \in T_x^*M$ and $v_1,\dots,v_l \in T_xM$. Suppose that $\wtilde{\omega}^1,\dots,\wtilde{\omega}^k \in \mathfrak{X}^*(M)$ and $\wtilde{X}_1,\dots,\wtilde{X}_l \in \mathfrak{X}(M)$ are any extensions, respectively. That is, $\wtilde{\omega}^i_x = \omega^i$ and $\wtilde{X}_i\vert_x = v_i$. They do always exist, since in a chart $(U,(x^i))$ we may write
	\begin{equation*}
		\omega^i = \omega^i_j dx^j\vert_x \qquad \text{and} \qquad v_i = v_i^j \frac{\partial}{\partial x^j}\bigg\vert_x
	\end{equation*}
	\noindent and so using a smooth bump function for $\cbr{x}$ supported in $U$ we can construct global maps as in step 2 if we consider the components as constant functions. Now define
	\begin{equation}
		\label{eq:def_A}
		A_x\del[1]{\omega^1,\dots,\omega^k,v_1,\dots,v_l} := \mathcal{A}\del[1]{\wtilde{\omega}^1,\dots,\wtilde{\omega}^k,\wtilde{X}_1,\dots,\wtilde{X}_l}(x).
	\end{equation}
	This is well-defined by step 2. Now if $\omega^1,\dots,\omega^k \in \mathfrak{X}^*(M)$ and $X_1,\dots,X_l \in \mathfrak{X}(M)$, we have that 
	\begin{equation*}
		\mathcal{A}\del[1]{\omega^1,\dots,\omega^k,X_1,\dots,X_l}(x) = A_x\del[1]{\omega^1_x,\dots,\omega^k_x,X_1\vert_x,\dots,X_l\vert_x},
	\end{equation*}
	\noindent since $\omega^i$ and $X_i$ are extensions of $\omega^i_x$ and $X_i\vert_x$, respectively, for all $x \in M$. So the assumption that $\mathcal{A}$ takes values in the space of smooth functions $C^\infty(M)$ together with part (d) of the smoothness criteria for tensor fields \ref{prop:smoothness_criteria_for_tensor_fields} yields that $A$ is a smooth $(k,l)$-tensor field which moreover induces $\mathcal{A}$.
\end{proof}


\begin{theorem}[{Bundle Homomorphism Characterisation Lemma \cite[262]{lee:smooth_manifolds:2013}}]
	\label{thm:bundle_homomorphism_characterisation_lemma}
	Let $\pi : E \to M$ and $\wtilde{\pi} : \wtilde{E} \to M$ be smooth vector bundles over a smooth manifold $M$. A map $\mathcal{F} : \upGamma(E) \to \upGamma(\wtilde{E})$ is linear over $C^\infty(M)$ if and only if there exists a smooth bundle homomorphism $F : E \to \wtilde{E}$ over $M$ such that $\mathcal{F}(\sigma) = F \circ \sigma$ for all $\sigma \in \upGamma(E)$.
\end{theorem}

\begin{proposition}[Tangent-Cotangent Bundle Isomorphism]
	\label{prop:tangent-cotangent_bundle_isomorphism}
	Let $(M,\omega)$ be a symplectic manifold. Define $\Omega : TM \to T^*M$ by
	\begin{equation}
		\label{eq:tangent-cotangent_isomorphism}
		\Omega(v)(w) := \omega_x(v,w)
	\end{equation}
	\noindent for all $x \in M$ and $v,w \in T_xM$. Then $\Omega$ is a well-defined smooth bundle isomorphism. The morphism $\Omega$ is called the \bld{tangent-cotangent bundle isomorphism}\index{Isomorphism!tangent-cotangent bundle}.
\end{proposition}

\begin{proof}
	Using the tensor field characterisation lemma \ref{thm:tensor_field_characterisation_lemma}, $\omega$ induces a map 
	\begin{equation*}
		\omega : \mathfrak{X}(M) \times \mathfrak{X}(M) \to C^\infty(M)
	\end{equation*}
	\noindent which is $C^\infty(M)$-multilinear. Thus for $X \in \mathfrak{X}(M)$ we define $\Omega_X : \mathfrak{X}(M) \to C^\infty(M)$ by
	\begin{equation*}
		\Omega_X(Y) := \omega(X,Y).
	\end{equation*}
	Since $\omega$ is multilinear over $C^\infty(M)$, so is $\Omega_X$, and thus again by the tensor field characterisation lemma \ref{thm:tensor_field_characterisation_lemma}, $\Omega_X$ belongs to $\mathfrak{X}^*(M)$. Hence we get a map $\Omega : \mathfrak{X}(M) \to \mathfrak{X}^*(M)$ by $\Omega(X) := \Omega_X$ which is also multilinear over $C^\infty(M)$. Finally, by the bundle homomorphism characterisation lemma \ref{thm:bundle_homomorphism_characterisation_lemma}, there exists a smooth vector bundle homomorphism $\Omega : TM \to T^*M$ such that $\Omega_X = \Omega \circ X$ for all $X \in \mathfrak{X}(M)$. Let $x \in M$, $v,w \in T_xM$ and $V,W \in \mathfrak{X}(M)$ be extensions of $v$ and $w$, respectively (see step 3 in the proof of the tensor field characterisation lemma \ref{thm:tensor_field_characterisation_lemma}). We compute
	\begin{equation*}
		\Omega_V\vert_x(w) = \Omega_V(W)(x) = \omega(V,W)(x) = \omega_x(V\vert_x,W\vert_x) = \omega_x(v,w)
	\end{equation*}
	\noindent and since $(\Omega \circ V)\vert_x(w) = \Omega(V\vert_x)(w) = \Omega(v)(w)$, we have that $\Omega$ coincides with the map defined in (\ref{eq:tangent-cotangent_isomorphism}). Next we show that $\Omega$ is injective. Let $v,\wtilde{v} \in TM$ such that $\Omega(v) = \Omega(\wtilde{v})$. Since $\Omega$ is a fibrewise mapping, we must have that $v,\wtilde{v} \in T_xM$ for some $x \in M$. Moreover, by definition we have that $\omega_x(v - \wtilde{v},w) = 0$ for every $w \in T_xM$. By nondegeneracy, it follows that $v = \wtilde{v}$. Moreover, since $T_xM$ is finite-dimensional, we get that $\Omega$ is also surjective, thus bijective. Since any bijective smooth bundle homomorphism over $M$ is automatically a smooth bundle isomorphism by \cite[262]{lee:smooth_manifolds:2013}, $\Omega$ is a smooth bundle isomorphism.
\end{proof}

\begin{remark}
	In what follows, we will denote both the smooth bundle isomorphism $\Omega : TM \to T^*M$ as well as the induced $C^\infty(M)$-linear morphism $\Omega : \mathfrak{X}(M) \to \mathfrak{X}^*(M)$ by the same letter $\Omega$. However, as a subtle distinction between those two maps, we will write $\Omega_X$ for the evaluation of the latter at some $X \in \mathfrak{X}(M)$.
\end{remark}

\section*{Hamiltonian Systems}
If the Legendre transform \ref{def:Legendre_transform} is a diffeomorphism, we can define an associated Hamiltonian function by \ref{def:Hamiltonian_function}, that is a smooth function $H$ on $T^*M$, where $M$ is a smooth manifold. By example \ref{ex:cotangent_bundle}, we know that the cotangent bundle $T^*M$ admits a canonical symplectic structure in terms of the tautological form \ref{def:tautological_form}. The tuple $(T^*M,H)$ turns out to be the prototype of a much more general structure.

\begin{definition}[Hamiltonian System]
	A \bld{Hamiltonian system}\index{Hamiltonian!system} is defined to be a tuple $\del[1]{(M,\omega),H}$ consisting of a symplectic manifold $(M,\omega)$, called a \bld{phase space}\index{Space!phase}, and a function $H \in C^\infty(M)$, called a \bld{Hamiltonian function}\index{Hamiltonian!function}.
\end{definition}

\begin{remark}
	In what follows, we will write simply $(M,\omega,H)$ for a Hamiltonian system instead of the more cumbersome $\del[1]{(M,\omega),H}$. The latter was choosen in the definition to emphasize the similarity to the definition of a Lagrangian system \ref{def:Lagrangian_system}.
\end{remark}

\subsection*{Hamiltonian Vector Fields}
As in Riemannian geometry, a main advantage of the symplectic structure is to reinstate the definition of the gradient of a smooth function as a vector field instead of a covector field using the tangent-cotangent bundle isomorphism (for the Riemannian case see \cite[342--343]{lee:smooth_manifolds:2013}).

\begin{definition}[Hamiltonian Vector Field]
	Let $(M,\omega,H)$ be a Hamiltonian system and denote by $\Omega : \mathfrak{X}(M) \to \mathfrak{X}^*(M)$ the tangent-cotangent bundle isomorphism from proposition \textup{\ref{thm:tangent-cotangent_bundle_isomorphism}}. The vector field $X_H$ defined by
	\begin{equation}
		\label{eq:Hamiltonian_vector_field}
		X_H := \Omega^{-1}(dH)
	\end{equation}
	\noindent is called the \bld{Hamiltonian vector field associated to the Hamiltonian system}\index{Hamiltonian!vector field}.
\end{definition}

\begin{lemma}
	\label{lem:interior_multiplication_Hamiltonian_vector_field}
	Let $(M,\omega,H)$ be a Hamiltonian system. Then $i_{X_H}\omega = dH$.
\end{lemma}

\begin{proof}
	By definition of the Hamiltonian vector field (\ref{eq:Hamiltonian_vector_field}) we have that $\Omega_{X_H} = dH$. Thus for any $x \in M$ and $v \in T_xM$ we compute
	\begin{equation*}
		dH_x(v) = \del[1]{\Omega_{X_H}}_x(v) = \Omega(X_H\vert_x)(v) = \omega_x\del[1]{X_H\vert_x,v} = (i_{X_H})_x(v). 
	\end{equation*}
\end{proof}

\begin{definition}[Invariance]
	\label{def:invariance}
	Let $M$ be a smooth manifold, $X \in \mathfrak{X}(M)$ a complete vector field with global flow $\theta : \mathbb{R} \times M \to M$ and $l \in \mathbb{N}$. A tensor field $A \in \upGamma\del[1]{T^{(0,l)}TM}$ is said to be \bld{invariant under the flow $\theta$ of $X$}\index{Invariance!under flows of a vector field}, iff
	\begin{equation*}
		\theta_t^*A = A
	\end{equation*}
	\noindent for all $t \in \mathbb{R}$.
\end{definition}

A useful characterisation of invariance under flows can be given in terms of a special derivative. Recall, that in the setting of definition \ref{def:invariance}, the \emph{Lie derivative of $A$ with respect to $X$}, written $\mathcal{L}_XA$, is defined to be the tensor field $\mathcal{L}_XA \in \upGamma\del[1]{T^{(0,l)}TM}$ given by
\begin{equation*}
	(\mathcal{L}_XA)_x := \frac{d}{dt}\bigg\vert_{t = 0}(\theta^*_tA)_x
\end{equation*}
\noindent for all $x \in M$. By \cite[324]{lee:smooth_manifolds:2013}, we have that $A$ is invariant under the flow of $X$ if and only if $\mathcal{L}_XA = 0$. The next proposition is a prime example why we require a symplectic structure to be both closed and nondegenerate. For the proof, we need one more preliminary result from the calculus of differential forms.

\begin{proposition}[{Cartan's Magic Formula \cite[372]{lee:smooth_manifolds:2013}}]
	\label{prop:Cartans_magic_formula}
	Let $M$ be a smooth manifold, $X \in \mathfrak{X}(M)$ and $\omega \in \upOmega^l(M)$ for some $l \in \mathbb{N}$. Then
	\begin{equation*}
		\mathcal{L}_X\omega = i_X(d\omega) + d(i_X\omega).
	\end{equation*}
\end{proposition}

\begin{proposition}
	Let $(M,\omega,H)$ be a Hamiltonian system such that the Hamiltonian vector field is complete. Then the symplectic form is invariant under the flow of the Hamiltonian vector field.
\end{proposition}

\begin{proof}
	By the previous discussion it is enough to show that $\mathcal{L}_{X_H}\omega = 0$. Using Cartan's magic formula \ref{prop:Cartans_magic_formula}, closedness of $\omega$ together with lemma \ref{lem:interior_multiplication_Hamiltonian_vector_field} we compute
	\begin{equation*}
		\mathcal{L}_{X_H}\omega = i_{X_H}(d\omega) + d(i_{X_H}\omega) = d(i_{X_H}\omega) = (d \circ d)H = 0.
	\end{equation*}
\end{proof}

\subsection*{Poisson Brackets}

\begin{definition}[Poisson Bracket]
	Let $(M,\omega)$ be a symplectic manifold. Define a mapping 
	\begin{equation*}
		\cbr{\cdot,\cdot} : C^\infty(M) \times C^\infty(M) \to C^\infty(M)
	\end{equation*}
	\noindent by
	\begin{equation*}
		\cbr{f,g} := \omega(X_f,X_g)
	\end{equation*}
	\noindent where $X_f$ and $X_g$ are Hamiltonian vector fields associated to the Hamiltonian systems $(M,\omega,f)$ and $(M,\omega,g)$, respectively. The mapping $\cbr{\cdot,\cdot}$ is called the \bld{Poisson bracket on $C^\infty(M)$}\index{Poisson!bracket}.	
\end{definition}

Recall, that if $f \in C^\infty(M)$ for a smooth manifold $M$, the \emph{differential of $f$} is defined to be the covector field given by $df_x(v) := vf$ for $x \in M$ and $v \in T_xM$. This is indeed a smooth covector field by part (d) of the smoothness criteria for tensor fields \ref{prop:smoothness_criteria_for_tensor_fields} since
\begin{equation}
	\label{eq:differential_vector_field}
	df(X)(x) = df_x(X\vert_x) = X\vert_x f = (Xf)(x)
\end{equation}
\noindent for any $X \in \mathfrak{X}(M)$ and $x \in M$, and $Xf$ is smooth by \cite[180]{lee:smooth_manifolds:2013} (proving this is analogous to the proof of the smoothness criteria for tensor fields \ref{prop:smoothness_criteria_for_tensor_fields}).

\begin{lemma}
	\label{lem:Poisson_bracket_equivalent}
	Let $(M,\omega)$ be a symplectic manifold. Then $\cbr{f,g} = X_gf$ holds for all $f,g \in C^\infty(M)$.
\end{lemma}

\begin{proof}
	Using lemma \ref{lem:interior_multiplication_Hamiltonian_vector_field} and equation (\ref{eq:differential_vector_field}), we compute
	\begin{equation*}
		\cbr{f,g} = \omega(X_f,X_g) = \del[0]{i_{X_f}\omega}(X_g) = df(X_g) = X_g f.
	\end{equation*}
\end{proof}

\begin{definition}[Integral of Motion]
	Let $(M,\omega,H)$ be a Hamiltonian system. A function $f \in C^\infty(M)$ is said to be an \bld{integral of motion for the Hamiltonian system $(M,\omega,H)$}\index{Integral of motion}, iff $\cbr{H,f} = 0$.
\end{definition}

\subsection*{Noether's Theorem}
Let us recall some basic facts from the theory of Lie groups and Lie algebras. A \emph{Lie group} is defined to be a group $(G,\cdot)$, such that $G$ is a smooth manifold and the multiplication $\cdot$ as well as the inversion map $\cdot^{-1} : G \to G$ defined by $g\mapsto g^{-1}$ are smooth. If $G$ is a Lie group, we can associate to $G$ its \emph{Lie algebra} $\mathfrak{g}$ defined to be $\mathfrak{g} := T_eG$, where $e$ denotes the neutral element of $G$. It can be shown that $\mathfrak{g} \cong \mathfrak{X}_L(G)$ as real vector spaces, where $\mathfrak{X}_L(G) \subseteq \mathfrak{X}(G)$ denotes the space of \emph{left invariant vector fields on $G$}, that is, the vector fields $X \in \mathfrak{X}(G)$ satisfying $(L_g)_* X = X$, where $L_g$ is the diffeomorphism $L_g : G \to G$ defined by $L_g(h) := gh$ and $(L_g)_*$ is the \emph{pushforward of $X$} defined to be the vector field $\del[1]{(L_g)_*X}_h := d(L_g)_{g^{-1}h}X\vert_{g^{-1}h}$ for $h \in G$. Most importantly, any left invariant vector field on $G$ is complete and so we can define the \emph{exponential map} $\exp : \mathfrak{g} \to G$ by 
\begin{equation*}
	\exp v := \gamma(1),
\end{equation*}
\noindent where $\gamma \in C^\infty(\mathbb{R},G)$ is the integral curve of the \emph{left invariant vector field $X_v$ associated to $v$ on $G$}, that is $X_v\vert_g := d(L_g)_e(v)$, with starting point $\gamma(0) = e$. Then we have that $\gamma(t) = \exp tv$ and $(\exp tv)^{-1} = \exp (-tv)$ for all $v \in \mathfrak{g}$ and $t \in \mathbb{R}$.

The most important applications of Lie groups to smooth manifold theory involve actions by Lie groups on manifolds. Let $G$ be a Lie group and $M$ be a smooth manifold. A map in $C^\infty(G \times M,M)$ given by $(g,x) \mapsto g \cdot x$, is said to be a \emph{left action of $G$ on $M$} iff
\begin{equation*}
	g \cdot (h \cdot x) = (gh) \cdot x \qquad \text{and} \qquad e \cdot x = x
\end{equation*}
\noindent holds for all $g,h \in G$ and $x \in M$. Similarly, a \emph{right action of $G$ on $M$} is defined to be a map in $C^\infty(M \times G,M)$ given by $(x,g) \mapsto x \cdot g$ satisfying
\begin{equation*}
	(x \cdot g) \cdot h = x \cdot (gh) \qquad \text{and} \qquad x \cdot e = x
\end{equation*}
\noindent for all $g,h \in G$ and $x \in M$. Note that any left action of $G$ on $M$ can be transformed into a right action of $G$ on $M$ by defining $x \cdot g := g^{-1} \cdot x$ for all $g \in G$ and $x \in M$, and similarly every right action of $G$ on $M$ can  be transformed into a left action of $G$ on $M$. 

Suppose we are given a right action of a Lie group $G$ on a smooth manifold $M$. Then each element $v \in \mathfrak{g}$ determines a global flow on $M$ by
\begin{equation*}
	(t,x) \mapsto x \cdot \exp tv.
\end{equation*}
Define $\what{v} \in \mathfrak{X}(M)$ by
\begin{equation*}
	\what{v}_x := \frac{d}{dt}\bigg\vert_{t = 0} x \cdot \exp tv
\end{equation*}
\noindent for all $x \in M$. This is the \emph{infinitesimal generator} associated to the above flow (see \cite[210]{lee:smooth_manifolds:2013}). Hence we get a map $\mathfrak{g} \to \mathfrak{X}(M)$ defined by $v \mapsto \what{v}$. By \cite[526]{lee:smooth_manifolds:2013}, this map is actually a \emph{Lie algebra homomorphism}. This is the main reason we are working with right actions rather than left actions. 

\begin{lemma}[{Computing the Differential Using a Velocity Vector \cite[70]{lee:smooth_manifolds:2013}}]
	\label{lem:computing_the_differential_using_a_velocity_vector}
	Let $F \in C^\infty(M,N)$ for two smooth manifolds $M$ and $N$, $x \in M$ and $v \in T_xM$. Then
	\begin{equation*}
		dF_x(v) = (F \circ \gamma)'(0)
	\end{equation*}
	\noindent for any path $\gamma \in C^\infty(J,M)$, where $J \subseteq \mathbb{R}$ is an interval such that $0 \in J$, $\gamma(0) = x$ and $\gamma'(0) = v$.
\end{lemma}

\begin{proposition}
	\label{prop:v_hat_action}
	Suppose we are given a right action of a Lie group $G$ on a smooth manifold $M$. Then for each $v \in \mathfrak{g}$, the infinitesimal generator $\what{v}$ associated to the flow generated by $v$ satisfies
	\begin{equation*}
		(\what{v}f)(x) = \frac{d}{dt}\bigg\vert_{t = 0} f(x \cdot \exp tv)
	\end{equation*}
	\noindent for all $x \in M$ and $f \in C^\infty(M)$.
\end{proposition}

\begin{proof}
	Let $x \in M$ and denote by $\theta : M \times G \to M$ the right action of $G$ on $M$. Define $\theta^x : G \to M$ by $\theta^x(g) := x \cdot g$. Then $\theta^x$ is smooth since
	\begin{equation*}
		\begin{tikzcd}
			\theta^x\colon \quad G \cong \cbr{x} \times G \arrow[r,hook] & M \times G \arrow[r,"\theta"] & M
		\end{tikzcd}
	\end{equation*}
	\noindent where the first two maps steem from \cite[100]{lee:smooth_manifolds:2013}. Set $\gamma(t) := \exp tv$ for all $t \in \mathbb{R}$. Then it is immediate, that
	\begin{equation*}
		x \cdot \exp tv = \theta^x\del[1]{\gamma(t)}.
	\end{equation*}
	Thus we compute
	\begin{align*}
		(\what{v}f)(x) &= \what{v}_xf\\
		&= \frac{d}{dt}\bigg\vert_{t = 0}\theta^x\del[1]{\gamma(t)} f\\ 
		&= d(\theta^x)_e(v)f & (\text{by lemma } \ref{lem:computing_the_differential_using_a_velocity_vector})\\
		&= v(f \circ \theta^x) & (\text{by definition of } d\theta^x)\\
		&= d(f \circ \theta^x)_e(v) & (\text{by definition of } d(f \circ \theta^x))\\
		&= \del[0]{f \circ \theta^x \circ \gamma}'(0) & (\text{by lemma } \ref{lem:computing_the_differential_using_a_velocity_vector})\\
		&= \frac{d}{dt}\bigg\vert_{t = 0} f(x \cdot \exp tv). 
	\end{align*}
\end{proof}

\begin{remark}
	From now on, we will consider left actions of Lie groups $G$ on smooth manifolds $M$ only instead of right actions, since they are more common. This is however no drawback, since any left action can be converted into a right action. Hence if $v \in \mathfrak{g}$, the corresponding infinitesimal generator $V$ is given by
	\begin{equation*}
		\what{v}_x = \frac{d}{dt}\bigg\vert_{t = 0}\exp(-tv) \cdot x.
	\end{equation*}
\end{remark}

Let $\theta : G \times M \to M$ be a left action of a Lie group $G$ on a symplectic manifold $(M,\omega)$. We say that \bld{$G$ acts on $M$ by symplectomorphisms}\index{Action!by symplectomorphisms}, iff for all $g \in G$, the map $\theta_g : M \to M$ defined by $\theta_g(x) := g \cdot x$ is a symplectomorphism. We adapt the terminology provided in \cite[203]{salamon:symplectic_topology:2017}.

\begin{definition}[Weakly Hamiltonian Action and Hamiltonian Action]
	A left action of a Lie group $G$ on a symplectic manifold $(M,\omega)$ by symplectomorphisms is said to be a \bld{Hamiltonian action of $G$ on $(M,\omega)$}\index{Action!weakly Hamiltonian}, iff for each $v \in \mathfrak{g}$, there exists a Hamiltonian system $(M,\omega,H_v)$, such that $X_{H_v} = \what{v}$. If additionally the induced mapping $\mathfrak{g} \to C^\infty(M)$ defined by $v \mapsto H_v$
\end{definition}

\begin{definition}[Symmetry Group]
	A Lie group $G$ is said to be a \bld{symmetry group of a Hamiltonian system $(M,\omega,H)$}\index{Symmetry group}, iff there exists a weakly Hamiltonian action of $G$ on $(M,\omega)$, such that
	\begin{equation*}
		H(g \cdot x) = H(x)
	\end{equation*}
	\noindent holds for all $g \in G$ and $x \in M$.
\end{definition}

\begin{theorem}[Noether's Theorem, Hamiltonian Version\index{Noether!'s theorem, Hamiltonian version}]
	\label{thm:Noethers_theorem_Hamiltonian_case}
	Let $G$ be a symmetry group of a Hamiltonian system $(M,\omega,H)$. Then for each $v \in \mathfrak{g}$, the function $H_v \in C^\infty(M)$ such that $X_{H_v} = \what{v}$ is an integral of motion.
\end{theorem}

\begin{proof}
	Let $x \in M$. We compute
	\begin{align*}
		\cbr[0]{H,H_v}(x) &= \del[1]{X_{H_v}H}(x) & (\text{by lemma } \ref{lem:Poisson_bracket_equivalent})\\
		&= (\what{v}H)(x)\\
		&= \frac{d}{dt}\bigg\vert_{t = 0}H\del[1]{\exp(-tv) \cdot x} & (\text{by proposition } \ref{prop:v_hat_action})\\
		&= \frac{d}{dt}\bigg\vert_{t = 0}H(x)\\
		&= 0.
	\end{align*}
\end{proof}



%Appendix
\appendix

\printbibliography
\printindex
\end{document}
