\section*{Tangent Spaces and the Differential}
Let $M$ be a smooth manifold and let $x \in M$. Define a binary relation on the set
\begin{equation*}
	X := \cbr[1]{(U,f) : U \subseteq M \text{ neighbourhood of } x, f \in C^\infty(U)}
\end{equation*}
\noindent by
\begin{equation*}
	(U,f)\sim(V,g) \quad :\Leftrightarrow \quad \exists W \subseteq U \cap V \text{ neighbourhood of $x$, such that } f\vert_W = g\vert_W.
\end{equation*}

\begin{exercise}
	Show that the above relation is actually an equivalence relation.
\end{exercise}

\begin{definition}[Germ]
	Let $M$ be a smooth manifold and let $x \in M$. The set of \bld{germs at $p$}, written $C^\infty_x(M)$ is defined to be $C^\infty_x(M) := X/{\sim}$.
\end{definition}

\begin{exercise}
	Show that $C^\infty_x(M)$ is an $\mathbb{R}$-algebra under the obvious operations.
\end{exercise}

\begin{remark}
	Note that if $f \in C^\infty(M)$, then $\sbr[0]{(M,f)} \sim \sbr[0]{(U,f\vert_U)}$ for any neighbourhood $U$ of $x$. Thus any germ at $p$ contains a representant which is defined on the whole manifold and we thus may simply write $\sbr[0]{f}$ for a germ at $p$. 
\end{remark}

\begin{remark}
	Let $\sbr[0]{f}$ be a germ at $x \in M$. Then $f(x)$ is well-defined. Indeed, if $f\vert_U = g\vert_U$ on some neighbourhood of $x$, then in particular $f(x) = g(x)$.
\end{remark}

\begin{definition}[Tangent Space]
	Let $M$ be a smooth manifold and let $x \in M$. The \bld{tangent space of $M$ at $x$}, written $T_xM$, is defined to be the vector space $\del[1]{C^\infty_x(M)}^*$ such that
	\begin{equation*}
		v(\sbr[0]{f}\sbr[0]{g}) = v\sbr[0]{f}g(x) + f(x)v\sbr[0]{g}
	\end{equation*}
	\noindent holds.
\end{definition}

\begin{lemma}
	\label{lem:derivation_of_constant_germ}
	Let $M$ be a smooth manifold and $x \in M$. Suppose $\lambda \in C^\infty(M)$ is a constant function. Then $v\sbr{\lambda} = 0$ for all $v \in T_xM$.
\end{lemma}

\begin{proof}
	This immediately follows from
	\begin{equation*}
		v\sbr{\lambda} = v\sbr{\lambda \cdot 1} = \lambda v\sbr{1} = \lambda v\sbr{1 \cdot 1} = 2\lambda v\sbr{1} = 2 v\sbr{\lambda}.
	\end{equation*}
\end{proof}

\begin{definition}[Derivation]
	Let $M$ be a smooth manifold, $x \in M$ and $U$ a neighbourhood of $x$. The \bld{space of derivations of $C^\infty(U)$ at $x$}, written $\mathcal{D}_x(U)$, is defined to be the vector space $\del[1]{C^\infty(U)}^*$ such that
	\begin{equation*}
		v(fg) = v(f)g(x) + f(x)v(g)
	\end{equation*}
	\noindent holds.
\end{definition}

\begin{proposition}
	\label{prop:isomorphism_tangent_space}
	Let $M$ be a a smooth manifold, $x \in M$ and $U$ be a neighbourhood of $x$. Then
	\begin{equation*}
		T_xM \cong \mathcal{D}_x(U).
	\end{equation*}
\end{proposition}

\begin{proof}
	Let $\Phi : T_xM \to \mathcal{D}_x(U)$ be defined by
	\begin{equation*}
		\Phi(v)(f) := v\sbr[0]{f}
	\end{equation*}
	\noindent for all $f \in C^\infty(U)$. Clearly $\Phi$ is well-defined and linear. We want to construct an inverse $\Psi : \mathcal{D}_x(U) \to T_xM$ for $\Phi$. This implies, that we should define
	\begin{equation*}
		\Psi(v)\sbr[0]{f} = v\del[1]{\wtilde{f}}
	\end{equation*}
	\noindent where $\wtilde{f} \in C^\infty(U)$ such that $\sbr[1]{\wtilde{f}} = \sbr[0]{f}$. 
	\begin{enumerate}[label=\textit{Step \arabic*:},leftmargin=*,wide=0pt]
		\item \textit{Existence of $\wtilde{f}$.} Let $(V,f)$ be a representant of $\sbr[0]{f}$. As in the proof of the smoothness criteria for tensor fields \ref{prop:smoothness_criteria_for_tensor_fields}, we find a neighbourhood $W$ about $x$ such that $\wwbar{W} \subseteq U \cap V$. Then there exists a smooth bump function $\psi \in C^\infty(U \cap V)$ such that $\psi\vert_W = 1$ and $\supp \psi \subseteq U\cap V$. Let $\wtilde{f} := \psi f$ extended to be zero on $U$. Then clearly $\sbr[1]{\wtilde{f}} = \sbr[0]{f}$ since $\wtilde{f} = f$ on $W$.
		\item \textit{$\Psi$ is well-defined.} Suppose that $\sbr[0]{f} = \sbr[0]{g}$ in $C_x^\infty(M)$. Then $f = g$ on some neighbourhood $V$ of $x$. We claim that $v(f) = v(g)$ on $U \cap V$. Indeed, let $\psi$ be a smooth bump function for $\cbr{x}$ supported in $U \cap V$. Then $\psi(f - g) = 0$ on $U$ and we compute
			\begin{equation*}
				0 = v\del[1]{\psi(f - g)} = v(\psi)(f - g)(x) + \psi(x)v(f - g) = v(f - g).
			\end{equation*}
	\end{enumerate}
\end{proof}

\begin{lemma}
	\label{lem:derivation_of_constant_function}
	Let $M$ be a smooth manifold and $U$ a neighbourhood of $x \in M$. Suppose $\lambda \in C^\infty(U)$ is a constant function. Then $v(\lambda) = 0$ for all $v \in \mathcal{D}_x(U)$.
\end{lemma}

\begin{proof}
	Using the notation of the proof of proposition \ref{prop:isomorphism_tangent_space}, lemma \ref{lem:derivation_of_constant_germ} yields	
	\begin{equation*}
		v(\lambda) = (\Phi \circ \Psi)(v)(\lambda) = \Psi(v)\sbr{\lambda} = 0.
	\end{equation*}
\end{proof}

\begin{example}[Coordinate Derivation]
	Let $M^n$ be a smooth manifold and $(U,\varphi)$ be a chart on $M$. For every $x \in U$ and every $i = 1,\dots,n$ define 
	\begin{equation*}
		\frac{\partial}{\partial x^i}\bigg\vert_x : C^\infty(U) \to \mathbb{R}
	\end{equation*}
	\noindent by
	\begin{equation*}
		\frac{\partial}{\partial x^i}\bigg\vert_x(f) := D_i(f \circ \varphi^{-1})\del[1]{\varphi(x)}.
	\end{equation*}
	Then clearly $\frac{\partial}{\partial x^i}\big\vert_x$ is a derivation of $C^\infty(U)$ at $x$. Thus by proposition \ref{prop:isomorphism_tangent_space}, $\frac{\partial}{\partial x^i}\big\vert_x \in T_xM$. 
\end{example}

One of the profound features of tangent spaces to a smooth manifold are that they are finite dimensional. In fact, they admit the same dimension as the manifold.

\begin{lemma}
	\label{lem:star_shaped}
	Let $\upOmega \subseteq \mathbb{R}^n$ be open and star-shaped about $x_0 \in \upOmega$. Suppose $f \in C^\infty(\upOmega)$. Then there exists $\varphi_1,\dots,\varphi_n \in C^\infty(\upOmega)$ such that $\varphi_i(x_0) = D_if(x_0)$ and
	\begin{equation*}
		f(x) = f(x_0) + \pi^i(x - x_0) \varphi_i(x)
	\end{equation*}
	\noindent holds for all $x \in \upOmega$
\end{lemma}

\begin{proof}
	For $x \in \upOmega$ define $\gamma_x : \intcc{0,1} \to \upOmega$ by $\gamma_x(t) := x_0 + t(x - x_0)$ (note that this is only possible since $\upOmega$ is assumed to be star-shaped with centre $x_0$). Then
	\begin{align*}
		f(x) - f(x_0) &= \int_0^1 (f \circ \gamma_x)'(t) dt\\
		&= \int_0^1 D_if\del[1]{\gamma_x(t)} \dot{\gamma}^i_x(t) dt\\
		&= \int_0^1 D_if\del[1]{\gamma_x(t)} \pi^i(x - x_0)dt\\
		&= \pi^i(x - x_0)\varphi_i(x)
	\end{align*}
	\noindent where
	\begin{equation*}
		\varphi_i(x) := \int_0^1 D_if\del[1]{\gamma_x(t)} dt.
	\end{equation*}
\end{proof}

\begin{proposition}[Basis for the Tangent Space]
	\label{prop:basis_tangent_space}
	Let $M^n$ be a smooth manifold and $\del[0]{U,\varphi}$ a chart on $M$. Then 
	\begin{equation*}
		\cbr[3]{\frac{\partial}{\partial x^i}\bigg\vert_x : i = 1,\dots,n}
	\end{equation*}
	\noindent is a basis for $T_xM$ for all $x \in U$, where $x^i := \pi^i \circ \varphi$.
\end{proposition}

\begin{proof}
	Since $\varphi(U) \subseteq \mathbb{R}^n$ is open, there exists $\varepsilon > 0$ such that $B_\varepsilon\del[1]{\varphi(x)} \subseteq \varphi(U)$. Set $V := \varphi^{-1}\del[1]{B_\varepsilon\del[1]{\varphi(x)}}$. Then $V$ is a neighbourhood of $x$ in $M$ and thus by proposition \ref{prop:isomorphism_tangent_space}, we have that $T_xM \cong \mathcal{D}_x(V)$. Let $f \in C^\infty(V)$. An application of lemma \ref{lem:star_shaped} to $f \circ \varphi^{-1} \in C^\infty\del[1]{B_\varepsilon\del[1]{\varphi(x)}}$ yields
	\begin{align*}
		(f \circ \varphi^{-1})(y) &= f(x) + \pi^i\del[1]{y - \varphi(x)}\varphi_i(y)\\
		&= f(x) + \del[1]{\pi^i(y) - x^i(x)}\varphi_i(y)\\
		&= f(x) + \del[1]{(\pi^i \circ \varphi)\del[1]{\varphi^{-1}(y)} - x^i(x)}(\varphi_i \circ \varphi)\del[1]{\varphi^{-1}(y)}.\\
	\end{align*}
	Thus 
	\begin{equation*}
		f = f(x) + \del[1]{x^i - x^i(x)}(\varphi_i \circ \varphi) 
	\end{equation*}
	\noindent on $V$. Using lemma \ref{lem:derivation_of_constant_function} we compute
	\begin{equation}
		\label{eq:derivation_in_coordinates}
		v(f) = v\del[1]{\del[1]{x^i - x^i(x)}(\varphi_i \circ \varphi)} = v(x^i)\varphi_i\del[1]{\varphi(x)} = v(x^i)\frac{\partial}{\partial x^i}\bigg\vert_x(f).
	\end{equation}
	Suppose that $\lambda^i\frac{\partial}{\partial x^i}\big\vert_x = 0$. Then using example \ref{ex:linear_map_differential} and proposition \ref{prop:computing_differential_via_directional_derivative} we compute
	\begin{equation}
		\label{eq:linear_independence_tangent_basis}
		0 = \lambda^i\frac{\partial}{\partial x^i}\bigg\vert_x(x^j) = \lambda^i D_i\pi^j\del[1]{\varphi(x)} = \lambda^i \pi^j(e_i) = \lambda^i \delta^j_i = \lambda^j.
	\end{equation}
\end{proof}

\begin{definition}[Derivative]
	Let $M$ and $N$ be smooth manifolds and $F \in C^\infty(M,N)$. For $x \in M$, define a map $DF_x : T_xM \to T_{F(x)}N$ by
	\begin{equation*}
		DF_x(v)(f) := v(f \circ F)
	\end{equation*}
	\noindent for all $f \in C^\infty(N)$. This map is called the \bld{derivative of $F$ at $x$}.
\end{definition}

\begin{definition}[Velocity of a Curve]
	\label{def:velocity_of_a_curve}
	Let $J \subseteq \mathbb{R}$ be an open interval and $\gamma \in C^\infty(J,M)$ be a curve in a smooth manifold $M$. For every $t \in J$, define the \bld{velocity vector of $\gamma$ at $t$}, written $\gamma'(t)$, by
	\begin{equation*}
		\gamma'(t) := D\gamma_t\del[3]{\frac{d}{dt}\bigg\vert_t} \in T_{\gamma(t)}M.
	\end{equation*}
\end{definition}

It is immediate from the definition of the velocity vector of a curve \ref{def:velocity_of_a_curve}, that
\begin{equation*}
	\gamma'(t)(f) = D\gamma_t\del[3]{\frac{d}{dt}\bigg\vert_t}(f) = \frac{d}{dt}\bigg\vert_t(f \circ \gamma) = (f \circ \gamma)'(t)
\end{equation*}
\noindent for all $f \in C^\infty(M)$. Moreover, if $(U,\varphi)$ is a chart on $M$, then equation \ref{eq:derivation_in_coordinates} yields
\begin{equation}
	\label{eq:velocity_vector_in_coordinates}
	\gamma'(t) = \gamma'(t)(x^i)\frac{\partial}{\partial x^i}\bigg\vert_{\gamma(t)} = (x^i \circ \gamma)'(t)\frac{\partial}{\partial x^i}\bigg\vert_{\gamma(t)} = \dot{\gamma}^i(t)\frac{\partial}{\partial x^i}\bigg\vert_{\gamma(t)}
\end{equation}
\noindent at least sufficiently close to $t$. Velocity vectors to a curve give yet another way to think about the tangent space $T_xM$ to a point $x \in M$ of a smooth maifold $M$. Consider the set
\begin{equation*}
	X := \cbr{\gamma \in C^\infty(J,M) : J \subseteq \mathbb{R} \text{ open interval with } 0\in J, \gamma(0) = x}.
\end{equation*}
Define a binary relation on $X$ as follows:
\begin{equation*}
	\gamma_1 \sim \gamma_2 \quad :\Leftrightarrow \quad \exists \text{ chart $(U,\varphi)$ about $x$ such that } (\varphi \circ \gamma_1)'(0) = (\varphi \circ \gamma_2)'(0). 
\end{equation*}

\begin{exercise}
	Show that the above relation is an equivalence relation.
\end{exercise}

Let $\mathcal{V}_xM := X/{\sim}$.

\begin{proposition}
	Let $M$ be a smooth manifold and $x \in M$. Then $T_xM \cong \mathcal{V}_xM$ as sets.
\end{proposition}

\begin{proof}
	Define $\Phi : \mathcal{V}_xM \to T_xM$ by $\Phi\sbr[0]{\gamma} := \gamma'(0)$. This map is well-defined. Indeed, if $\sbr[0]{\gamma_1} = \sbr[0]{\gamma_2}$, there exists a chart $(U,\varphi)$ about $x$ such that $(\varphi \circ \gamma_1)'(0) = (\varphi \circ \gamma_2)'(0)$. This immediately implies that $\dot{\gamma}^i_1(0) = \dot{\gamma}^i_2(0)$ for all $i = 1,\dots,n$. Thus (\ref{eq:velocity_vector_in_coordinates}) yields $\gamma_1'(0) = \gamma_2'(0)$. From this also follows that $\Phi$ is injective. Indeed, if $\gamma_1'(0) = \gamma_2'(0)$, then $\dot{\gamma}_1^i(0) = \dot{\gamma}_2^i(0)$ for all $i = 1,\dots,n$ by (\ref{eq:velocity_vector_in_coordinates}) and proposition \ref{prop:basis_tangent_space}. Let $v \in T_xM$. Then in any chart $(U,\varphi)$ about $x$ we have that $v = v^i\frac{\partial}{\partial x^i}\big\vert_x$. Hence for $\varepsilon > 0$ sufficiently small we can define $\gamma_v : \intoo{-\varepsilon,\varepsilon} \to M^n$ by
	\begin{equation*}
		\gamma_v(t) := \varphi^{-1}\del[1]{tv^i,\dots,tv^n}.
	\end{equation*}
	Thus $\Phi$ is surjective.
\end{proof}

We can equip $\mathcal{V}_xM$ with the structure of a vector space by means of the following lemma.

\begin{lemma}
	Let $V$ be a a finite-dimensional real vector space and $S$ be a set. If there exists a bijection $\varphi : S \to V$, we can equip $V$ with a structure of a real vector space such that $\varphi$ is an isomorphism.
\end{lemma}

\begin{proof}
	Just define 
	\begin{equation*}
		\lambda x + y := \varphi^{-1}\del[1]{\lambda \varphi(x) + \varphi(y)}
	\end{equation*}
	\noindent for all $x,y \in S$ and $\lambda \in \mathbb{R}$.
\end{proof}

\begin{definition}[Cotangent Space]
	Let $M$ be a smooth manifold. For $x \in M$, define the \bld{cotangent space of $M$ at $x$}, written $T^*_xM$, to be 
	\begin{equation*}
		T^*_xM := (T_xM)^*.
	\end{equation*}
\end{definition}

\begin{definition}[Differential]
	Let $M$ be a smooth manifold, $U$ a neighbourhood of $x \in M$ and $f \in C^\infty(U)$. Define the \bld{differential of $f$ at $x$}, written $df_x$, to be the element $df_x \in T^*_xM$ given by
	\begin{equation*}
		df_x(v) := v(f).
	\end{equation*}
\end{definition}

\begin{lemma}[Basis for the Cotangent Space]
	\label{lem:basis_cotangent_space}
	Let $M^n$ be a smooth manifold and $\del[0]{U,\varphi}$ a chart on $M$. Then 
	\begin{equation*}
		\cbr[1]{dx^i\vert_x : i = 1,\dots,n}
	\end{equation*}
	\noindent is a basis for $T^*_xM$ for all $x \in U$, where $x^i := \pi^i \circ \varphi$.
\end{lemma}

\begin{proof}
	We only need to note that this is the dual basis of the tangent space basis \ref{prop:basis_tangent_space}. This follows from (\ref{eq:linear_independence_tangent_basis}) since
	\begin{equation*}
		dx^i\vert_x\del[3]{\frac{\partial}{\partial x^j}\bigg\vert_x} = \frac{\partial}{\partial x^j}\bigg\vert_x(x^i) = \delta^i_j.
	\end{equation*}
\end{proof}
