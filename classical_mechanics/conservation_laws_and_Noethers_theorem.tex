\section*{Conservation Laws and Noether's Theorem}

\begin{definition}[Conservation Law]
	A \bld{conservation law for a Lagrangian system $(M,L)$}\index{Law!Conservation} is defined to be a function $I \in C^\infty(TM)$ such that
	\begin{equation*}
		\frac{d}{dt}I\del[1]{\gamma(t),\dot{\gamma}(t)} = 0
	\end{equation*}
	\noindent for all extremals of the action functional \textup{(\ref{def:action_functional})}.
\end{definition}

\begin{proposition}[Conservation of Energy]
	The energy of an autonomous Lagrangian system is a conservation law.	
\end{proposition}

\begin{proof}
	By definition of the fibrewise differential \ref{def:fibrewise_differential} we have that
	\begin{equation*}
		D^\mathcal{F}L_{(\gamma,\dot{\gamma})}\del[0]{\dot{\gamma}} = \frac{\partial L}{\partial v^i}\del[1]{\gamma,\dot{\gamma}}dx^i\del[3]{\dot{\gamma}^j\frac{\partial}{\partial x^j}}
		= \frac{\partial L}{\partial v^i}\del[0]{\gamma,\dot{\gamma}} \dot{\gamma}^j\delta^i_j
		= \frac{\partial L}{\partial v^i}\del[0]{\gamma,\dot{\gamma}} \dot{\gamma}^i.
	\end{equation*}
	Thus by definition of the energy \ref{def:energy} and the Euler-Lagrange equations \ref{thm:EL_equations} we compute
	\begin{align*}
		\frac{d}{dt}E\del[0]{\gamma,\dot{\gamma}} &= \frac{d}{dt}\del[3]{\frac{\partial L}{\partial v^i}\del[0]{\gamma,\dot{\gamma}} \dot{\gamma}^i} - \frac{d}{dt}L\del[0]{\gamma,\dot{\gamma}}\\
		&= \frac{d}{dt}\frac{\partial L}{\partial v^i}\del[0]{\gamma,\dot{\gamma}} \dot{\gamma}^i + \frac{\partial L}{\partial v^i}\del[0]{\gamma,\dot{\gamma}} \ddot{\gamma}^i - \frac{\partial L}{\partial x^i}(\gamma,\dot{\gamma})\dot{\gamma}^i - \frac{\partial L}{\partial v^i}(\gamma,\dot{\gamma})\ddot{\gamma}^i\\
		&= \frac{d}{dt}\frac{\partial L}{\partial v^i}\del[0]{\gamma,\dot{\gamma}} \dot{\gamma}^i - \frac{\partial L}{\partial x^i}(\gamma,\dot{\gamma})\dot{\gamma}^i\\
		&= \frac{\partial L}{\partial x^i}\del[0]{\gamma,\dot{\gamma}} \dot{\gamma}^i - \frac{\partial L}{\partial x^i}(\gamma,\dot{\gamma})\dot{\gamma}^i\\
		&= 0. 
	\end{align*}
\end{proof}

Recall, that for a smooth manifold $M$, we define the \emph{set of diffeomorphisms on $M$} by
\begin{equation*}
	\Diff(M) := \cbr[0]{\varphi \in C^{\infty}(M,M) : \varphi \text{ is a diffeomorphism}}.
\end{equation*}
In fact $\Diff(M)$ constitutes a group under ordinary composition of maps. Thus we define a \emph{one-parameter group of diffeomorphisms of $M$} to be a group homomorphisms 
\begin{equation*}
	(\mathbb{R},+) \to \Diff(M)
\end{equation*}
Explicitely, given any one-parameter group $\theta : (\mathbb{R},+) \to \Diff(M)$, we define $\theta_s := \theta(s)$ for all $s \in \mathbb{R}$ and we can therefore write $(\theta_s)_{s \in \mathbb{R}}$ for the one-parameter group $\theta$ of diffeomorphisms of $M$. Since $\theta$ is a homomorphism of groups, we have that
\begin{equation*}
	\theta_{s + t} = \theta_s \circ \theta_t \qquad \text{and} \qquad \theta_0 = \id_M
\end{equation*}
\noindent for all $s,t \in \mathbb{R}$. We say that the one-parameter group $(\theta_s)_{s \in \mathbb{R}}$ of diffeomorphisms of $M$ is smooth, iff the corresponding map $\theta : \mathbb{R} \times M \to M$ defined by $(s,x) \mapsto \theta_s(x)$ is smooth. If $F \in C^\infty(M,N)$ for two smooth manifolds $M$ and $N$, for $x \in M$ we define the \emph{differential of $F$ at $x$} to be the mapping $dF_x : T_xM \to T_{F(x)}N$, given by $dF_x(v)(f) := v(f \circ F)$ for all $f \in C^\infty(N)$. These fibrewise mappings can be assembled to the \emph{global differential of $F$}, defined to be the mapping $dF : TM \to TN$ given by $dF(x,v) := \del[1]{F(x),dF_x(v)}$. The global differential is a smooth map (see \cite[68]{lee:smooth_manifolds:2013}) and has the following properties. 

\begin{proposition}[{Properties of the Global Differential \cite[68]{lee:smooth_manifolds:2013}}]
	\label{prop:properties_of_the_global_differential}
	Let $M,N,P$ be smooth manifolds, $F \in C^\infty(M,N)$ and $G \in C^\infty(N,P)$. Then:
	\begin{enumerate}[label = \textup{(\alph*)},leftmargin=*]
		\item $d(G \circ F) = dG \circ dF$.
		\item $d(\id_M) = \id_{TM}$.
		\item If $F$ is a diffeomorphism, then $dF$ is a diffeomorphism with $(dF)^{-1} = d\del[1]{F^{-1}}$.
	\end{enumerate} 
\end{proposition}

\begin{lemma}
	\label{lem:induced_one-parameter_group_of_diffeomorphisms}
	Let $(\theta_s)_{s \in \mathbb{R}}$ be a smooth one-parameter group of diffeomorphisms of a smooth manifold $M$. Then $(d\theta_s)_{s \in \mathbb{R}}$ is a smooth one-prameter group of diffeomorphisms of $TM$.
\end{lemma}

\begin{proof}
	Part (c) of the properties of the global differential \ref{prop:properties_of_the_global_differential} implies that $d\theta_s$ is a diffeomorphism for all $s \in \mathbb{R}$. Moreover, by part (c) of the properties of the global differential \ref{prop:properties_of_the_global_differential} we compute
	\begin{equation*}
		d\theta_{s + t} = d(\theta_s \circ \theta_t) = d\theta_s \circ d\theta_t
	\end{equation*}
	\noindent for all $s,t \in \mathbb{R}$. Lastly, part (b) of the properties of the global differential \ref{prop:properties_of_the_global_differential} implies
	\begin{equation*}
		d\theta_0 = d(\id_M) = \id_{TM}.
	\end{equation*}
\end{proof}

Given a one-parameter group $(\theta_s)_{s \in \mathbb{R}}$ of diffeomorphisms of a smooth manifold $M$, we can define a vector field $V$ by
\begin{equation*}
	V_x := \frac{d}{ds}\bigg\vert_{s = 0} \theta_s(x)
\end{equation*}
\noindent for all $x \in M$. This vector field is actually smooth by \cite[210]{lee:smooth_manifolds:2013} and is called the \emph{infinitesimal generator of $\theta$}.

\begin{definition}[Symmetry]
	A \bld{symmetry of an autonomous Lagrangian system $(M,L)$}\index{Symmetry} is defined to be a diffeomorphism $F \in \Diff(M)$, such that
	\begin{equation*}
		L \circ dF = L.
	\end{equation*}
\end{definition}

Recall, that if $k \in \mathbb{N}$ and $X \in \mathfrak{X}(M)$ for a smooth manifold $M$, we can define a mapping $i_X : \upOmega^{k + 1}(M) \to \upOmega^k(M)$, called \emph{interior multiplication}, by
\begin{equation*}
	(i_X\omega)_x(v_1,\dots,v_k) := \omega_x\del[1]{X\vert_x,v_1,\dots,v_k}
\end{equation*}
\noindent for all $x \in M$ and $v_1,\dots,v_k \in T_xM$. One-parameter groups of symmetries of autonomous Lagrangian systems give rise to conservation laws. 

\begin{theorem}[{Noether's Theorem, Lagrangian Version\index{Noether!'s theorem, Lagrangian version}}]
	\label{thm:Noethers_theorem_Lagrangian_version}	
	Let $(\theta_s)_{s \in \mathbb{R}}$ be a smooth one-parameter group of symmetries of an autonomous Lagrangian system. Then $i_V\del[1]{D^\mathcal{F}L}$ is a conservation law, where $V$ denotes the infinitesimal generator of the one-parameter group $(d\theta_s)_{s \in \mathbb{R}}$ of diffeomorphisms of $TM$. The conservation law $i_V\del[1]{D^\mathcal{F}L}$ is called the \bld{Noether integral}\index{Noether!integral}.
\end{theorem}

\begin{proof}
	Let $\del[1]{TU,(x^i,v^i)}$ be a chart on $TM$. First we compute the infinitesimal generator $V$ of the one-parameter group $(\theta_s)_{s \in \mathbb{R}}$ in the chart $\del[1]{U,(x^i)}$. Let $x \in U$. Then
	\begin{equation*}
		V_x = \frac{d}{ds}\bigg\vert_{s = 0}\theta_s(x) = \frac{d\theta_s^i(x)}{ds}(0) \frac{\partial}{\partial x^i}\bigg\vert_{\theta_0(x)} = \frac{d\theta_s^i(x)}{ds}(0) \frac{\partial}{\partial x^i}\bigg\vert_x.
	\end{equation*}
	Thus 
	\begin{equation*}
		V_x = V^i(x) \frac{\partial}{\partial x^i}\bigg\vert_x 
	\end{equation*}
	\noindent where $V^i : U \to \mathbb{R}$ are given by
	\begin{equation*}
		V^i(x) := \frac{d \theta_s^i(x)}{ds}(0).
	\end{equation*}
	Next consider the infinitesimal generator $V$ of the one-parameter group $(d\theta_s)_{s \in \mathbb{R}}$. For $(x,v) \in TU$, where $v = v^i \frac{\partial}{\partial x^i}$, we compute
	\begin{align*}
		V_{(x,v)} &= \frac{d}{ds}\bigg\vert_{s = 0} \del[1]{\theta_s(x),d(\theta_s)_x(v)}\\
		&= \frac{d}{ds}\bigg\vert_{s = 0} \del[3]{\theta_s(x),v^j\frac{\partial \theta_s^i}{\partial x^j}(x)\frac{\partial}{\partial x^i}\bigg\vert_{\theta_s(x)}}\\
		&= \frac{d \theta_s^i(x)}{ds}(0)\frac{\partial}{\partial x^i}\bigg\vert_{(x,v)} + v^j\frac{\partial^2 \theta^i}{\partial s\partial x^j}(0,x) \frac{\partial}{\partial v^i}\bigg\vert_{(x,v)}\\
		&= V^i(x)\frac{\partial}{\partial x^i}\bigg\vert_{(x,v)} + v^j\frac{\partial^2 \theta^i}{\partial x^j\partial s}(0,x) \frac{\partial}{\partial v^i}\bigg\vert_{(x,v)}\\
		&= V^i(x)\frac{\partial}{\partial x^i}\bigg\vert_{(x,v)} + v^j\frac{\partial}{\partial x^j} \frac{d \theta_s^i(x)}{ds}(0) \frac{\partial}{\partial v^i}\bigg\vert_{(x,v)}\\
		&= V^i(x)\frac{\partial}{\partial x^i}\bigg\vert_{(x,v)} + v^j\frac{\partial V^i}{\partial x^j}(x) \frac{\partial}{\partial v^i}\bigg\vert_{(x,v)}.
	\end{align*}
	Therefore
	\begin{align*}
		i_V\del[1]{D^\mathcal{F}L}(x,v) &= D^\mathcal{F}L_{(x,v)}\del[1]{V_{(x,v)}}\\
		&= \frac{\partial L}{\partial v^i}(x,v)dx^i\vert_{(x,v)}\del[1]{V_{(x,v)}}\\
		&= \frac{\partial L}{\partial v^i}(x,v) V^i(x).
	\end{align*}
	For $(x,v) \in TM$ set $\gamma(s) := d\theta_s(x,v)$. If $f \in C^\infty(TM)$, the definition of the velocity of a curve and of the differential yields
	\begin{equation*}
		(Vf)(x,v) = V_{(x,v)}f = \del[3]{\frac{d}{ds}\bigg\vert_{s = 0}\gamma(s)}f= d\gamma\del[3]{\frac{d}{ds}\bigg\vert_{s = 0}}f = \frac{d}{ds}\bigg\vert_{s = 0}(f \circ \gamma).
	\end{equation*}
	So using the Euler-Lagrange equations \ref{thm:EL_equations} and the assumptions that $\theta_s$ is a symmetry of $(M,L)$ for all $s \in \mathbb{R}$, we get
	\begin{align*}
		\frac{d}{dt}i_V\del[1]{D^\mathcal{F}L}(\gamma,\dot{\gamma}) &= \frac{d}{dt}\del[3]{\frac{\partial L}{\partial v^i}(\gamma,\dot{\gamma}) V^i(\gamma)}\\
		&= \frac{d}{dt}\frac{\partial L}{\partial v^i}(\gamma,\dot{\gamma}) V^i(\gamma) + \frac{\partial L}{\partial v^i}(\gamma,\dot{\gamma}) \frac{d}{dt}V^i(\gamma)\\
		&= \frac{\partial L}{\partial x^i}(\gamma,\dot{\gamma}) V^i(\gamma) + \frac{\partial L}{\partial v^i}(\gamma,\dot{\gamma}) \frac{d}{dt}V^i(\gamma)\\
		&= \frac{\partial L}{\partial x^i}(\gamma,\dot{\gamma}) V^i(\gamma) + \frac{\partial L}{\partial v^i}(\gamma,\dot{\gamma}) \frac{\partial V^i}{\partial x^j}(\gamma)\dot{\gamma}^j\\
		&= V_{(\gamma,\dot{\gamma})}L\\
		&= \frac{d}{ds}\bigg\vert_{s = 0} \del[1]{L \circ d\theta_s}(\gamma,\dot{\gamma})\\
		&= \frac{d}{ds}\bigg\vert_{s = 0} L(\gamma,\dot{\gamma})\\
		&= 0.
	\end{align*}
\end{proof}
