\section*{Hamiltonian Systems}
If the Legendre transform \ref{def:Legendre_transform} is a diffeomorphism, we can define an associated Hamiltonian function by \ref{def:Hamiltonian_function}, that is a smooth function $H$ on $T^*M$, where $M$ is a smooth manifold. By example \ref{ex:cotangent_bundle}, we know that the cotangent bundle $T^*M$ admits a canonical symplectic structure in terms of the tautological form \ref{def:tautological_form}. The tuple $(T^*M,H)$ turns out to be the prototype of a much more general structure.

\begin{definition}[Hamiltonian System]
	A \bld{Hamiltonian system}\index{Hamiltonian!system} is defined to be a tuple $\del[1]{(M,\omega),H}$ consisting of a symplectic manifold $(M,\omega)$, called a \bld{phase space}\index{Space!phase}, and a function $H \in C^\infty(M)$, called a \bld{Hamiltonian function}\index{Hamiltonian!function}.
\end{definition}

\begin{remark}
	In what follows, we will write simply $(M,\omega,H)$ for a Hamiltonian system instead of the more cumbersome $\del[1]{(M,\omega),H}$. The latter was choosen in the definition to emphasize the similarity to the definition of a Lagrangian system \ref{def:Lagrangian_system}.
\end{remark}

As in Riemannian geometry, a main advantage of the symplectic structure is to reinstate the definition of the gradient of a smooth function as a vector field instead of a covector field using the tangent-cotangent bundle isomorphism (for the Riemannian case see \cite[342--343]{lee:smooth_manifolds:2013}).

\begin{definition}[Hamiltonian Vector Field]
	Let $(M,\omega,H)$ be a Hamiltonian system and denote by $\Omega : \mathfrak{X}(M) \to \mathfrak{X}^*(M)$ the tangent-cotangent bundle isomorphism from proposition \textup{\ref{prop:tangent-cotangent_bundle_isomorphism}}. The vector field $X_H$ defined by
	\begin{equation}
		\label{eq:Hamiltonian_vector_field}
		X_H := \Omega^{-1}(dH)
	\end{equation}
	\noindent is called the \bld{Hamiltonian vector field associated to the Hamiltonian system}\index{Hamiltonian!vector field}.
\end{definition}

Recall, that if $k \in \mathbb{N}$ and $X \in \mathfrak{X}(M)$ for a smooth manifold $M$, we can define a mapping $i_X : \upOmega^{k + 1}(M) \to \upOmega^k(M)$, called \emph{interior multiplication}, by
\begin{equation*}
	(i_X\omega)_x(v_1,\dots,v_k) := \omega_x\del[1]{X\vert_x,v_1,\dots,v_k}
\end{equation*}
\noindent for all $x \in M$ and $v_1,\dots,v_k \in T_xM$.

\begin{lemma}
	\label{lem:interior_multiplication_Hamiltonian_vector_field}
	Let $(M,\omega,H)$ be a Hamiltonian system. Then $i_{X_H}\omega = dH$.
\end{lemma}

\begin{proof}
	By definition of the Hamiltonian vector field (\ref{eq:Hamiltonian_vector_field}) we have that $\Omega_{X_H} = dH$. Thus for any $x \in M$ and $v \in T_xM$ we compute
	\begin{equation*}
		dH_x(v) = \del[1]{\Omega_{X_H}}_x(v) = \Omega(X_H\vert_x)(v) = \omega_x\del[1]{X_H\vert_x,v} = (i_{X_H})_x(v). 
	\end{equation*}
\end{proof}

\begin{definition}[Invariance]
	\label{def:invariance}
	Let $M$ be a smooth manifold, $X \in \mathfrak{X}(M)$ a complete vector field with global flow $\theta : \mathbb{R} \times M \to M$ and $l \in \mathbb{N}$. A tensor field $A \in \upGamma\del[1]{T^{(0,l)}TM}$ is said to be \bld{invariant under the flow $\theta$ of $X$}\index{Invariance!under flows of a vector field}, iff
	\begin{equation*}
		\theta_t^*A = A
	\end{equation*}
	\noindent for all $t \in \mathbb{R}$.
\end{definition}

A useful characterisation of invariance under flows can be given in terms of a special derivative. Recall, that in the setting of definition \ref{def:invariance}, the \emph{Lie derivative of $A$ with respect to $X$}, written $\mathcal{L}_XA$, is defined to be the tensor field $\mathcal{L}_XA \in \upGamma\del[1]{T^{(0,l)}TM}$ given by
\begin{equation*}
	(\mathcal{L}_XA)_x := \frac{d}{dt}\bigg\vert_{t = 0}(\theta^*_tA)_x
\end{equation*}
\noindent for all $x \in M$. By \cite[324]{lee:smooth_manifolds:2013}, we have that $A$ is invariant under the flow of $X$ if and only if $\mathcal{L}_XA = 0$. The next proposition is a prime example why we require a symplectic structure to be both closed and nondegenerate. For the proof, we need one more preliminary result from the calculus of differential forms.

\begin{proposition}[{Cartan's Magic Formula \cite[372]{lee:smooth_manifolds:2013}}]
	\label{prop:Cartans_magic_formula}
	Let $M$ be a smooth manifold, $X \in \mathfrak{X}(M)$ and $\omega \in \upOmega^l(M)$ for some $l \in \mathbb{N}$. Then
	\begin{equation*}
		\mathcal{L}_X\omega = i_X(d\omega) + d(i_X\omega).
	\end{equation*}
\end{proposition}

\begin{proposition}
	Let $(M,\omega,H)$ be a Hamiltonian system such that the Hamiltonian vector field is complete. Then the symplectic form is invariant under the flow of the Hamiltonian vector field.
\end{proposition}

\begin{proof}
	By the previous discussion it is enough to show that $\mathcal{L}_{X_H}\omega = 0$. Using Cartan's magic formula \ref{prop:Cartans_magic_formula}, closedness of $\omega$ together with lemma \ref{lem:interior_multiplication_Hamiltonian_vector_field} we compute
	\begin{equation*}
		\mathcal{L}_{X_H}\omega = i_{X_H}(d\omega) + d(i_{X_H}\omega) = d(i_{X_H}\omega) = (d \circ d)H = 0.
	\end{equation*}
\end{proof}


