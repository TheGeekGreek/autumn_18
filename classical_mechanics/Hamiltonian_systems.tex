\section*{Hamiltonian Systems}
If the Legendre transform \ref{def:Legendre_transform} is a diffeomorphism, we can define an associated Hamiltonian function by \ref{def:Hamiltonian_function}, that is a smooth function $H$ on $T^*M$, where $M$ is a smooth manifold. By example \ref{ex:cotangent_bundle}, we know that the cotangent bundle $T^*M$ admits a canonical symplectic structure in terms of the tautological form \ref{def:tautological_form}. The tuple $(T^*M,H)$ turns out to be the prototype of a much more general structure.

\begin{definition}[Hamiltonian System]
	A \bld{Hamiltonian system}\index{Hamiltonian!system} is defined to be a tuple $\del[1]{(M,\omega),H}$ consisting of a symplectic manifold $(M,\omega)$, called a \bld{phase space}\index{Space!phase}, and a function $H \in C^\infty(M)$, called a \bld{Hamiltonian function}\index{Hamiltonian!function}.
\end{definition}

\begin{remark}
	In what follows, we will write simply $(M,\omega,H)$ for a Hamiltonian system instead of the more cumbersome $\del[1]{(M,\omega),H}$. The latter was choosen in the definition to emphasize the similarity to the definition of a Lagrangian system \ref{def:Lagrangian_system}.
\end{remark}

As in Riemannian geometry, a main advantage of the symplectic structure is to reinstate the definition of the gradient of a smooth function as a vector field instead of a covector field using the tangent-cotangent bundle isomorphism (for the Riemannian case see \cite[342--343]{lee:smooth_manifolds:2013}).

\begin{definition}[Hamiltonian Vector Field]
	Let $(M,\omega,H)$ be a Hamiltonian system and denote by $\Omega : \mathfrak{X}(M) \to \mathfrak{X}^*(M)$ the tangent-cotangent bundle isomorphism from proposition \textup{\ref{thm:tangent-cotangent_bundle_isomorphism}}. The vector field $X_H$ defined by
	\begin{equation}
		\label{eq:Hamiltonian_vector_field}
		X_H := \Omega^{-1}(dH)
	\end{equation}
	\noindent is called the \bld{Hamiltonian vector field associated to the Hamiltonian system}\index{Hamiltonian!vector field}.
\end{definition}

Recall, that if $k \in \mathbb{N}$ and $X \in \mathfrak{X}(M)$ for a smooth manifold $M$, we can define a mapping $i_X : \upOmega^{k + 1}(M) \to \upOmega^k(M)$, called \emph{interior multiplication}, by
\begin{equation*}
	(i_X\omega)_x(v_1,\dots,v_k) := \omega_x\del[1]{X\vert_x,v_1,\dots,v_k}
\end{equation*}
\noindent for all $x \in M$ and $v_1,\dots,v_k \in T_xM$.

\begin{lemma}
	\label{lem:interior_multiplication_Hamiltonian_vector_field}
	Let $(M,\omega,H)$ be a Hamiltonian system. Then $i_{X_H}\omega = dH$.
\end{lemma}

\begin{proof}
	By definition of the Hamiltonian vector field (\ref{eq:Hamiltonian_vector_field}) we have that $\Omega_{X_H} = dH$. Thus for any $x \in M$ and $v \in T_xM$ we compute
	\begin{equation*}
		dH_x(v) = \del[1]{\Omega_{X_H}}_x(v) = \Omega(X_H\vert_x)(v) = \omega_x\del[1]{X_H\vert_x,v} = (i_{X_H})_x(v). 
	\end{equation*}
\end{proof}

\begin{definition}[Invariance]
	\label{def:invariance}
	Let $M$ be a smooth manifold, $X \in \mathfrak{X}(M)$ a complete vector field with global flow $\theta : \mathbb{R} \times M \to M$ and $l \in \mathbb{N}$. A tensor field $A \in \upGamma\del[1]{T^{(0,l)}TM}$ is said to be \bld{invariant under the flow $\theta$ of $X$}\index{Invariance!under flows of a vector field}, iff
	\begin{equation*}
		\theta_t^*A = A
	\end{equation*}
	\noindent for all $t \in \mathbb{R}$.
\end{definition}

A useful characterisation of invariance under flows can be given in terms of a special derivative. Recall, that in the setting of definition \ref{def:invariance}, the \emph{Lie derivative of $A$ with respect to $X$}, written $\mathcal{L}_XA$, is defined to be the tensor field $\mathcal{L}_XA \in \upGamma\del[1]{T^{(0,l)}TM}$ given by
\begin{equation*}
	(\mathcal{L}_XA)_x := \frac{d}{dt}\bigg\vert_{t = 0}(\theta^*_tA)_x
\end{equation*}
\noindent for all $x \in M$. By \cite[324]{lee:smooth_manifolds:2013}, we have that $A$ is invariant under the flow of $X$ if and only if $\mathcal{L}_XA = 0$. The next proposition is a prime example why we require a symplectic structure to be both closed and nondegenerate. For the proof, we need one more preliminary result from the calculus of differential forms.

\begin{proposition}[{Cartan's Magic Formula \cite[372]{lee:smooth_manifolds:2013}}]
	\label{prop:Cartans_magic_formula}
	Let $M$ be a smooth manifold, $X \in \mathfrak{X}(M)$ and $\omega \in \upOmega^l(M)$ for some $l \in \mathbb{N}$. Then
	\begin{equation*}
		\mathcal{L}_X\omega = i_X(d\omega) + d(i_X\omega).
	\end{equation*}
\end{proposition}

\begin{proposition}
	Let $(M,\omega,H)$ be a Hamiltonian system such that the Hamiltonian vector field is complete. Then the symplectic form is invariant under the flow of the Hamiltonian vector field.
\end{proposition}

\begin{proof}
	By the previous discussion it is enough to show that $\mathcal{L}_{X_H}\omega = 0$. Using Cartan's magic formula \ref{prop:Cartans_magic_formula}, closedness of $\omega$ together with lemma \ref{lem:interior_multiplication_Hamiltonian_vector_field} we compute
	\begin{equation*}
		\mathcal{L}_{X_H}\omega = i_{X_H}(d\omega) + d(i_{X_H}\omega) = d(i_{X_H}\omega) = (d \circ d)H = 0.
	\end{equation*}
\end{proof}

\begin{definition}[Poisson Bracket]
	Let $(M,\omega)$ be a symplectic manifold. Define a mapping 
	\begin{equation*}
		\cbr{\cdot,\cdot} : C^\infty(M) \times C^\infty(M) \to C^\infty(M)
	\end{equation*}
	\noindent by
	\begin{equation*}
		\cbr{f,g} := \omega(X_f,X_g)
	\end{equation*}
	\noindent where $X_f$ and $X_g$ are Hamiltonian vector fields associated to the Hamiltonian systems $(M,\omega,f)$ and $(M,\omega,g)$, respectively. The mapping $\cbr{\cdot,\cdot}$ is called the \bld{Poisson bracket on $C^\infty(M)$}\index{Poisson!bracket}.	
\end{definition}

Recall, that if $f \in C^\infty(M)$ for a smooth manifold $M$, the \emph{differential of $f$} is defined to be the covector field given by $df_x(v) := vf$ for $x \in M$ and $v \in T_xM$. This is indeed a smooth covector field by part (d) of the smoothness criteria for tensor fields \ref{prop:smoothness_criteria_for_tensor_fields} since
\begin{equation}
	\label{eq:differential_vector_field}
	df(X)(x) = df_x(X\vert_x) = X\vert_x f = (Xf)(x)
\end{equation}
\noindent for any $X \in \mathfrak{X}(M)$ and $x \in M$, and $Xf$ is smooth by \cite[180]{lee:smooth_manifolds:2013} (proving this is analogous to the proof of the smoothness criteria for tensor fields \ref{prop:smoothness_criteria_for_tensor_fields}).

\begin{lemma}
	\label{lem:Poisson_bracket_equivalent}
	Let $(M,\omega)$ be a symplectic manifold. Then $\cbr{f,g} = X_gf$ holds for all $f,g \in C^\infty(M)$.
\end{lemma}

\begin{proof}
	Using lemma \ref{lem:interior_multiplication_Hamiltonian_vector_field} and equation (\ref{eq:differential_vector_field}), we compute
	\begin{equation*}
		\cbr{f,g} = \omega(X_f,X_g) = \del[0]{i_{X_f}\omega}(X_g) = df(X_g) = X_g f.
	\end{equation*}
\end{proof}

\begin{definition}[Integral of Motion]
	Let $(M,\omega,H)$ be a Hamiltonian system. A function $f \in C^\infty(M)$ is said to be an \bld{integral of motion for the Hamiltonian system $(M,\omega,H)$}\index{Integral of motion}, iff $\cbr{H,f} = 0$.
\end{definition}

Let us recall some basic facts from the theory of Lie groups and Lie algebras. A \emph{Lie group} is defined to be a group $(G,\cdot)$, such that $G$ is a smooth manifold and the multiplication $\cdot$ as well as the inversion map $\cdot^{-1} : G \to G$ defined by $g\mapsto g^{-1}$ are smooth. If $G$ is a Lie group, we can associate to $G$ its \emph{Lie algebra} $\mathfrak{g}$ defined to be $\mathfrak{g} := T_eG$, where $e$ denotes the neutral element of $G$. It can be shown that $\mathfrak{g} \cong \mathfrak{X}_L(G)$ as real vector spaces, where $\mathfrak{X}_L(G) \subseteq \mathfrak{X}(G)$ denotes the space of \emph{left invariant vector fields on $G$}, that is, the vector fields $X \in \mathfrak{X}(G)$ satisfying $(L_g)_* X = X$, where $L_g$ is the diffeomorphism $L_g : G \to G$ defined by $L_g(h) := gh$ and $(L_g)_*$ is the \emph{pushforward of $X$} defined to be the vector field $\del[1]{(L_g)_*X}_h := d(L_g)_{g^{-1}h}X\vert_{g^{-1}h}$ for $h \in G$. Most importantly, any left invariant vector field on $G$ is complete and so we can define the \emph{exponential map} $\exp : \mathfrak{g} \to G$ by 
\begin{equation*}
	\exp v := \gamma(1),
\end{equation*}
\noindent where $\gamma \in C^\infty(\mathbb{R},G)$ is the integral curve of the \emph{left invariant vector field $X_v$ associated to $v$ on $G$}, that is $X_v\vert_g := d(L_g)_e(v)$, with starting point $\gamma(0) = e$. Then we have that $\gamma(t) = \exp tv$ and $(\exp tv)^{-1} = \exp (-tv)$ for all $v \in \mathfrak{g}$ and $t \in \mathbb{R}$.

The most important applications of Lie groups to smooth manifold theory involve actions by Lie groups on manifolds. Let $G$ be a Lie group and $M$ be a smooth manifold. A map in $C^\infty(G \times M,M)$ given by $(g,x) \mapsto g \cdot x$, is said to be a \emph{left action of $G$ on $M$} iff
\begin{equation*}
	g \cdot (h \cdot x) = (gh) \cdot x \qquad \text{and} \qquad e \cdot x = x
\end{equation*}
\noindent holds for all $g,h \in G$ and $x \in M$. Similarly, a \emph{right action of $G$ on $M$} is defined to be a map in $C^\infty(M \times G,M)$ given by $(x,g) \mapsto x \cdot g$ satisfying
\begin{equation*}
	(x \cdot g) \cdot h = x \cdot (gh) \qquad \text{and} \qquad x \cdot e = x
\end{equation*}
\noindent for all $g,h \in G$ and $x \in M$. Note that any left action of $G$ on $M$ can be transformed into a right action of $G$ on $M$ by defining $x \cdot g := g^{-1} \cdot x$ for all $g \in G$ and $x \in M$, and similarly every right action of $G$ on $M$ can  be transformed into a left action of $G$ on $M$. 

Suppose we are given a right action of a Lie group $G$ on a smooth manifold $M$. Then each element $v \in \mathfrak{g}$ determines a global flow on $M$ by
\begin{equation*}
	(t,x) \mapsto x \cdot \exp tv.
\end{equation*}
Define $\what{v} \in \mathfrak{X}(M)$ by
\begin{equation*}
	\what{v}_x := \frac{d}{dt}\bigg\vert_{t = 0} x \cdot \exp tv
\end{equation*}
\noindent for all $x \in M$. This is the \emph{infinitesimal generator} associated to the above flow (see \cite[210]{lee:smooth_manifolds:2013}). Hence we get a map $\mathfrak{g} \to \mathfrak{X}(M)$ defined by $v \mapsto \what{v}$. By \cite[526]{lee:smooth_manifolds:2013}, this map is actually a \emph{Lie algebra homomorphism}. This is the main reason we are working with right actions rather than left actions. 

Recall, that if $F \in C^\infty(M,N)$ for two smooth manifolds $M$ and $N$, for $x \in M$ we define the \emph{differential of $F$ at $x$} to be the mapping $dF_x : T_xM \to T_{F(x)}N$, given by $dF_x(v)(f) := v(f \circ F)$ for all $f \in C^\infty(N)$.

\begin{lemma}[{Computing the Differential Using a Velocity Vector \cite[70]{lee:smooth_manifolds:2013}}]
	\label{lem:computing_the_differential_using_a_velocity_vector}
	Let $F \in C^\infty(M,N)$ for two smooth manifolds $M$ and $N$, $x \in M$ and $v \in T_xM$. Then
	\begin{equation*}
		dF_x(v) = (F \circ \gamma)'(0)
	\end{equation*}
	\noindent for any path $\gamma \in C^\infty(J,M)$, where $J \subseteq \mathbb{R}$ is an interval such that $0 \in J$, $\gamma(0) = x$ and $\gamma'(0) = v$.
\end{lemma}

\begin{proposition}
	\label{prop:v_hat_action}
	Suppose we are given a right action of a Lie group $G$ on a smooth manifold $M$. Then for each $v \in \mathfrak{g}$, the infinitesimal generator $\what{v}$ associated to the flow generated by $v$ satisfies
	\begin{equation*}
		(\what{v}f)(x) = \frac{d}{dt}\bigg\vert_{t = 0} f(x \cdot \exp tv)
	\end{equation*}
	\noindent for all $x \in M$ and $f \in C^\infty(M)$.
\end{proposition}

\begin{proof}
	Let $x \in M$ and denote by $\theta : M \times G \to M$ the right action of $G$ on $M$. Define $\theta^x : G \to M$ by $\theta^x(g) := x \cdot g$. Then $\theta^x$ is smooth since
	\begin{equation*}
		\begin{tikzcd}
			\theta^x\colon \quad G \cong \cbr{x} \times G \arrow[r,hook] & M \times G \arrow[r,"\theta"] & M
		\end{tikzcd}
	\end{equation*}
	\noindent where the first two maps steem from \cite[100]{lee:smooth_manifolds:2013}. Set $\gamma(t) := \exp tv$ for all $t \in \mathbb{R}$. Then it is immediate, that
	\begin{equation*}
		x \cdot \exp tv = \theta^x\del[1]{\gamma(t)}.
	\end{equation*}
	Thus we compute
	\begin{align*}
		(\what{v}f)(x) &= \what{v}_xf\\
		&= \frac{d}{dt}\bigg\vert_{t = 0}\theta^x\del[1]{\gamma(t)} f\\ 
		&= d(\theta^x)_e(v)f & (\text{by lemma } \ref{lem:computing_the_differential_using_a_velocity_vector})\\
		&= v(f \circ \theta^x) & (\text{by definition of } d\theta^x)\\
		&= d(f \circ \theta^x)_e(v) & (\text{by definition of } d(f \circ \theta^x))\\
		&= \del[0]{f \circ \theta^x \circ \gamma}'(0) & (\text{by lemma } \ref{lem:computing_the_differential_using_a_velocity_vector})\\
		&= \frac{d}{dt}\bigg\vert_{t = 0} f(x \cdot \exp tv). 
	\end{align*}
\end{proof}

\begin{remark}
	From now on, we will consider left actions of Lie groups $G$ on smooth manifolds $M$ instead of right actions, since they are more common. This is however no drawback, since any left action can be converted into a right action. Hence if $v \in \mathfrak{g}$, the corresponding infinitesimal generator $V$ is given by
	\begin{equation*}
		\what{v}_x = \frac{d}{dt}\bigg\vert_{t = 0}\exp(-tv) \cdot x.
	\end{equation*}
\end{remark}

\begin{definition}[Hamiltonian Action]
	A left action of a Lie group $G$ on a symplectic manifold $(M,\omega)$ is said to be a \bld{Hamiltonian action of $G$ on $(M,\omega)$}\index{Hamiltonian!action}, iff for each $v \in \mathfrak{g}$, there exists a Hamiltonian system $(M,\omega,H_v)$, such that $X_{H_v} = \what{v}$. 	
\end{definition}

\begin{definition}[Symmetry Group]
	A \bld{symmetry group of a Hamiltonian system $(M,\omega,H)$}\index{Symmetry group} is defined to be a Lie group $G$, such that there exists a Hamiltonian action of $G$ on $(M,\omega)$, such that
	\begin{equation*}
		H(g \cdot x) = H(x)
	\end{equation*}
	\noindent holds for all $g \in G$ and $x \in M$.
\end{definition}

\begin{theorem}[Noether\index{Noether!'s theorem}]
	\label{thm:Noether_theorem}
	Let $G$ be a symmetry group of a Hamiltonian system $(M,\omega,H)$. Then for each $v \in \mathfrak{g}$, the function $H_v \in C^\infty(M)$ such that $X_{H_v} = \what{v}$ is an integral of motion.
\end{theorem}

\begin{proof}
	Let $x \in M$. We compute
	\begin{align*}
		\cbr[0]{H,H_v}(x) &= \del[1]{X_{H_v}H}(x) & (\text{by lemma } \ref{lem:Poisson_bracket_equivalent})\\
		&= (\what{v}H)(x)\\
		&= \frac{d}{dt}\bigg\vert_{t = 0}H\del[1]{\exp(-tv) \cdot x} & (\text{by proposition } \ref{prop:v_hat_action})\\
		&= \frac{d}{dt}\bigg\vert_{t = 0}H(x)\\
		&= 0.
	\end{align*}
\end{proof}
